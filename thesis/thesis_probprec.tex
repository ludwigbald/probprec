%%%%%%%%%%%%%%%%%%%%%%%%%%%%%%%%%%%%%%%%%%%%%%%%%%%%%%%%%%%%%%%%%%%%%%%%%%%%%
%%% LaTeX-Rahmen fuer das Erstellen von Masterarbeiten
%%%%%%%%%%%%%%%%%%%%%%%%%%%%%%%%%%%%%%%%%%%%%%%%%%%%%%%%%%%%%%%%%%%%%%%%%%%%%

%%%%%%%%%%%%%%%%%%%%%%%%%%%%%%%%%%%%%%%%%%%%%%%%%%%%%%%%%%%%%%%%%%%%%%%%%%%%%
%%% allgemeine Einstellungen
%%%%%%%%%%%%%%%%%%%%%%%%%%%%%%%%%%%%%%%%%%%%%%%%%%%%%%%%%%%%%%%%%%%%%%%%%%%%%

\documentclass[twoside,12pt,a4paper]{report}
%\usepackage{reportpage}
\usepackage{epsf}
\usepackage{graphics, graphicx}
\usepackage{latexsym}
\usepackage[margin=10pt,font=small,labelfont=bf]{caption}
\usepackage[utf8]{inputenc}
\usepackage[toc,page]{appendix}
\usepackage{todonotes}
\usepackage{markdown}
\usepackage[ngerman, english]{babel}

\textwidth 14cm
\textheight 22cm
\topmargin 0.0cm
\evensidemargin 1cm
\oddsidemargin 1cm
%\footskip 2cm
\parskip0.5explus0.1exminus0.1ex

% Kann von Student auch nach pers\"onlichem Geschmack ver\"andert werden.
\pagestyle{headings}

\sloppy

\begin{document}

%%%%%%%%%%%%%%%%%%%%%%%%%%%%%%%%%%%%%%%%%%%%%%%%%%%%%%%%%%%%%%%%%%%%%%%%%%%%
%%% hier steht die neue Titelseite 
%%%%%%%%%%%%%%%%%%%%%%%%%%%%%%%%%%%%%%%%%%%%%%%%%%%%%%%%%%%%%%%%%%%%%%%%%%%%

\begin{titlepage}
	\begin{center}
		{\LARGE Eberhard Karls Universit\"at T\"ubingen}\\
		{\large Mathematisch-Naturwissenschaftliche Fakult\"at \\
			Wilhelm-Schickard-Institut f\"ur Informatik\\[4cm]}
		{\huge Bachelor Thesis Cognitive Science\\[2cm]}
		{\Large\bf  Title of thesis\\[1.5cm]}
		{\large Ludwig Bald}\\[0.5cm]
		\today \\[4cm]
		\parbox{7cm}{\begin{center}{{\small\bf Gutachter}\\
				[0.5cm]\large Prof. Dr. Philipp Hennig}\\
				(Methoden des Maschinellen Lernens)\\
				{\footnotesize Wilhelm-Schickard-Institut f\"ur Informatik\\
				Universit\"at T\"ubingen}\end{center}}\hfill\parbox{7cm}
				{\begin{center}
				{\small\bf Betreuer}\\[0.5cm]
				{\large Filip De Roos}\\
				(Methoden des Maschinellen Lernens)\\
				{\footnotesize Wilhelm-Schickard-Institut f\"ur Informatik\\
					Universit\"at T\"ubingen}\end{center}
		}
	\end{center}
\end{titlepage}


%%%%%%%%%%%%%%%%%%%%%%%%%%%%%%%%%%%%%%%%%%%%%%%%%%%%%%%%%%%%%%%%%%%%%%%%%%%%
%%% Titelr"uckseite: Bibliographische Angaben
%%%%%%%%%%%%%%%%%%%%%%%%%%%%%%%%%%%%%%%%%%%%%%%%%%%%%%%%%%%%%%%%%%%%%%%%%%%%

\thispagestyle{empty}
\vspace*{\fill}
\begin{minipage}{11.2cm}
	\textbf{Bald, Ludwig:}\\
	\emph{Title of thesis}\\ Bachelor Thesis Cognitive Science\\
	Eberhard Karls Universit\"at T\"ubingen\\
	Thesis period: von-bis
\end{minipage}
\newpage

%%%%%%%%%%%%%%%%%%%%%%%%%%%%%%%%%%%%%%%%%%%%%%%%%%%%%%%%%%%%%%%%%%%%%%%%%%%%

\pagenumbering{roman}
\setcounter{page}{1}

%%%%%%%%%%%%%%%%%%%%%%%%%%%%%%%%%%%%%%%%%%%%%%%%%%%%%%%%%%%%%%%%%%%%%%%%%%%%
%%% Seite I: Zusammenfassug, Danksagung
%%%%%%%%%%%%%%%%%%%%%%%%%%%%%%%%%%%%%%%%%%%%%%%%%%%%%%%%%%%%%%%%%%%%%%%%%%%%


\section*{Abstract}
\todo[inline]{Note to the reader: This thesis is unfinished and I put zero confidence in its correctness, quality of citations or completeness.}
\todo{theabstract, citing!}
In machine learning, stochastic gradient descent is a widely used optimizsation algorithm, used to update the parameters of a model after a minibatch of data has been observed, in order to improve the model's predictions. It has been shown to converge much faster when the condition number (i.e. the ratio between the largest and the smallest eigenvalue) of ... is closer to 1. A preconditioner reduces the condition value.
Tho goal of this thesis was to reimplement the algorithm as an easy-to-use-class in python and take part in the development of DeepOBS, by being able to give feedback as a naive user of the benchmarking suite's features.
In this thesis I present my implementation of the probabilistic preconditioning algorithm proposed in \cite{de2019active}. I use DeepOBS \todo{cite} as a benchmarking toolbox, examining the effect of this kind of preconditioning on various optimizers and test problems. 
The results...

\newpage
\section*{Zusammenfassung}
\todo{Abstract auf Deutsch}

\newpage
\section*{Acknowledgments}
If you have someone to Acknowledge ;)
\todo{Aaron, Filip}
\cleardoublepage

%%%%%%%%%%%%%%%%%%%%%%%%%%%%%%%%%%%%%%%%%%%%%%%%%%%%%%%%%%%%%%%%%%%%%%%%%%%%%
%%% Table of Contents
%%%%%%%%%%%%%%%%%%%%%%%%%%%%%%%%%%%%%%%%%%%%%%%%%%%%%%%%%%%%%%%%%%%%%%%%%%%%%

\renewcommand{\baselinestretch}{1.3}
\small\normalsize

\tableofcontents

\renewcommand{\baselinestretch}{1}
\small\normalsize

\cleardoublepage

%%%%%%%%%%%%%%%%%%%%%%%%%%%%%%%%%%%%%%%%%%%%%%%%%%%%%%%%%%%%%%%%%%%%%%%%%%%%%
%%% Der Haupttext, ab hier mit arabischer Numerierung
%%% Mit \input{dateiname} werden die Datei `dateiname' eingebunden
%%%%%%%%%%%%%%%%%%%%%%%%%%%%%%%%%%%%%%%%%%%%%%%%%%%%%%%%%%%%%%%%%%%%%%%%%%%%%

\pagenumbering{arabic}
\setcounter{page}{1}


\chapter{Introduction}
What is this all about?
%Cite like this: \cite{agarwal2011}


\chapter{Fundamentals and Related Work}

\section{Deep learning}
\subsection{Artificial Neural Networks}
- Neural Networks exist. There are a few varieties and key developments, but they all share the underlying neural structure.
The original idea was to somewhat model biological neurons in order to mimic their capacity for structural memory.
As a mathematical object, a neuron is an activation function which takes on a certain value based on the weighted activations of the inputs (usually of their sum), which are either neurons themselves, or outside information given as an input to these neurons. Often there is a bias input, which is always 1 and allows for static activation.
Neurons are usually organized in layers, where each neuron has as input all the previous layers' neurons.

\subsection{Loss functions}
How do we recognize the optimal solution, or even tell which of two solutions is better?
Depending on the task that the model is meant to perform, we can define different so-called Loss Functions (also called "risk"). These are typically some sort of distance measure between the model's output and the desired, output.
However, there are other values we care about, such as the model's ability to generalize, i.e. how well it performs on data it has never seen before. If a model performs well on the trainig data set, but fails to generalize, this is called overfitting and means the model is not very useful on new data. The model memorizes the data, but doesn't get the underlying structure. To combat overfitting, people add a regularization term to the Loss function, which penalizes large parameters, which are usually an indicator that the model is overfitting.

\subsection{Regularization vs. weight decay}
If the model's parameters are too large, that often means it is overfitting. in order to discourage this, a regularization term is sometimes added to the loss function. Usually this is the L2 norm of the parameter vector, but sometimes the L1 norm is also used

\section{Optimization}
In order to find the optimal values for the parameters of a neural network (usually weights and biases), different methods have been proposed. For small optimization problems (small number of parameters), it is often possible to directly find the optimal values analytically. For large optimization problems, the analytical solution is computationally intractable. As a result, numerical algorithms have emerged. These iterative algorithms use the available limited information in order to gradually infer an approximation for the true best values for the parameters. The number of iterations needed until the algorithm reaches convergence (i.e. the new solution is no longer signigicantly better than the previous one) varies based on the implicit prior assumptions about the data that the algorithm makes. These optimization processes have "hyperparameters" (different from the models "parameters" that we want to optimize). A great deal of research has gone into these numerical optimization processes, both in the general case and more specifically, for deep learining.
For the most interesting class of optimizers, it is required to know the gradient at the current point in parameter space. In neural networks, this is achieved by Automatic Differentiation.

\subsection{Automatic Differentiation}
Loss functions in neural networks have an interesting structure, in that they are a composition of all the layerwise activation functions. This structure leads to the algorithm of automatic differentiation:
During training, the model (using its current value for its parameters) is fed with data from the data set. The loss function of this minibatch is computed. While this happens, the computations that take place are tracked and built into a graph structure. This is necessary, because the loss function in practice isn't defined in a closed-form way directly on the parameters, but as a composition of layerwise activation functions. This means that for every parameter tracked by the graph, we can infer its influence on the loss. The parameters need to be leaves of the tree, which means they don't themselves depend on something else. (provide example graph image here).
The gradient in the direction of an individual leaf (partial derivative) is then computed by applying the chain rule, starting from the output of the loss function. This is called the backward pass.

\subsection{Stochastic Gradient Descent}
Once the gradient at a certain point is known, we can use this information to update the parameters in a way that takes us closer to the solution. A popular family of algorithms is derived from SGD. SGD means: Just take a step in the direction of steepest gradient. Scale the step size by the steepness of the gradient. If the gradient is very steep, take a larger step. If it is small, take a smaller step, like a drunk student rolling down a hill and ending up on a local minimum.
SGD has only a single hyperparameter, the "learning rate" $\alpha$, which is multiplied with the step. Its optimal choice depends on the data, the model and is generally not obvious.
Traditional Gradient Descent uses the whole data set to compute the true gradient, which is computationally instractable for large datasets, the usual case in machine learning. Instead, we use a variation called SGD. We compute only an estimate for the true gradient by "minibatching", using only a few, for example 64 data points in the forward pass. This greatly improves convergence speed, as the required computations are much easier to perform. However, especially for smaller batch sizes, this adds noise to the system, meaning that our parameter update step points only roughly in the direction of the steepest actual gradient. It has also been shown to improve generalization of the model.
Many variants of SGD have been proposed, adding things like a momentum term or otherwise adapting the learning rate dynamically.

\subsection{Adaptive methods (First-Order overview?)}
-RMSprop
-Adam
- Conjugate gradient?

\subsection{Second Order Methods(related Methods)}
-Newton: Needs access to the Hessian.
-Quasi-Newton.
We can keep track of an estimate of the Hessian along the way.

\section{Preconditioning}
The performance of SGD does not only depend on the choice of the learning rate, but also on the structure of the Loss landscape. The gradient of the loss function is a jacobian matrix of partial derivatives. Imagine an optimization proble with two parameters and a loss landscape that looks like an ellipse. If the random starting point is chosen to be towards the direction of the longer symmetry axis, the gradient will be rather flat and SGD will take a long time to converge. With the same logic, a circular, bowl-shaped loss landscape is optimal.
If we start at the steep wall, SGD will converge quite quickly. If you know the explicit form of the Hessian matrix, you can see this problem by comparing its eigenvalues. A circular, bowl-shaped loss function will have only eigenvalues that are the same size. In the elliptical case, which has previously been reported to be the standard case in machine learning, at least one eigenvalue is much larger than the others.
The mathematical measure for this phenomenon is called the "condition number" and is defined as the ratio between the largest and the smallest eigenvalues. SGD converges much faster when the condition number is closer to 1. The largest eigenvalue is sometimes referred to as the "spectral radius".

Preconditioning is a way to reduce the condition number. A preconditioner is a matrix that rescales the other matrix in such a way that the ratio of eigenvalues approaches 1.
If the eigenvalues are known, this is quite easy. However, in deep learning, they are not known. In this thesis I present my implementation of an algorithm which aims to efficiently estimate the eigenvalues and construct a Preconditioner, while taking into account the noise caused by minibatching.

Condition number vs Spectral Radius

The PROBLEM:

Up until now, there was no easy way to make use of preconditioning in a noisy setting, such as minibatched deep learning. I present an implementation of Filips Algorithm in an easy-to-use python class and demonstrate its strengths and weaknesses.

\section{Something about Probabilistic things}
- Probablilty Distribution, normal distribution, multivariate Gaussian distribution

\section{Benchmarking}
\begin{markdown}
There are no standard established benchmarking protocols for new optimizers. It isn't even clear what measures to consider, or how they are to be measured. As a result, nobody knows which optimizers are actually good. And some bad optimizers will seem good.
DeepOBS is a solution to this problem, standardizing a protocol, providing benchmarks and standard test problems.

- Description and examples of previous optimizer plots.
- Short overview of how deepobs handles stuff:
    - Test problems:
      	- DeepOBS includes the most used standard datasets and a variety of neural network models to train. This ensures that everyone is evaluated on the same problem.
    - Tuning:
    	- Many optimizers have hyperparameters that greatly affect the optimizer's performance
 	    - These need to be tuned by running many settings seperately, for example in a grid search. The actual deepobs protocol isn't ready yet.
      - DeepOBS generates commands for the grid search.
    - Running:
      	- DeepOBS provides a standard way to run your optimizer, taking care of logging parameters and evaluating success measures.
    - Analyzing:
      	- DeepOBS provides the analyzer class, which is able to automatically generate matplotlib plots showing the results of your runs.
\end{markdown}

\section{Related Work}
\todo{in which other ways has this problem been adressed? (What even is the problem?)}


\chapter{The algorithm}
\section{Description of the algorithm}
The exact inner workings of the algorithm are described in more detail in \cite{de2019active}.
Here I will give an overview of the algorithm's structure and steps:
part 1: - Gather observations of the curvature in-place:
part 2: - Estimating the Hessian and construct the preconditioner
part 3: - Every step, re-scale the gradients by applying the preconditioner

\section{Modifications of the algorithm}
\subsection{Parameter groups}
Mainly due to technical reasons (see \todo{reference chapter}), support was added for parameter groups, but this also yields an interesting theoretical change. The algorithm already treated every parameter layer as an independent task for inversion (???), but estimated a global step size, which was the same for every parameter. With this modification, the algorithm is able to use larger step sizes for parameters that allow for it. In theory, this would allow for faster learning in the direction of those parameters. However, the time to reach total convergence relies on the slowest parameter, not on the fastest. The benefit of this approach in practice remanins to be tested.
\subsection{Automatic Assessment of the Hessian's quality}
In the original algorithm, the Hessian is arbitrarily re-estimated every epoch. Hessian re-estimation is needed, as demonstrated in the original paper, alpha changes over time. The goal of this modification was to expose a measure of how useful/correct the estimated Hessian is, after observing a bit of data. A new estimation proces would be started if the current estimate was worse than a certain threshold. Multiple approaches were tried. A key point of trouble was the variance-less estimate of the Hessian. The original paper mentions that the Hessian is taken as the mean of a multivariate Gaussian distribution, but it fails to take this distribution's variance into account. Instead, it just assumes this Hessian is the optimal choice.
- Comparing the predicted gradient to the actual observed gradient.

- Maybe: Automatic restart of the estimation process, once the old Hessian is determined to be out of date.

\subsection{Getting rid of preconditioning}
For experiment x, it was important to

\chapter{Implementation/Methods}
\todo{High-Level to low-level details}

\section{Documentation for the class Preconditioner}
\subsection{Methods}
\subsection{Implementation Details}
\begin{markdown}
In the following, I will highlight and explain some key software design decisions I took while implementing and refactoring the algorithm.

* Self-contained class with minimal interface, while marking all the internal functions as hidden
 * Being able to restart the estimation process from outside, without having to do a new instance of anything
 - Class inherits from torch.optim.Optimizer, because that makes it easier to use and easier to investigate using DeepOBS.
 - The class functions as a wrapper for other optimizer classes, using the decorator pattern on the `step()` function. This means other optimizers can be used.
 - All the parameter-wise variables and values are stored in the state dict, which means we can save and load backups of the optimizer state.
 - There is support for parameter groups, which all pytorch Optimizers should exhibit.
 - external logging of internal variables can be done using the `get_log()` method
 - Create different versions of the class by subclassing. This allows me to reuse code. For example, by knocking out the preconditioning behavior.
\end{markdown}	


\section{Realization of the Test problems}
The test problems used were deep learning models.
- mnist/fmnist
- cifar10
- quadratic\_deep

\section{Technical details}
The experiments were run on the TCML cluster at the University of Tübingen.
A Singularity container was set up on Ubuntu 16.4 LTS with python 3.5, pytorch (version) and DeepOBS (see Appendix for Singulariy recipe).
Computation was distributed over multiple GPU compute nodes using the workload manager Slurm.


\chapter{Experiment}
\section{Experiment x: Preconditioning}
In order to investigate the performance benefit of the Preconditioning behavior, the Preconditioner class was adapted to have two different versions, behaving almost exactly the same. They both compute the Hessian, they both estimate a learning rate. However only in one version the Preconditioner is applied, while in the other one it isn't. This was achieved by subclassing and overwriting the \_apply\_precenditioner() method with an empty method. Hyperparameters: Num\_observations 10, prior\_iterations 5, est\_rank 2, optim\_class torch.SGD

% Redraw %% Creator: Matplotlib, PGF backend
%%
%% To include the figure in your LaTeX document, write
%%   \input{<filename>.pgf}
%%
%% Make sure the required packages are loaded in your preamble
%%   \usepackage{pgf}
%%
%% Figures using additional raster images can only be included by \input if
%% they are in the same directory as the main LaTeX file. For loading figures
%% from other directories you can use the `import` package
%%   \usepackage{import}
%% and then include the figures with
%%   \import{<path to file>}{<filename>.pgf}
%%
%% Matplotlib used the following preamble
%%   \usepackage{fontspec}
%%   \setmainfont{DejaVuSerif.ttf}[Path=/home/ludwig/anaconda3/lib/python3.7/site-packages/matplotlib/mpl-data/fonts/ttf/]
%%   \setsansfont{DejaVuSans.ttf}[Path=/home/ludwig/anaconda3/lib/python3.7/site-packages/matplotlib/mpl-data/fonts/ttf/]
%%   \setmonofont{DejaVuSansMono.ttf}[Path=/home/ludwig/anaconda3/lib/python3.7/site-packages/matplotlib/mpl-data/fonts/ttf/]
%%
\begingroup%
\makeatletter%
\begin{pgfpicture}%
\pgfpathrectangle{\pgfpointorigin}{\pgfqpoint{8.430000in}{18.390000in}}%
\pgfusepath{use as bounding box, clip}%
\begin{pgfscope}%
\pgfsetbuttcap%
\pgfsetmiterjoin%
\definecolor{currentfill}{rgb}{1.000000,1.000000,1.000000}%
\pgfsetfillcolor{currentfill}%
\pgfsetlinewidth{0.000000pt}%
\definecolor{currentstroke}{rgb}{1.000000,1.000000,1.000000}%
\pgfsetstrokecolor{currentstroke}%
\pgfsetdash{}{0pt}%
\pgfpathmoveto{\pgfqpoint{0.000000in}{0.000000in}}%
\pgfpathlineto{\pgfqpoint{8.430000in}{0.000000in}}%
\pgfpathlineto{\pgfqpoint{8.430000in}{18.390000in}}%
\pgfpathlineto{\pgfqpoint{0.000000in}{18.390000in}}%
\pgfpathclose%
\pgfusepath{fill}%
\end{pgfscope}%
\begin{pgfscope}%
\pgfsetbuttcap%
\pgfsetmiterjoin%
\definecolor{currentfill}{rgb}{1.000000,1.000000,1.000000}%
\pgfsetfillcolor{currentfill}%
\pgfsetlinewidth{0.000000pt}%
\definecolor{currentstroke}{rgb}{0.000000,0.000000,0.000000}%
\pgfsetstrokecolor{currentstroke}%
\pgfsetstrokeopacity{0.000000}%
\pgfsetdash{}{0pt}%
\pgfpathmoveto{\pgfqpoint{1.053750in}{13.104874in}}%
\pgfpathlineto{\pgfqpoint{7.587000in}{13.104874in}}%
\pgfpathlineto{\pgfqpoint{7.587000in}{16.183200in}}%
\pgfpathlineto{\pgfqpoint{1.053750in}{16.183200in}}%
\pgfpathclose%
\pgfusepath{fill}%
\end{pgfscope}%
\begin{pgfscope}%
\pgfpathrectangle{\pgfqpoint{1.053750in}{13.104874in}}{\pgfqpoint{6.533250in}{3.078326in}}%
\pgfusepath{clip}%
\pgfsetbuttcap%
\pgfsetroundjoin%
\definecolor{currentfill}{rgb}{0.121569,0.466667,0.705882}%
\pgfsetfillcolor{currentfill}%
\pgfsetfillopacity{0.300000}%
\pgfsetlinewidth{0.000000pt}%
\definecolor{currentstroke}{rgb}{0.000000,0.000000,0.000000}%
\pgfsetstrokecolor{currentstroke}%
\pgfsetdash{}{0pt}%
\pgfpathmoveto{\pgfqpoint{-3231.325327in}{16.134803in}}%
\pgfpathlineto{\pgfqpoint{-3231.325327in}{16.134803in}}%
\pgfpathlineto{\pgfqpoint{1.053750in}{14.421922in}}%
\pgfpathlineto{\pgfqpoint{2.026793in}{14.269008in}}%
\pgfpathlineto{\pgfqpoint{2.595987in}{14.111725in}}%
\pgfpathlineto{\pgfqpoint{2.999836in}{14.097315in}}%
\pgfpathlineto{\pgfqpoint{3.313086in}{14.030784in}}%
\pgfpathlineto{\pgfqpoint{3.569030in}{14.002687in}}%
\pgfpathlineto{\pgfqpoint{3.785427in}{14.002251in}}%
\pgfpathlineto{\pgfqpoint{3.972879in}{13.978332in}}%
\pgfpathlineto{\pgfqpoint{4.138224in}{13.971930in}}%
\pgfpathlineto{\pgfqpoint{4.286129in}{13.936451in}}%
\pgfpathlineto{\pgfqpoint{4.419926in}{13.929064in}}%
\pgfpathlineto{\pgfqpoint{4.542073in}{13.911774in}}%
\pgfpathlineto{\pgfqpoint{4.654437in}{13.906638in}}%
\pgfpathlineto{\pgfqpoint{4.758470in}{13.915283in}}%
\pgfpathlineto{\pgfqpoint{4.855323in}{13.880754in}}%
\pgfpathlineto{\pgfqpoint{4.945922in}{13.906506in}}%
\pgfpathlineto{\pgfqpoint{5.031027in}{13.876820in}}%
\pgfpathlineto{\pgfqpoint{5.111267in}{13.871350in}}%
\pgfpathlineto{\pgfqpoint{5.187166in}{13.852461in}}%
\pgfpathlineto{\pgfqpoint{5.259172in}{13.848651in}}%
\pgfpathlineto{\pgfqpoint{5.327664in}{13.849556in}}%
\pgfpathlineto{\pgfqpoint{5.392969in}{13.842636in}}%
\pgfpathlineto{\pgfqpoint{5.455371in}{13.838238in}}%
\pgfpathlineto{\pgfqpoint{5.515116in}{13.830079in}}%
\pgfpathlineto{\pgfqpoint{5.572422in}{13.820083in}}%
\pgfpathlineto{\pgfqpoint{5.627480in}{13.823478in}}%
\pgfpathlineto{\pgfqpoint{5.680460in}{13.821814in}}%
\pgfpathlineto{\pgfqpoint{5.731513in}{13.840577in}}%
\pgfpathlineto{\pgfqpoint{5.780775in}{13.816940in}}%
\pgfpathlineto{\pgfqpoint{5.828366in}{13.838488in}}%
\pgfpathlineto{\pgfqpoint{5.874396in}{13.801332in}}%
\pgfpathlineto{\pgfqpoint{5.918965in}{13.802920in}}%
\pgfpathlineto{\pgfqpoint{5.962163in}{13.807718in}}%
\pgfpathlineto{\pgfqpoint{6.004070in}{13.788165in}}%
\pgfpathlineto{\pgfqpoint{6.044763in}{13.794107in}}%
\pgfpathlineto{\pgfqpoint{6.084310in}{13.786459in}}%
\pgfpathlineto{\pgfqpoint{6.122772in}{13.779426in}}%
\pgfpathlineto{\pgfqpoint{6.160209in}{13.783188in}}%
\pgfpathlineto{\pgfqpoint{6.196674in}{13.784505in}}%
\pgfpathlineto{\pgfqpoint{6.232215in}{13.784600in}}%
\pgfpathlineto{\pgfqpoint{6.266879in}{13.771615in}}%
\pgfpathlineto{\pgfqpoint{6.300707in}{13.777780in}}%
\pgfpathlineto{\pgfqpoint{6.333739in}{13.774221in}}%
\pgfpathlineto{\pgfqpoint{6.366012in}{13.778250in}}%
\pgfpathlineto{\pgfqpoint{6.397560in}{13.778362in}}%
\pgfpathlineto{\pgfqpoint{6.428414in}{13.766382in}}%
\pgfpathlineto{\pgfqpoint{6.458604in}{13.775802in}}%
\pgfpathlineto{\pgfqpoint{6.488159in}{13.761418in}}%
\pgfpathlineto{\pgfqpoint{6.517104in}{13.763389in}}%
\pgfpathlineto{\pgfqpoint{6.545465in}{13.757323in}}%
\pgfpathlineto{\pgfqpoint{6.573264in}{13.758388in}}%
\pgfpathlineto{\pgfqpoint{6.600523in}{13.763190in}}%
\pgfpathlineto{\pgfqpoint{6.627263in}{13.759488in}}%
\pgfpathlineto{\pgfqpoint{6.653503in}{13.754991in}}%
\pgfpathlineto{\pgfqpoint{6.679262in}{13.755148in}}%
\pgfpathlineto{\pgfqpoint{6.704556in}{13.759556in}}%
\pgfpathlineto{\pgfqpoint{6.729403in}{13.750747in}}%
\pgfpathlineto{\pgfqpoint{6.753818in}{13.750970in}}%
\pgfpathlineto{\pgfqpoint{6.777815in}{13.754082in}}%
\pgfpathlineto{\pgfqpoint{6.801409in}{13.743692in}}%
\pgfpathlineto{\pgfqpoint{6.824613in}{13.749743in}}%
\pgfpathlineto{\pgfqpoint{6.847439in}{13.751467in}}%
\pgfpathlineto{\pgfqpoint{6.869901in}{13.744740in}}%
\pgfpathlineto{\pgfqpoint{6.892008in}{13.751813in}}%
\pgfpathlineto{\pgfqpoint{6.913773in}{13.754455in}}%
\pgfpathlineto{\pgfqpoint{6.935206in}{13.732955in}}%
\pgfpathlineto{\pgfqpoint{6.956316in}{13.743991in}}%
\pgfpathlineto{\pgfqpoint{6.977113in}{13.747701in}}%
\pgfpathlineto{\pgfqpoint{6.997607in}{13.746901in}}%
\pgfpathlineto{\pgfqpoint{7.017806in}{13.739523in}}%
\pgfpathlineto{\pgfqpoint{7.037719in}{13.737344in}}%
\pgfpathlineto{\pgfqpoint{7.057353in}{13.763152in}}%
\pgfpathlineto{\pgfqpoint{7.076716in}{13.761882in}}%
\pgfpathlineto{\pgfqpoint{7.095816in}{13.734531in}}%
\pgfpathlineto{\pgfqpoint{7.114659in}{13.729348in}}%
\pgfpathlineto{\pgfqpoint{7.133253in}{13.730260in}}%
\pgfpathlineto{\pgfqpoint{7.151603in}{13.743141in}}%
\pgfpathlineto{\pgfqpoint{7.169717in}{13.793916in}}%
\pgfpathlineto{\pgfqpoint{7.187600in}{13.730551in}}%
\pgfpathlineto{\pgfqpoint{7.205258in}{13.734194in}}%
\pgfpathlineto{\pgfqpoint{7.222697in}{13.732417in}}%
\pgfpathlineto{\pgfqpoint{7.239922in}{13.732174in}}%
\pgfpathlineto{\pgfqpoint{7.256938in}{13.726038in}}%
\pgfpathlineto{\pgfqpoint{7.273750in}{13.738364in}}%
\pgfpathlineto{\pgfqpoint{7.290363in}{13.729418in}}%
\pgfpathlineto{\pgfqpoint{7.306782in}{13.726234in}}%
\pgfpathlineto{\pgfqpoint{7.323011in}{13.735800in}}%
\pgfpathlineto{\pgfqpoint{7.339055in}{13.731308in}}%
\pgfpathlineto{\pgfqpoint{7.354917in}{13.733272in}}%
\pgfpathlineto{\pgfqpoint{7.370603in}{13.731179in}}%
\pgfpathlineto{\pgfqpoint{7.386114in}{13.729148in}}%
\pgfpathlineto{\pgfqpoint{7.401457in}{13.725663in}}%
\pgfpathlineto{\pgfqpoint{7.416633in}{13.728375in}}%
\pgfpathlineto{\pgfqpoint{7.431647in}{13.727352in}}%
\pgfpathlineto{\pgfqpoint{7.446502in}{13.730074in}}%
\pgfpathlineto{\pgfqpoint{7.461202in}{13.725173in}}%
\pgfpathlineto{\pgfqpoint{7.475749in}{13.722238in}}%
\pgfpathlineto{\pgfqpoint{7.490148in}{13.732176in}}%
\pgfpathlineto{\pgfqpoint{7.504399in}{13.724361in}}%
\pgfpathlineto{\pgfqpoint{7.518508in}{13.727216in}}%
\pgfpathlineto{\pgfqpoint{7.518508in}{13.727216in}}%
\pgfpathlineto{\pgfqpoint{7.518508in}{13.727216in}}%
\pgfpathlineto{\pgfqpoint{7.504399in}{13.724361in}}%
\pgfpathlineto{\pgfqpoint{7.490148in}{13.732176in}}%
\pgfpathlineto{\pgfqpoint{7.475749in}{13.722238in}}%
\pgfpathlineto{\pgfqpoint{7.461202in}{13.725173in}}%
\pgfpathlineto{\pgfqpoint{7.446502in}{13.730074in}}%
\pgfpathlineto{\pgfqpoint{7.431647in}{13.727352in}}%
\pgfpathlineto{\pgfqpoint{7.416633in}{13.728375in}}%
\pgfpathlineto{\pgfqpoint{7.401457in}{13.725663in}}%
\pgfpathlineto{\pgfqpoint{7.386114in}{13.729148in}}%
\pgfpathlineto{\pgfqpoint{7.370603in}{13.731179in}}%
\pgfpathlineto{\pgfqpoint{7.354917in}{13.733272in}}%
\pgfpathlineto{\pgfqpoint{7.339055in}{13.731308in}}%
\pgfpathlineto{\pgfqpoint{7.323011in}{13.735800in}}%
\pgfpathlineto{\pgfqpoint{7.306782in}{13.726234in}}%
\pgfpathlineto{\pgfqpoint{7.290363in}{13.729418in}}%
\pgfpathlineto{\pgfqpoint{7.273750in}{13.738364in}}%
\pgfpathlineto{\pgfqpoint{7.256938in}{13.726038in}}%
\pgfpathlineto{\pgfqpoint{7.239922in}{13.732174in}}%
\pgfpathlineto{\pgfqpoint{7.222697in}{13.732417in}}%
\pgfpathlineto{\pgfqpoint{7.205258in}{13.734194in}}%
\pgfpathlineto{\pgfqpoint{7.187600in}{13.730551in}}%
\pgfpathlineto{\pgfqpoint{7.169717in}{13.793916in}}%
\pgfpathlineto{\pgfqpoint{7.151603in}{13.743141in}}%
\pgfpathlineto{\pgfqpoint{7.133253in}{13.730260in}}%
\pgfpathlineto{\pgfqpoint{7.114659in}{13.729348in}}%
\pgfpathlineto{\pgfqpoint{7.095816in}{13.734531in}}%
\pgfpathlineto{\pgfqpoint{7.076716in}{13.761882in}}%
\pgfpathlineto{\pgfqpoint{7.057353in}{13.763152in}}%
\pgfpathlineto{\pgfqpoint{7.037719in}{13.737344in}}%
\pgfpathlineto{\pgfqpoint{7.017806in}{13.739523in}}%
\pgfpathlineto{\pgfqpoint{6.997607in}{13.746901in}}%
\pgfpathlineto{\pgfqpoint{6.977113in}{13.747701in}}%
\pgfpathlineto{\pgfqpoint{6.956316in}{13.743991in}}%
\pgfpathlineto{\pgfqpoint{6.935206in}{13.732955in}}%
\pgfpathlineto{\pgfqpoint{6.913773in}{13.754455in}}%
\pgfpathlineto{\pgfqpoint{6.892008in}{13.751813in}}%
\pgfpathlineto{\pgfqpoint{6.869901in}{13.744740in}}%
\pgfpathlineto{\pgfqpoint{6.847439in}{13.751467in}}%
\pgfpathlineto{\pgfqpoint{6.824613in}{13.749743in}}%
\pgfpathlineto{\pgfqpoint{6.801409in}{13.743692in}}%
\pgfpathlineto{\pgfqpoint{6.777815in}{13.754082in}}%
\pgfpathlineto{\pgfqpoint{6.753818in}{13.750970in}}%
\pgfpathlineto{\pgfqpoint{6.729403in}{13.750747in}}%
\pgfpathlineto{\pgfqpoint{6.704556in}{13.759556in}}%
\pgfpathlineto{\pgfqpoint{6.679262in}{13.755148in}}%
\pgfpathlineto{\pgfqpoint{6.653503in}{13.754991in}}%
\pgfpathlineto{\pgfqpoint{6.627263in}{13.759488in}}%
\pgfpathlineto{\pgfqpoint{6.600523in}{13.763190in}}%
\pgfpathlineto{\pgfqpoint{6.573264in}{13.758388in}}%
\pgfpathlineto{\pgfqpoint{6.545465in}{13.757323in}}%
\pgfpathlineto{\pgfqpoint{6.517104in}{13.763389in}}%
\pgfpathlineto{\pgfqpoint{6.488159in}{13.761418in}}%
\pgfpathlineto{\pgfqpoint{6.458604in}{13.775802in}}%
\pgfpathlineto{\pgfqpoint{6.428414in}{13.766382in}}%
\pgfpathlineto{\pgfqpoint{6.397560in}{13.778362in}}%
\pgfpathlineto{\pgfqpoint{6.366012in}{13.778250in}}%
\pgfpathlineto{\pgfqpoint{6.333739in}{13.774221in}}%
\pgfpathlineto{\pgfqpoint{6.300707in}{13.777780in}}%
\pgfpathlineto{\pgfqpoint{6.266879in}{13.771615in}}%
\pgfpathlineto{\pgfqpoint{6.232215in}{13.784600in}}%
\pgfpathlineto{\pgfqpoint{6.196674in}{13.784505in}}%
\pgfpathlineto{\pgfqpoint{6.160209in}{13.783188in}}%
\pgfpathlineto{\pgfqpoint{6.122772in}{13.779426in}}%
\pgfpathlineto{\pgfqpoint{6.084310in}{13.786459in}}%
\pgfpathlineto{\pgfqpoint{6.044763in}{13.794107in}}%
\pgfpathlineto{\pgfqpoint{6.004070in}{13.788165in}}%
\pgfpathlineto{\pgfqpoint{5.962163in}{13.807718in}}%
\pgfpathlineto{\pgfqpoint{5.918965in}{13.802920in}}%
\pgfpathlineto{\pgfqpoint{5.874396in}{13.801332in}}%
\pgfpathlineto{\pgfqpoint{5.828366in}{13.838488in}}%
\pgfpathlineto{\pgfqpoint{5.780775in}{13.816940in}}%
\pgfpathlineto{\pgfqpoint{5.731513in}{13.840577in}}%
\pgfpathlineto{\pgfqpoint{5.680460in}{13.821814in}}%
\pgfpathlineto{\pgfqpoint{5.627480in}{13.823478in}}%
\pgfpathlineto{\pgfqpoint{5.572422in}{13.820083in}}%
\pgfpathlineto{\pgfqpoint{5.515116in}{13.830079in}}%
\pgfpathlineto{\pgfqpoint{5.455371in}{13.838238in}}%
\pgfpathlineto{\pgfqpoint{5.392969in}{13.842636in}}%
\pgfpathlineto{\pgfqpoint{5.327664in}{13.849556in}}%
\pgfpathlineto{\pgfqpoint{5.259172in}{13.848651in}}%
\pgfpathlineto{\pgfqpoint{5.187166in}{13.852461in}}%
\pgfpathlineto{\pgfqpoint{5.111267in}{13.871350in}}%
\pgfpathlineto{\pgfqpoint{5.031027in}{13.876820in}}%
\pgfpathlineto{\pgfqpoint{4.945922in}{13.906506in}}%
\pgfpathlineto{\pgfqpoint{4.855323in}{13.880754in}}%
\pgfpathlineto{\pgfqpoint{4.758470in}{13.915283in}}%
\pgfpathlineto{\pgfqpoint{4.654437in}{13.906638in}}%
\pgfpathlineto{\pgfqpoint{4.542073in}{13.911774in}}%
\pgfpathlineto{\pgfqpoint{4.419926in}{13.929064in}}%
\pgfpathlineto{\pgfqpoint{4.286129in}{13.936451in}}%
\pgfpathlineto{\pgfqpoint{4.138224in}{13.971930in}}%
\pgfpathlineto{\pgfqpoint{3.972879in}{13.978332in}}%
\pgfpathlineto{\pgfqpoint{3.785427in}{14.002251in}}%
\pgfpathlineto{\pgfqpoint{3.569030in}{14.002687in}}%
\pgfpathlineto{\pgfqpoint{3.313086in}{14.030784in}}%
\pgfpathlineto{\pgfqpoint{2.999836in}{14.097315in}}%
\pgfpathlineto{\pgfqpoint{2.595987in}{14.111725in}}%
\pgfpathlineto{\pgfqpoint{2.026793in}{14.269008in}}%
\pgfpathlineto{\pgfqpoint{1.053750in}{14.421922in}}%
\pgfpathlineto{\pgfqpoint{-3231.325327in}{16.134803in}}%
\pgfpathclose%
\pgfusepath{fill}%
\end{pgfscope}%
\begin{pgfscope}%
\pgfpathrectangle{\pgfqpoint{1.053750in}{13.104874in}}{\pgfqpoint{6.533250in}{3.078326in}}%
\pgfusepath{clip}%
\pgfsetbuttcap%
\pgfsetroundjoin%
\definecolor{currentfill}{rgb}{1.000000,0.498039,0.054902}%
\pgfsetfillcolor{currentfill}%
\pgfsetfillopacity{0.300000}%
\pgfsetlinewidth{0.000000pt}%
\definecolor{currentstroke}{rgb}{0.000000,0.000000,0.000000}%
\pgfsetstrokecolor{currentstroke}%
\pgfsetdash{}{0pt}%
\pgfpathmoveto{\pgfqpoint{-3231.325327in}{16.134803in}}%
\pgfpathlineto{\pgfqpoint{-3231.325327in}{16.134803in}}%
\pgfpathlineto{\pgfqpoint{1.053750in}{14.466482in}}%
\pgfpathlineto{\pgfqpoint{2.026793in}{14.249408in}}%
\pgfpathlineto{\pgfqpoint{2.595987in}{14.124231in}}%
\pgfpathlineto{\pgfqpoint{2.999836in}{14.142269in}}%
\pgfpathlineto{\pgfqpoint{3.313086in}{14.046464in}}%
\pgfpathlineto{\pgfqpoint{3.569030in}{13.997126in}}%
\pgfpathlineto{\pgfqpoint{3.785427in}{13.977050in}}%
\pgfpathlineto{\pgfqpoint{3.972879in}{13.989187in}}%
\pgfpathlineto{\pgfqpoint{4.138224in}{13.948632in}}%
\pgfpathlineto{\pgfqpoint{4.286129in}{13.986830in}}%
\pgfpathlineto{\pgfqpoint{4.419926in}{13.900301in}}%
\pgfpathlineto{\pgfqpoint{4.542073in}{13.887524in}}%
\pgfpathlineto{\pgfqpoint{4.654437in}{13.887533in}}%
\pgfpathlineto{\pgfqpoint{4.758470in}{13.974440in}}%
\pgfpathlineto{\pgfqpoint{4.855323in}{13.837011in}}%
\pgfpathlineto{\pgfqpoint{4.945922in}{13.897508in}}%
\pgfpathlineto{\pgfqpoint{5.031027in}{13.834651in}}%
\pgfpathlineto{\pgfqpoint{5.111267in}{13.895571in}}%
\pgfpathlineto{\pgfqpoint{5.187166in}{13.789134in}}%
\pgfpathlineto{\pgfqpoint{5.259172in}{13.811724in}}%
\pgfpathlineto{\pgfqpoint{5.327664in}{13.799627in}}%
\pgfpathlineto{\pgfqpoint{5.392969in}{13.800542in}}%
\pgfpathlineto{\pgfqpoint{5.455371in}{13.838366in}}%
\pgfpathlineto{\pgfqpoint{5.515116in}{13.758504in}}%
\pgfpathlineto{\pgfqpoint{5.572422in}{13.750309in}}%
\pgfpathlineto{\pgfqpoint{5.627480in}{13.753697in}}%
\pgfpathlineto{\pgfqpoint{5.680460in}{13.866869in}}%
\pgfpathlineto{\pgfqpoint{5.731513in}{13.771636in}}%
\pgfpathlineto{\pgfqpoint{5.780775in}{13.751210in}}%
\pgfpathlineto{\pgfqpoint{5.828366in}{14.040861in}}%
\pgfpathlineto{\pgfqpoint{5.874396in}{13.721999in}}%
\pgfpathlineto{\pgfqpoint{5.918965in}{13.793624in}}%
\pgfpathlineto{\pgfqpoint{5.962163in}{13.736080in}}%
\pgfpathlineto{\pgfqpoint{6.004070in}{13.731793in}}%
\pgfpathlineto{\pgfqpoint{6.044763in}{13.711731in}}%
\pgfpathlineto{\pgfqpoint{6.084310in}{13.747047in}}%
\pgfpathlineto{\pgfqpoint{6.122772in}{13.721604in}}%
\pgfpathlineto{\pgfqpoint{6.160209in}{13.731511in}}%
\pgfpathlineto{\pgfqpoint{6.196674in}{13.728507in}}%
\pgfpathlineto{\pgfqpoint{6.232215in}{13.750739in}}%
\pgfpathlineto{\pgfqpoint{6.266879in}{13.723556in}}%
\pgfpathlineto{\pgfqpoint{6.300707in}{13.717569in}}%
\pgfpathlineto{\pgfqpoint{6.333739in}{13.692370in}}%
\pgfpathlineto{\pgfqpoint{6.366012in}{13.773247in}}%
\pgfpathlineto{\pgfqpoint{6.397560in}{13.713948in}}%
\pgfpathlineto{\pgfqpoint{6.428414in}{13.732359in}}%
\pgfpathlineto{\pgfqpoint{6.458604in}{13.720245in}}%
\pgfpathlineto{\pgfqpoint{6.488159in}{13.694728in}}%
\pgfpathlineto{\pgfqpoint{6.517104in}{13.785864in}}%
\pgfpathlineto{\pgfqpoint{6.545465in}{13.681186in}}%
\pgfpathlineto{\pgfqpoint{6.573264in}{13.750737in}}%
\pgfpathlineto{\pgfqpoint{6.600523in}{13.854592in}}%
\pgfpathlineto{\pgfqpoint{6.627263in}{13.728390in}}%
\pgfpathlineto{\pgfqpoint{6.653503in}{13.706973in}}%
\pgfpathlineto{\pgfqpoint{6.679262in}{13.708523in}}%
\pgfpathlineto{\pgfqpoint{6.704556in}{13.739504in}}%
\pgfpathlineto{\pgfqpoint{6.729403in}{13.730545in}}%
\pgfpathlineto{\pgfqpoint{6.753818in}{13.727676in}}%
\pgfpathlineto{\pgfqpoint{6.777815in}{13.722369in}}%
\pgfpathlineto{\pgfqpoint{6.801409in}{13.702546in}}%
\pgfpathlineto{\pgfqpoint{6.824613in}{13.717024in}}%
\pgfpathlineto{\pgfqpoint{6.847439in}{13.745017in}}%
\pgfpathlineto{\pgfqpoint{6.869901in}{13.706849in}}%
\pgfpathlineto{\pgfqpoint{6.892008in}{13.741721in}}%
\pgfpathlineto{\pgfqpoint{6.913773in}{13.737361in}}%
\pgfpathlineto{\pgfqpoint{6.935206in}{13.716396in}}%
\pgfpathlineto{\pgfqpoint{6.956316in}{13.739695in}}%
\pgfpathlineto{\pgfqpoint{6.977113in}{13.751541in}}%
\pgfpathlineto{\pgfqpoint{6.997607in}{13.747937in}}%
\pgfpathlineto{\pgfqpoint{7.017806in}{13.757510in}}%
\pgfpathlineto{\pgfqpoint{7.037719in}{13.735992in}}%
\pgfpathlineto{\pgfqpoint{7.057353in}{13.750420in}}%
\pgfpathlineto{\pgfqpoint{7.076716in}{13.733108in}}%
\pgfpathlineto{\pgfqpoint{7.095816in}{13.738131in}}%
\pgfpathlineto{\pgfqpoint{7.114659in}{13.776994in}}%
\pgfpathlineto{\pgfqpoint{7.133253in}{13.739636in}}%
\pgfpathlineto{\pgfqpoint{7.151603in}{13.797421in}}%
\pgfpathlineto{\pgfqpoint{7.169717in}{13.862936in}}%
\pgfpathlineto{\pgfqpoint{7.187600in}{13.740754in}}%
\pgfpathlineto{\pgfqpoint{7.205258in}{13.805589in}}%
\pgfpathlineto{\pgfqpoint{7.222697in}{13.749826in}}%
\pgfpathlineto{\pgfqpoint{7.239922in}{13.792793in}}%
\pgfpathlineto{\pgfqpoint{7.256938in}{13.745465in}}%
\pgfpathlineto{\pgfqpoint{7.273750in}{13.766098in}}%
\pgfpathlineto{\pgfqpoint{7.290363in}{13.772429in}}%
\pgfpathlineto{\pgfqpoint{7.306782in}{13.744835in}}%
\pgfpathlineto{\pgfqpoint{7.323011in}{13.777279in}}%
\pgfpathlineto{\pgfqpoint{7.339055in}{13.765907in}}%
\pgfpathlineto{\pgfqpoint{7.354917in}{13.791128in}}%
\pgfpathlineto{\pgfqpoint{7.370603in}{13.761891in}}%
\pgfpathlineto{\pgfqpoint{7.386114in}{13.769122in}}%
\pgfpathlineto{\pgfqpoint{7.401457in}{13.773651in}}%
\pgfpathlineto{\pgfqpoint{7.416633in}{13.762757in}}%
\pgfpathlineto{\pgfqpoint{7.431647in}{13.782071in}}%
\pgfpathlineto{\pgfqpoint{7.446502in}{13.782824in}}%
\pgfpathlineto{\pgfqpoint{7.461202in}{13.782126in}}%
\pgfpathlineto{\pgfqpoint{7.475749in}{13.806776in}}%
\pgfpathlineto{\pgfqpoint{7.490148in}{13.797421in}}%
\pgfpathlineto{\pgfqpoint{7.504399in}{13.774733in}}%
\pgfpathlineto{\pgfqpoint{7.518508in}{13.783809in}}%
\pgfpathlineto{\pgfqpoint{7.518508in}{13.783809in}}%
\pgfpathlineto{\pgfqpoint{7.518508in}{13.783809in}}%
\pgfpathlineto{\pgfqpoint{7.504399in}{13.774733in}}%
\pgfpathlineto{\pgfqpoint{7.490148in}{13.797421in}}%
\pgfpathlineto{\pgfqpoint{7.475749in}{13.806776in}}%
\pgfpathlineto{\pgfqpoint{7.461202in}{13.782126in}}%
\pgfpathlineto{\pgfqpoint{7.446502in}{13.782824in}}%
\pgfpathlineto{\pgfqpoint{7.431647in}{13.782071in}}%
\pgfpathlineto{\pgfqpoint{7.416633in}{13.762757in}}%
\pgfpathlineto{\pgfqpoint{7.401457in}{13.773651in}}%
\pgfpathlineto{\pgfqpoint{7.386114in}{13.769122in}}%
\pgfpathlineto{\pgfqpoint{7.370603in}{13.761891in}}%
\pgfpathlineto{\pgfqpoint{7.354917in}{13.791128in}}%
\pgfpathlineto{\pgfqpoint{7.339055in}{13.765907in}}%
\pgfpathlineto{\pgfqpoint{7.323011in}{13.777279in}}%
\pgfpathlineto{\pgfqpoint{7.306782in}{13.744835in}}%
\pgfpathlineto{\pgfqpoint{7.290363in}{13.772429in}}%
\pgfpathlineto{\pgfqpoint{7.273750in}{13.766098in}}%
\pgfpathlineto{\pgfqpoint{7.256938in}{13.745465in}}%
\pgfpathlineto{\pgfqpoint{7.239922in}{13.792793in}}%
\pgfpathlineto{\pgfqpoint{7.222697in}{13.749826in}}%
\pgfpathlineto{\pgfqpoint{7.205258in}{13.805589in}}%
\pgfpathlineto{\pgfqpoint{7.187600in}{13.740754in}}%
\pgfpathlineto{\pgfqpoint{7.169717in}{13.862936in}}%
\pgfpathlineto{\pgfqpoint{7.151603in}{13.797421in}}%
\pgfpathlineto{\pgfqpoint{7.133253in}{13.739636in}}%
\pgfpathlineto{\pgfqpoint{7.114659in}{13.776994in}}%
\pgfpathlineto{\pgfqpoint{7.095816in}{13.738131in}}%
\pgfpathlineto{\pgfqpoint{7.076716in}{13.733108in}}%
\pgfpathlineto{\pgfqpoint{7.057353in}{13.750420in}}%
\pgfpathlineto{\pgfqpoint{7.037719in}{13.735992in}}%
\pgfpathlineto{\pgfqpoint{7.017806in}{13.757510in}}%
\pgfpathlineto{\pgfqpoint{6.997607in}{13.747937in}}%
\pgfpathlineto{\pgfqpoint{6.977113in}{13.751541in}}%
\pgfpathlineto{\pgfqpoint{6.956316in}{13.739695in}}%
\pgfpathlineto{\pgfqpoint{6.935206in}{13.716396in}}%
\pgfpathlineto{\pgfqpoint{6.913773in}{13.737361in}}%
\pgfpathlineto{\pgfqpoint{6.892008in}{13.741721in}}%
\pgfpathlineto{\pgfqpoint{6.869901in}{13.706849in}}%
\pgfpathlineto{\pgfqpoint{6.847439in}{13.745017in}}%
\pgfpathlineto{\pgfqpoint{6.824613in}{13.717024in}}%
\pgfpathlineto{\pgfqpoint{6.801409in}{13.702546in}}%
\pgfpathlineto{\pgfqpoint{6.777815in}{13.722369in}}%
\pgfpathlineto{\pgfqpoint{6.753818in}{13.727676in}}%
\pgfpathlineto{\pgfqpoint{6.729403in}{13.730545in}}%
\pgfpathlineto{\pgfqpoint{6.704556in}{13.739504in}}%
\pgfpathlineto{\pgfqpoint{6.679262in}{13.708523in}}%
\pgfpathlineto{\pgfqpoint{6.653503in}{13.706973in}}%
\pgfpathlineto{\pgfqpoint{6.627263in}{13.728390in}}%
\pgfpathlineto{\pgfqpoint{6.600523in}{13.854592in}}%
\pgfpathlineto{\pgfqpoint{6.573264in}{13.750737in}}%
\pgfpathlineto{\pgfqpoint{6.545465in}{13.681186in}}%
\pgfpathlineto{\pgfqpoint{6.517104in}{13.785864in}}%
\pgfpathlineto{\pgfqpoint{6.488159in}{13.694728in}}%
\pgfpathlineto{\pgfqpoint{6.458604in}{13.720245in}}%
\pgfpathlineto{\pgfqpoint{6.428414in}{13.732359in}}%
\pgfpathlineto{\pgfqpoint{6.397560in}{13.713948in}}%
\pgfpathlineto{\pgfqpoint{6.366012in}{13.773247in}}%
\pgfpathlineto{\pgfqpoint{6.333739in}{13.692370in}}%
\pgfpathlineto{\pgfqpoint{6.300707in}{13.717569in}}%
\pgfpathlineto{\pgfqpoint{6.266879in}{13.723556in}}%
\pgfpathlineto{\pgfqpoint{6.232215in}{13.750739in}}%
\pgfpathlineto{\pgfqpoint{6.196674in}{13.728507in}}%
\pgfpathlineto{\pgfqpoint{6.160209in}{13.731511in}}%
\pgfpathlineto{\pgfqpoint{6.122772in}{13.721604in}}%
\pgfpathlineto{\pgfqpoint{6.084310in}{13.747047in}}%
\pgfpathlineto{\pgfqpoint{6.044763in}{13.711731in}}%
\pgfpathlineto{\pgfqpoint{6.004070in}{13.731793in}}%
\pgfpathlineto{\pgfqpoint{5.962163in}{13.736080in}}%
\pgfpathlineto{\pgfqpoint{5.918965in}{13.793624in}}%
\pgfpathlineto{\pgfqpoint{5.874396in}{13.721999in}}%
\pgfpathlineto{\pgfqpoint{5.828366in}{14.040861in}}%
\pgfpathlineto{\pgfqpoint{5.780775in}{13.751210in}}%
\pgfpathlineto{\pgfqpoint{5.731513in}{13.771636in}}%
\pgfpathlineto{\pgfqpoint{5.680460in}{13.866869in}}%
\pgfpathlineto{\pgfqpoint{5.627480in}{13.753697in}}%
\pgfpathlineto{\pgfqpoint{5.572422in}{13.750309in}}%
\pgfpathlineto{\pgfqpoint{5.515116in}{13.758504in}}%
\pgfpathlineto{\pgfqpoint{5.455371in}{13.838366in}}%
\pgfpathlineto{\pgfqpoint{5.392969in}{13.800542in}}%
\pgfpathlineto{\pgfqpoint{5.327664in}{13.799627in}}%
\pgfpathlineto{\pgfqpoint{5.259172in}{13.811724in}}%
\pgfpathlineto{\pgfqpoint{5.187166in}{13.789134in}}%
\pgfpathlineto{\pgfqpoint{5.111267in}{13.895571in}}%
\pgfpathlineto{\pgfqpoint{5.031027in}{13.834651in}}%
\pgfpathlineto{\pgfqpoint{4.945922in}{13.897508in}}%
\pgfpathlineto{\pgfqpoint{4.855323in}{13.837011in}}%
\pgfpathlineto{\pgfqpoint{4.758470in}{13.974440in}}%
\pgfpathlineto{\pgfqpoint{4.654437in}{13.887533in}}%
\pgfpathlineto{\pgfqpoint{4.542073in}{13.887524in}}%
\pgfpathlineto{\pgfqpoint{4.419926in}{13.900301in}}%
\pgfpathlineto{\pgfqpoint{4.286129in}{13.986830in}}%
\pgfpathlineto{\pgfqpoint{4.138224in}{13.948632in}}%
\pgfpathlineto{\pgfqpoint{3.972879in}{13.989187in}}%
\pgfpathlineto{\pgfqpoint{3.785427in}{13.977050in}}%
\pgfpathlineto{\pgfqpoint{3.569030in}{13.997126in}}%
\pgfpathlineto{\pgfqpoint{3.313086in}{14.046464in}}%
\pgfpathlineto{\pgfqpoint{2.999836in}{14.142269in}}%
\pgfpathlineto{\pgfqpoint{2.595987in}{14.124231in}}%
\pgfpathlineto{\pgfqpoint{2.026793in}{14.249408in}}%
\pgfpathlineto{\pgfqpoint{1.053750in}{14.466482in}}%
\pgfpathlineto{\pgfqpoint{-3231.325327in}{16.134803in}}%
\pgfpathclose%
\pgfusepath{fill}%
\end{pgfscope}%
\begin{pgfscope}%
\pgfsetbuttcap%
\pgfsetroundjoin%
\definecolor{currentfill}{rgb}{0.000000,0.000000,0.000000}%
\pgfsetfillcolor{currentfill}%
\pgfsetlinewidth{0.803000pt}%
\definecolor{currentstroke}{rgb}{0.000000,0.000000,0.000000}%
\pgfsetstrokecolor{currentstroke}%
\pgfsetdash{}{0pt}%
\pgfsys@defobject{currentmarker}{\pgfqpoint{0.000000in}{-0.048611in}}{\pgfqpoint{0.000000in}{0.000000in}}{%
\pgfpathmoveto{\pgfqpoint{0.000000in}{0.000000in}}%
\pgfpathlineto{\pgfqpoint{0.000000in}{-0.048611in}}%
\pgfusepath{stroke,fill}%
}%
\begin{pgfscope}%
\pgfsys@transformshift{1.053750in}{13.104874in}%
\pgfsys@useobject{currentmarker}{}%
\end{pgfscope}%
\end{pgfscope}%
\begin{pgfscope}%
\pgfsetbuttcap%
\pgfsetroundjoin%
\definecolor{currentfill}{rgb}{0.000000,0.000000,0.000000}%
\pgfsetfillcolor{currentfill}%
\pgfsetlinewidth{0.803000pt}%
\definecolor{currentstroke}{rgb}{0.000000,0.000000,0.000000}%
\pgfsetstrokecolor{currentstroke}%
\pgfsetdash{}{0pt}%
\pgfsys@defobject{currentmarker}{\pgfqpoint{0.000000in}{-0.048611in}}{\pgfqpoint{0.000000in}{0.000000in}}{%
\pgfpathmoveto{\pgfqpoint{0.000000in}{0.000000in}}%
\pgfpathlineto{\pgfqpoint{0.000000in}{-0.048611in}}%
\pgfusepath{stroke,fill}%
}%
\begin{pgfscope}%
\pgfsys@transformshift{4.286129in}{13.104874in}%
\pgfsys@useobject{currentmarker}{}%
\end{pgfscope}%
\end{pgfscope}%
\begin{pgfscope}%
\pgfsetbuttcap%
\pgfsetroundjoin%
\definecolor{currentfill}{rgb}{0.000000,0.000000,0.000000}%
\pgfsetfillcolor{currentfill}%
\pgfsetlinewidth{0.803000pt}%
\definecolor{currentstroke}{rgb}{0.000000,0.000000,0.000000}%
\pgfsetstrokecolor{currentstroke}%
\pgfsetdash{}{0pt}%
\pgfsys@defobject{currentmarker}{\pgfqpoint{0.000000in}{-0.048611in}}{\pgfqpoint{0.000000in}{0.000000in}}{%
\pgfpathmoveto{\pgfqpoint{0.000000in}{0.000000in}}%
\pgfpathlineto{\pgfqpoint{0.000000in}{-0.048611in}}%
\pgfusepath{stroke,fill}%
}%
\begin{pgfscope}%
\pgfsys@transformshift{7.518508in}{13.104874in}%
\pgfsys@useobject{currentmarker}{}%
\end{pgfscope}%
\end{pgfscope}%
\begin{pgfscope}%
\pgfsetbuttcap%
\pgfsetroundjoin%
\definecolor{currentfill}{rgb}{0.000000,0.000000,0.000000}%
\pgfsetfillcolor{currentfill}%
\pgfsetlinewidth{0.602250pt}%
\definecolor{currentstroke}{rgb}{0.000000,0.000000,0.000000}%
\pgfsetstrokecolor{currentstroke}%
\pgfsetdash{}{0pt}%
\pgfsys@defobject{currentmarker}{\pgfqpoint{0.000000in}{-0.027778in}}{\pgfqpoint{0.000000in}{0.000000in}}{%
\pgfpathmoveto{\pgfqpoint{0.000000in}{0.000000in}}%
\pgfpathlineto{\pgfqpoint{0.000000in}{-0.027778in}}%
\pgfusepath{stroke,fill}%
}%
\begin{pgfscope}%
\pgfsys@transformshift{2.026793in}{13.104874in}%
\pgfsys@useobject{currentmarker}{}%
\end{pgfscope}%
\end{pgfscope}%
\begin{pgfscope}%
\pgfsetbuttcap%
\pgfsetroundjoin%
\definecolor{currentfill}{rgb}{0.000000,0.000000,0.000000}%
\pgfsetfillcolor{currentfill}%
\pgfsetlinewidth{0.602250pt}%
\definecolor{currentstroke}{rgb}{0.000000,0.000000,0.000000}%
\pgfsetstrokecolor{currentstroke}%
\pgfsetdash{}{0pt}%
\pgfsys@defobject{currentmarker}{\pgfqpoint{0.000000in}{-0.027778in}}{\pgfqpoint{0.000000in}{0.000000in}}{%
\pgfpathmoveto{\pgfqpoint{0.000000in}{0.000000in}}%
\pgfpathlineto{\pgfqpoint{0.000000in}{-0.027778in}}%
\pgfusepath{stroke,fill}%
}%
\begin{pgfscope}%
\pgfsys@transformshift{2.595987in}{13.104874in}%
\pgfsys@useobject{currentmarker}{}%
\end{pgfscope}%
\end{pgfscope}%
\begin{pgfscope}%
\pgfsetbuttcap%
\pgfsetroundjoin%
\definecolor{currentfill}{rgb}{0.000000,0.000000,0.000000}%
\pgfsetfillcolor{currentfill}%
\pgfsetlinewidth{0.602250pt}%
\definecolor{currentstroke}{rgb}{0.000000,0.000000,0.000000}%
\pgfsetstrokecolor{currentstroke}%
\pgfsetdash{}{0pt}%
\pgfsys@defobject{currentmarker}{\pgfqpoint{0.000000in}{-0.027778in}}{\pgfqpoint{0.000000in}{0.000000in}}{%
\pgfpathmoveto{\pgfqpoint{0.000000in}{0.000000in}}%
\pgfpathlineto{\pgfqpoint{0.000000in}{-0.027778in}}%
\pgfusepath{stroke,fill}%
}%
\begin{pgfscope}%
\pgfsys@transformshift{2.999836in}{13.104874in}%
\pgfsys@useobject{currentmarker}{}%
\end{pgfscope}%
\end{pgfscope}%
\begin{pgfscope}%
\pgfsetbuttcap%
\pgfsetroundjoin%
\definecolor{currentfill}{rgb}{0.000000,0.000000,0.000000}%
\pgfsetfillcolor{currentfill}%
\pgfsetlinewidth{0.602250pt}%
\definecolor{currentstroke}{rgb}{0.000000,0.000000,0.000000}%
\pgfsetstrokecolor{currentstroke}%
\pgfsetdash{}{0pt}%
\pgfsys@defobject{currentmarker}{\pgfqpoint{0.000000in}{-0.027778in}}{\pgfqpoint{0.000000in}{0.000000in}}{%
\pgfpathmoveto{\pgfqpoint{0.000000in}{0.000000in}}%
\pgfpathlineto{\pgfqpoint{0.000000in}{-0.027778in}}%
\pgfusepath{stroke,fill}%
}%
\begin{pgfscope}%
\pgfsys@transformshift{3.313086in}{13.104874in}%
\pgfsys@useobject{currentmarker}{}%
\end{pgfscope}%
\end{pgfscope}%
\begin{pgfscope}%
\pgfsetbuttcap%
\pgfsetroundjoin%
\definecolor{currentfill}{rgb}{0.000000,0.000000,0.000000}%
\pgfsetfillcolor{currentfill}%
\pgfsetlinewidth{0.602250pt}%
\definecolor{currentstroke}{rgb}{0.000000,0.000000,0.000000}%
\pgfsetstrokecolor{currentstroke}%
\pgfsetdash{}{0pt}%
\pgfsys@defobject{currentmarker}{\pgfqpoint{0.000000in}{-0.027778in}}{\pgfqpoint{0.000000in}{0.000000in}}{%
\pgfpathmoveto{\pgfqpoint{0.000000in}{0.000000in}}%
\pgfpathlineto{\pgfqpoint{0.000000in}{-0.027778in}}%
\pgfusepath{stroke,fill}%
}%
\begin{pgfscope}%
\pgfsys@transformshift{3.569030in}{13.104874in}%
\pgfsys@useobject{currentmarker}{}%
\end{pgfscope}%
\end{pgfscope}%
\begin{pgfscope}%
\pgfsetbuttcap%
\pgfsetroundjoin%
\definecolor{currentfill}{rgb}{0.000000,0.000000,0.000000}%
\pgfsetfillcolor{currentfill}%
\pgfsetlinewidth{0.602250pt}%
\definecolor{currentstroke}{rgb}{0.000000,0.000000,0.000000}%
\pgfsetstrokecolor{currentstroke}%
\pgfsetdash{}{0pt}%
\pgfsys@defobject{currentmarker}{\pgfqpoint{0.000000in}{-0.027778in}}{\pgfqpoint{0.000000in}{0.000000in}}{%
\pgfpathmoveto{\pgfqpoint{0.000000in}{0.000000in}}%
\pgfpathlineto{\pgfqpoint{0.000000in}{-0.027778in}}%
\pgfusepath{stroke,fill}%
}%
\begin{pgfscope}%
\pgfsys@transformshift{3.785427in}{13.104874in}%
\pgfsys@useobject{currentmarker}{}%
\end{pgfscope}%
\end{pgfscope}%
\begin{pgfscope}%
\pgfsetbuttcap%
\pgfsetroundjoin%
\definecolor{currentfill}{rgb}{0.000000,0.000000,0.000000}%
\pgfsetfillcolor{currentfill}%
\pgfsetlinewidth{0.602250pt}%
\definecolor{currentstroke}{rgb}{0.000000,0.000000,0.000000}%
\pgfsetstrokecolor{currentstroke}%
\pgfsetdash{}{0pt}%
\pgfsys@defobject{currentmarker}{\pgfqpoint{0.000000in}{-0.027778in}}{\pgfqpoint{0.000000in}{0.000000in}}{%
\pgfpathmoveto{\pgfqpoint{0.000000in}{0.000000in}}%
\pgfpathlineto{\pgfqpoint{0.000000in}{-0.027778in}}%
\pgfusepath{stroke,fill}%
}%
\begin{pgfscope}%
\pgfsys@transformshift{3.972879in}{13.104874in}%
\pgfsys@useobject{currentmarker}{}%
\end{pgfscope}%
\end{pgfscope}%
\begin{pgfscope}%
\pgfsetbuttcap%
\pgfsetroundjoin%
\definecolor{currentfill}{rgb}{0.000000,0.000000,0.000000}%
\pgfsetfillcolor{currentfill}%
\pgfsetlinewidth{0.602250pt}%
\definecolor{currentstroke}{rgb}{0.000000,0.000000,0.000000}%
\pgfsetstrokecolor{currentstroke}%
\pgfsetdash{}{0pt}%
\pgfsys@defobject{currentmarker}{\pgfqpoint{0.000000in}{-0.027778in}}{\pgfqpoint{0.000000in}{0.000000in}}{%
\pgfpathmoveto{\pgfqpoint{0.000000in}{0.000000in}}%
\pgfpathlineto{\pgfqpoint{0.000000in}{-0.027778in}}%
\pgfusepath{stroke,fill}%
}%
\begin{pgfscope}%
\pgfsys@transformshift{4.138224in}{13.104874in}%
\pgfsys@useobject{currentmarker}{}%
\end{pgfscope}%
\end{pgfscope}%
\begin{pgfscope}%
\pgfsetbuttcap%
\pgfsetroundjoin%
\definecolor{currentfill}{rgb}{0.000000,0.000000,0.000000}%
\pgfsetfillcolor{currentfill}%
\pgfsetlinewidth{0.602250pt}%
\definecolor{currentstroke}{rgb}{0.000000,0.000000,0.000000}%
\pgfsetstrokecolor{currentstroke}%
\pgfsetdash{}{0pt}%
\pgfsys@defobject{currentmarker}{\pgfqpoint{0.000000in}{-0.027778in}}{\pgfqpoint{0.000000in}{0.000000in}}{%
\pgfpathmoveto{\pgfqpoint{0.000000in}{0.000000in}}%
\pgfpathlineto{\pgfqpoint{0.000000in}{-0.027778in}}%
\pgfusepath{stroke,fill}%
}%
\begin{pgfscope}%
\pgfsys@transformshift{5.259172in}{13.104874in}%
\pgfsys@useobject{currentmarker}{}%
\end{pgfscope}%
\end{pgfscope}%
\begin{pgfscope}%
\pgfsetbuttcap%
\pgfsetroundjoin%
\definecolor{currentfill}{rgb}{0.000000,0.000000,0.000000}%
\pgfsetfillcolor{currentfill}%
\pgfsetlinewidth{0.602250pt}%
\definecolor{currentstroke}{rgb}{0.000000,0.000000,0.000000}%
\pgfsetstrokecolor{currentstroke}%
\pgfsetdash{}{0pt}%
\pgfsys@defobject{currentmarker}{\pgfqpoint{0.000000in}{-0.027778in}}{\pgfqpoint{0.000000in}{0.000000in}}{%
\pgfpathmoveto{\pgfqpoint{0.000000in}{0.000000in}}%
\pgfpathlineto{\pgfqpoint{0.000000in}{-0.027778in}}%
\pgfusepath{stroke,fill}%
}%
\begin{pgfscope}%
\pgfsys@transformshift{5.828366in}{13.104874in}%
\pgfsys@useobject{currentmarker}{}%
\end{pgfscope}%
\end{pgfscope}%
\begin{pgfscope}%
\pgfsetbuttcap%
\pgfsetroundjoin%
\definecolor{currentfill}{rgb}{0.000000,0.000000,0.000000}%
\pgfsetfillcolor{currentfill}%
\pgfsetlinewidth{0.602250pt}%
\definecolor{currentstroke}{rgb}{0.000000,0.000000,0.000000}%
\pgfsetstrokecolor{currentstroke}%
\pgfsetdash{}{0pt}%
\pgfsys@defobject{currentmarker}{\pgfqpoint{0.000000in}{-0.027778in}}{\pgfqpoint{0.000000in}{0.000000in}}{%
\pgfpathmoveto{\pgfqpoint{0.000000in}{0.000000in}}%
\pgfpathlineto{\pgfqpoint{0.000000in}{-0.027778in}}%
\pgfusepath{stroke,fill}%
}%
\begin{pgfscope}%
\pgfsys@transformshift{6.232215in}{13.104874in}%
\pgfsys@useobject{currentmarker}{}%
\end{pgfscope}%
\end{pgfscope}%
\begin{pgfscope}%
\pgfsetbuttcap%
\pgfsetroundjoin%
\definecolor{currentfill}{rgb}{0.000000,0.000000,0.000000}%
\pgfsetfillcolor{currentfill}%
\pgfsetlinewidth{0.602250pt}%
\definecolor{currentstroke}{rgb}{0.000000,0.000000,0.000000}%
\pgfsetstrokecolor{currentstroke}%
\pgfsetdash{}{0pt}%
\pgfsys@defobject{currentmarker}{\pgfqpoint{0.000000in}{-0.027778in}}{\pgfqpoint{0.000000in}{0.000000in}}{%
\pgfpathmoveto{\pgfqpoint{0.000000in}{0.000000in}}%
\pgfpathlineto{\pgfqpoint{0.000000in}{-0.027778in}}%
\pgfusepath{stroke,fill}%
}%
\begin{pgfscope}%
\pgfsys@transformshift{6.545465in}{13.104874in}%
\pgfsys@useobject{currentmarker}{}%
\end{pgfscope}%
\end{pgfscope}%
\begin{pgfscope}%
\pgfsetbuttcap%
\pgfsetroundjoin%
\definecolor{currentfill}{rgb}{0.000000,0.000000,0.000000}%
\pgfsetfillcolor{currentfill}%
\pgfsetlinewidth{0.602250pt}%
\definecolor{currentstroke}{rgb}{0.000000,0.000000,0.000000}%
\pgfsetstrokecolor{currentstroke}%
\pgfsetdash{}{0pt}%
\pgfsys@defobject{currentmarker}{\pgfqpoint{0.000000in}{-0.027778in}}{\pgfqpoint{0.000000in}{0.000000in}}{%
\pgfpathmoveto{\pgfqpoint{0.000000in}{0.000000in}}%
\pgfpathlineto{\pgfqpoint{0.000000in}{-0.027778in}}%
\pgfusepath{stroke,fill}%
}%
\begin{pgfscope}%
\pgfsys@transformshift{6.801409in}{13.104874in}%
\pgfsys@useobject{currentmarker}{}%
\end{pgfscope}%
\end{pgfscope}%
\begin{pgfscope}%
\pgfsetbuttcap%
\pgfsetroundjoin%
\definecolor{currentfill}{rgb}{0.000000,0.000000,0.000000}%
\pgfsetfillcolor{currentfill}%
\pgfsetlinewidth{0.602250pt}%
\definecolor{currentstroke}{rgb}{0.000000,0.000000,0.000000}%
\pgfsetstrokecolor{currentstroke}%
\pgfsetdash{}{0pt}%
\pgfsys@defobject{currentmarker}{\pgfqpoint{0.000000in}{-0.027778in}}{\pgfqpoint{0.000000in}{0.000000in}}{%
\pgfpathmoveto{\pgfqpoint{0.000000in}{0.000000in}}%
\pgfpathlineto{\pgfqpoint{0.000000in}{-0.027778in}}%
\pgfusepath{stroke,fill}%
}%
\begin{pgfscope}%
\pgfsys@transformshift{7.017806in}{13.104874in}%
\pgfsys@useobject{currentmarker}{}%
\end{pgfscope}%
\end{pgfscope}%
\begin{pgfscope}%
\pgfsetbuttcap%
\pgfsetroundjoin%
\definecolor{currentfill}{rgb}{0.000000,0.000000,0.000000}%
\pgfsetfillcolor{currentfill}%
\pgfsetlinewidth{0.602250pt}%
\definecolor{currentstroke}{rgb}{0.000000,0.000000,0.000000}%
\pgfsetstrokecolor{currentstroke}%
\pgfsetdash{}{0pt}%
\pgfsys@defobject{currentmarker}{\pgfqpoint{0.000000in}{-0.027778in}}{\pgfqpoint{0.000000in}{0.000000in}}{%
\pgfpathmoveto{\pgfqpoint{0.000000in}{0.000000in}}%
\pgfpathlineto{\pgfqpoint{0.000000in}{-0.027778in}}%
\pgfusepath{stroke,fill}%
}%
\begin{pgfscope}%
\pgfsys@transformshift{7.205258in}{13.104874in}%
\pgfsys@useobject{currentmarker}{}%
\end{pgfscope}%
\end{pgfscope}%
\begin{pgfscope}%
\pgfsetbuttcap%
\pgfsetroundjoin%
\definecolor{currentfill}{rgb}{0.000000,0.000000,0.000000}%
\pgfsetfillcolor{currentfill}%
\pgfsetlinewidth{0.602250pt}%
\definecolor{currentstroke}{rgb}{0.000000,0.000000,0.000000}%
\pgfsetstrokecolor{currentstroke}%
\pgfsetdash{}{0pt}%
\pgfsys@defobject{currentmarker}{\pgfqpoint{0.000000in}{-0.027778in}}{\pgfqpoint{0.000000in}{0.000000in}}{%
\pgfpathmoveto{\pgfqpoint{0.000000in}{0.000000in}}%
\pgfpathlineto{\pgfqpoint{0.000000in}{-0.027778in}}%
\pgfusepath{stroke,fill}%
}%
\begin{pgfscope}%
\pgfsys@transformshift{7.370603in}{13.104874in}%
\pgfsys@useobject{currentmarker}{}%
\end{pgfscope}%
\end{pgfscope}%
\begin{pgfscope}%
\pgfsetbuttcap%
\pgfsetroundjoin%
\definecolor{currentfill}{rgb}{0.000000,0.000000,0.000000}%
\pgfsetfillcolor{currentfill}%
\pgfsetlinewidth{0.803000pt}%
\definecolor{currentstroke}{rgb}{0.000000,0.000000,0.000000}%
\pgfsetstrokecolor{currentstroke}%
\pgfsetdash{}{0pt}%
\pgfsys@defobject{currentmarker}{\pgfqpoint{-0.048611in}{0.000000in}}{\pgfqpoint{0.000000in}{0.000000in}}{%
\pgfpathmoveto{\pgfqpoint{0.000000in}{0.000000in}}%
\pgfpathlineto{\pgfqpoint{-0.048611in}{0.000000in}}%
\pgfusepath{stroke,fill}%
}%
\begin{pgfscope}%
\pgfsys@transformshift{1.053750in}{15.186580in}%
\pgfsys@useobject{currentmarker}{}%
\end{pgfscope}%
\end{pgfscope}%
\begin{pgfscope}%
\definecolor{textcolor}{rgb}{0.000000,0.000000,0.000000}%
\pgfsetstrokecolor{textcolor}%
\pgfsetfillcolor{textcolor}%
\pgftext[x=0.727362in,y=15.123266in,left,base]{\color{textcolor}\sffamily\fontsize{12.000000}{14.400000}\selectfont \(\displaystyle {10^{0}}\)}%
\end{pgfscope}%
\begin{pgfscope}%
\pgfsetbuttcap%
\pgfsetroundjoin%
\definecolor{currentfill}{rgb}{0.000000,0.000000,0.000000}%
\pgfsetfillcolor{currentfill}%
\pgfsetlinewidth{0.602250pt}%
\definecolor{currentstroke}{rgb}{0.000000,0.000000,0.000000}%
\pgfsetstrokecolor{currentstroke}%
\pgfsetdash{}{0pt}%
\pgfsys@defobject{currentmarker}{\pgfqpoint{-0.027778in}{0.000000in}}{\pgfqpoint{0.000000in}{0.000000in}}{%
\pgfpathmoveto{\pgfqpoint{0.000000in}{0.000000in}}%
\pgfpathlineto{\pgfqpoint{-0.027778in}{0.000000in}}%
\pgfusepath{stroke,fill}%
}%
\begin{pgfscope}%
\pgfsys@transformshift{1.053750in}{13.397553in}%
\pgfsys@useobject{currentmarker}{}%
\end{pgfscope}%
\end{pgfscope}%
\begin{pgfscope}%
\pgfsetbuttcap%
\pgfsetroundjoin%
\definecolor{currentfill}{rgb}{0.000000,0.000000,0.000000}%
\pgfsetfillcolor{currentfill}%
\pgfsetlinewidth{0.602250pt}%
\definecolor{currentstroke}{rgb}{0.000000,0.000000,0.000000}%
\pgfsetstrokecolor{currentstroke}%
\pgfsetdash{}{0pt}%
\pgfsys@defobject{currentmarker}{\pgfqpoint{-0.027778in}{0.000000in}}{\pgfqpoint{0.000000in}{0.000000in}}{%
\pgfpathmoveto{\pgfqpoint{0.000000in}{0.000000in}}%
\pgfpathlineto{\pgfqpoint{-0.027778in}{0.000000in}}%
\pgfusepath{stroke,fill}%
}%
\begin{pgfscope}%
\pgfsys@transformshift{1.053750in}{13.848262in}%
\pgfsys@useobject{currentmarker}{}%
\end{pgfscope}%
\end{pgfscope}%
\begin{pgfscope}%
\pgfsetbuttcap%
\pgfsetroundjoin%
\definecolor{currentfill}{rgb}{0.000000,0.000000,0.000000}%
\pgfsetfillcolor{currentfill}%
\pgfsetlinewidth{0.602250pt}%
\definecolor{currentstroke}{rgb}{0.000000,0.000000,0.000000}%
\pgfsetstrokecolor{currentstroke}%
\pgfsetdash{}{0pt}%
\pgfsys@defobject{currentmarker}{\pgfqpoint{-0.027778in}{0.000000in}}{\pgfqpoint{0.000000in}{0.000000in}}{%
\pgfpathmoveto{\pgfqpoint{0.000000in}{0.000000in}}%
\pgfpathlineto{\pgfqpoint{-0.027778in}{0.000000in}}%
\pgfusepath{stroke,fill}%
}%
\begin{pgfscope}%
\pgfsys@transformshift{1.053750in}{14.168045in}%
\pgfsys@useobject{currentmarker}{}%
\end{pgfscope}%
\end{pgfscope}%
\begin{pgfscope}%
\pgfsetbuttcap%
\pgfsetroundjoin%
\definecolor{currentfill}{rgb}{0.000000,0.000000,0.000000}%
\pgfsetfillcolor{currentfill}%
\pgfsetlinewidth{0.602250pt}%
\definecolor{currentstroke}{rgb}{0.000000,0.000000,0.000000}%
\pgfsetstrokecolor{currentstroke}%
\pgfsetdash{}{0pt}%
\pgfsys@defobject{currentmarker}{\pgfqpoint{-0.027778in}{0.000000in}}{\pgfqpoint{0.000000in}{0.000000in}}{%
\pgfpathmoveto{\pgfqpoint{0.000000in}{0.000000in}}%
\pgfpathlineto{\pgfqpoint{-0.027778in}{0.000000in}}%
\pgfusepath{stroke,fill}%
}%
\begin{pgfscope}%
\pgfsys@transformshift{1.053750in}{14.416088in}%
\pgfsys@useobject{currentmarker}{}%
\end{pgfscope}%
\end{pgfscope}%
\begin{pgfscope}%
\pgfsetbuttcap%
\pgfsetroundjoin%
\definecolor{currentfill}{rgb}{0.000000,0.000000,0.000000}%
\pgfsetfillcolor{currentfill}%
\pgfsetlinewidth{0.602250pt}%
\definecolor{currentstroke}{rgb}{0.000000,0.000000,0.000000}%
\pgfsetstrokecolor{currentstroke}%
\pgfsetdash{}{0pt}%
\pgfsys@defobject{currentmarker}{\pgfqpoint{-0.027778in}{0.000000in}}{\pgfqpoint{0.000000in}{0.000000in}}{%
\pgfpathmoveto{\pgfqpoint{0.000000in}{0.000000in}}%
\pgfpathlineto{\pgfqpoint{-0.027778in}{0.000000in}}%
\pgfusepath{stroke,fill}%
}%
\begin{pgfscope}%
\pgfsys@transformshift{1.053750in}{14.618754in}%
\pgfsys@useobject{currentmarker}{}%
\end{pgfscope}%
\end{pgfscope}%
\begin{pgfscope}%
\pgfsetbuttcap%
\pgfsetroundjoin%
\definecolor{currentfill}{rgb}{0.000000,0.000000,0.000000}%
\pgfsetfillcolor{currentfill}%
\pgfsetlinewidth{0.602250pt}%
\definecolor{currentstroke}{rgb}{0.000000,0.000000,0.000000}%
\pgfsetstrokecolor{currentstroke}%
\pgfsetdash{}{0pt}%
\pgfsys@defobject{currentmarker}{\pgfqpoint{-0.027778in}{0.000000in}}{\pgfqpoint{0.000000in}{0.000000in}}{%
\pgfpathmoveto{\pgfqpoint{0.000000in}{0.000000in}}%
\pgfpathlineto{\pgfqpoint{-0.027778in}{0.000000in}}%
\pgfusepath{stroke,fill}%
}%
\begin{pgfscope}%
\pgfsys@transformshift{1.053750in}{14.790105in}%
\pgfsys@useobject{currentmarker}{}%
\end{pgfscope}%
\end{pgfscope}%
\begin{pgfscope}%
\pgfsetbuttcap%
\pgfsetroundjoin%
\definecolor{currentfill}{rgb}{0.000000,0.000000,0.000000}%
\pgfsetfillcolor{currentfill}%
\pgfsetlinewidth{0.602250pt}%
\definecolor{currentstroke}{rgb}{0.000000,0.000000,0.000000}%
\pgfsetstrokecolor{currentstroke}%
\pgfsetdash{}{0pt}%
\pgfsys@defobject{currentmarker}{\pgfqpoint{-0.027778in}{0.000000in}}{\pgfqpoint{0.000000in}{0.000000in}}{%
\pgfpathmoveto{\pgfqpoint{0.000000in}{0.000000in}}%
\pgfpathlineto{\pgfqpoint{-0.027778in}{0.000000in}}%
\pgfusepath{stroke,fill}%
}%
\begin{pgfscope}%
\pgfsys@transformshift{1.053750in}{14.938537in}%
\pgfsys@useobject{currentmarker}{}%
\end{pgfscope}%
\end{pgfscope}%
\begin{pgfscope}%
\pgfsetbuttcap%
\pgfsetroundjoin%
\definecolor{currentfill}{rgb}{0.000000,0.000000,0.000000}%
\pgfsetfillcolor{currentfill}%
\pgfsetlinewidth{0.602250pt}%
\definecolor{currentstroke}{rgb}{0.000000,0.000000,0.000000}%
\pgfsetstrokecolor{currentstroke}%
\pgfsetdash{}{0pt}%
\pgfsys@defobject{currentmarker}{\pgfqpoint{-0.027778in}{0.000000in}}{\pgfqpoint{0.000000in}{0.000000in}}{%
\pgfpathmoveto{\pgfqpoint{0.000000in}{0.000000in}}%
\pgfpathlineto{\pgfqpoint{-0.027778in}{0.000000in}}%
\pgfusepath{stroke,fill}%
}%
\begin{pgfscope}%
\pgfsys@transformshift{1.053750in}{15.069463in}%
\pgfsys@useobject{currentmarker}{}%
\end{pgfscope}%
\end{pgfscope}%
\begin{pgfscope}%
\pgfsetbuttcap%
\pgfsetroundjoin%
\definecolor{currentfill}{rgb}{0.000000,0.000000,0.000000}%
\pgfsetfillcolor{currentfill}%
\pgfsetlinewidth{0.602250pt}%
\definecolor{currentstroke}{rgb}{0.000000,0.000000,0.000000}%
\pgfsetstrokecolor{currentstroke}%
\pgfsetdash{}{0pt}%
\pgfsys@defobject{currentmarker}{\pgfqpoint{-0.027778in}{0.000000in}}{\pgfqpoint{0.000000in}{0.000000in}}{%
\pgfpathmoveto{\pgfqpoint{0.000000in}{0.000000in}}%
\pgfpathlineto{\pgfqpoint{-0.027778in}{0.000000in}}%
\pgfusepath{stroke,fill}%
}%
\begin{pgfscope}%
\pgfsys@transformshift{1.053750in}{15.957071in}%
\pgfsys@useobject{currentmarker}{}%
\end{pgfscope}%
\end{pgfscope}%
\begin{pgfscope}%
\definecolor{textcolor}{rgb}{0.000000,0.000000,0.000000}%
\pgfsetstrokecolor{textcolor}%
\pgfsetfillcolor{textcolor}%
\pgftext[x=0.671806in,y=14.644037in,,bottom,rotate=90.000000]{\color{textcolor}\sffamily\fontsize{14.000000}{16.800000}\selectfont test\_losses}%
\end{pgfscope}%
\begin{pgfscope}%
\pgfpathrectangle{\pgfqpoint{1.053750in}{13.104874in}}{\pgfqpoint{6.533250in}{3.078326in}}%
\pgfusepath{clip}%
\pgfsetrectcap%
\pgfsetroundjoin%
\pgfsetlinewidth{1.505625pt}%
\definecolor{currentstroke}{rgb}{0.121569,0.466667,0.705882}%
\pgfsetstrokecolor{currentstroke}%
\pgfsetdash{}{0pt}%
\pgfpathmoveto{\pgfqpoint{1.043750in}{14.421927in}}%
\pgfpathlineto{\pgfqpoint{1.053750in}{14.421922in}}%
\pgfpathlineto{\pgfqpoint{2.026793in}{14.269008in}}%
\pgfpathlineto{\pgfqpoint{2.595987in}{14.111725in}}%
\pgfpathlineto{\pgfqpoint{2.999836in}{14.097315in}}%
\pgfpathlineto{\pgfqpoint{3.313086in}{14.030784in}}%
\pgfpathlineto{\pgfqpoint{3.569030in}{14.002687in}}%
\pgfpathlineto{\pgfqpoint{3.785427in}{14.002251in}}%
\pgfpathlineto{\pgfqpoint{3.972879in}{13.978332in}}%
\pgfpathlineto{\pgfqpoint{4.138224in}{13.971930in}}%
\pgfpathlineto{\pgfqpoint{4.286129in}{13.936451in}}%
\pgfpathlineto{\pgfqpoint{4.419926in}{13.929064in}}%
\pgfpathlineto{\pgfqpoint{4.542073in}{13.911774in}}%
\pgfpathlineto{\pgfqpoint{4.654437in}{13.906638in}}%
\pgfpathlineto{\pgfqpoint{4.758470in}{13.915283in}}%
\pgfpathlineto{\pgfqpoint{4.855323in}{13.880754in}}%
\pgfpathlineto{\pgfqpoint{4.945922in}{13.906506in}}%
\pgfpathlineto{\pgfqpoint{5.031027in}{13.876820in}}%
\pgfpathlineto{\pgfqpoint{5.111267in}{13.871350in}}%
\pgfpathlineto{\pgfqpoint{5.187166in}{13.852461in}}%
\pgfpathlineto{\pgfqpoint{5.259172in}{13.848651in}}%
\pgfpathlineto{\pgfqpoint{5.327664in}{13.849556in}}%
\pgfpathlineto{\pgfqpoint{5.392969in}{13.842636in}}%
\pgfpathlineto{\pgfqpoint{5.455371in}{13.838238in}}%
\pgfpathlineto{\pgfqpoint{5.515116in}{13.830079in}}%
\pgfpathlineto{\pgfqpoint{5.572422in}{13.820083in}}%
\pgfpathlineto{\pgfqpoint{5.627480in}{13.823478in}}%
\pgfpathlineto{\pgfqpoint{5.680460in}{13.821814in}}%
\pgfpathlineto{\pgfqpoint{5.731513in}{13.840577in}}%
\pgfpathlineto{\pgfqpoint{5.780775in}{13.816940in}}%
\pgfpathlineto{\pgfqpoint{5.828366in}{13.838488in}}%
\pgfpathlineto{\pgfqpoint{5.874396in}{13.801332in}}%
\pgfpathlineto{\pgfqpoint{5.918965in}{13.802920in}}%
\pgfpathlineto{\pgfqpoint{5.962163in}{13.807718in}}%
\pgfpathlineto{\pgfqpoint{6.004070in}{13.788165in}}%
\pgfpathlineto{\pgfqpoint{6.044763in}{13.794107in}}%
\pgfpathlineto{\pgfqpoint{6.084310in}{13.786459in}}%
\pgfpathlineto{\pgfqpoint{6.122772in}{13.779426in}}%
\pgfpathlineto{\pgfqpoint{6.160209in}{13.783188in}}%
\pgfpathlineto{\pgfqpoint{6.196674in}{13.784505in}}%
\pgfpathlineto{\pgfqpoint{6.232215in}{13.784600in}}%
\pgfpathlineto{\pgfqpoint{6.266879in}{13.771615in}}%
\pgfpathlineto{\pgfqpoint{6.300707in}{13.777780in}}%
\pgfpathlineto{\pgfqpoint{6.333739in}{13.774221in}}%
\pgfpathlineto{\pgfqpoint{6.366012in}{13.778250in}}%
\pgfpathlineto{\pgfqpoint{6.397560in}{13.778362in}}%
\pgfpathlineto{\pgfqpoint{6.428414in}{13.766382in}}%
\pgfpathlineto{\pgfqpoint{6.458604in}{13.775802in}}%
\pgfpathlineto{\pgfqpoint{6.488159in}{13.761418in}}%
\pgfpathlineto{\pgfqpoint{6.517104in}{13.763389in}}%
\pgfpathlineto{\pgfqpoint{6.545465in}{13.757323in}}%
\pgfpathlineto{\pgfqpoint{6.573264in}{13.758388in}}%
\pgfpathlineto{\pgfqpoint{6.600523in}{13.763190in}}%
\pgfpathlineto{\pgfqpoint{6.627263in}{13.759488in}}%
\pgfpathlineto{\pgfqpoint{6.653503in}{13.754991in}}%
\pgfpathlineto{\pgfqpoint{6.679262in}{13.755148in}}%
\pgfpathlineto{\pgfqpoint{6.704556in}{13.759556in}}%
\pgfpathlineto{\pgfqpoint{6.729403in}{13.750747in}}%
\pgfpathlineto{\pgfqpoint{6.753818in}{13.750970in}}%
\pgfpathlineto{\pgfqpoint{6.777815in}{13.754082in}}%
\pgfpathlineto{\pgfqpoint{6.801409in}{13.743692in}}%
\pgfpathlineto{\pgfqpoint{6.824613in}{13.749743in}}%
\pgfpathlineto{\pgfqpoint{6.847439in}{13.751467in}}%
\pgfpathlineto{\pgfqpoint{6.869901in}{13.744740in}}%
\pgfpathlineto{\pgfqpoint{6.892008in}{13.751813in}}%
\pgfpathlineto{\pgfqpoint{6.913773in}{13.754455in}}%
\pgfpathlineto{\pgfqpoint{6.935206in}{13.732955in}}%
\pgfpathlineto{\pgfqpoint{6.956316in}{13.743991in}}%
\pgfpathlineto{\pgfqpoint{6.977113in}{13.747701in}}%
\pgfpathlineto{\pgfqpoint{6.997607in}{13.746901in}}%
\pgfpathlineto{\pgfqpoint{7.017806in}{13.739523in}}%
\pgfpathlineto{\pgfqpoint{7.037719in}{13.737344in}}%
\pgfpathlineto{\pgfqpoint{7.057353in}{13.763152in}}%
\pgfpathlineto{\pgfqpoint{7.076716in}{13.761882in}}%
\pgfpathlineto{\pgfqpoint{7.095816in}{13.734531in}}%
\pgfpathlineto{\pgfqpoint{7.114659in}{13.729348in}}%
\pgfpathlineto{\pgfqpoint{7.133253in}{13.730260in}}%
\pgfpathlineto{\pgfqpoint{7.151603in}{13.743141in}}%
\pgfpathlineto{\pgfqpoint{7.169717in}{13.793916in}}%
\pgfpathlineto{\pgfqpoint{7.187600in}{13.730551in}}%
\pgfpathlineto{\pgfqpoint{7.205258in}{13.734194in}}%
\pgfpathlineto{\pgfqpoint{7.222697in}{13.732417in}}%
\pgfpathlineto{\pgfqpoint{7.239922in}{13.732174in}}%
\pgfpathlineto{\pgfqpoint{7.256938in}{13.726038in}}%
\pgfpathlineto{\pgfqpoint{7.273750in}{13.738364in}}%
\pgfpathlineto{\pgfqpoint{7.290363in}{13.729418in}}%
\pgfpathlineto{\pgfqpoint{7.306782in}{13.726234in}}%
\pgfpathlineto{\pgfqpoint{7.323011in}{13.735800in}}%
\pgfpathlineto{\pgfqpoint{7.339055in}{13.731308in}}%
\pgfpathlineto{\pgfqpoint{7.354917in}{13.733272in}}%
\pgfpathlineto{\pgfqpoint{7.370603in}{13.731179in}}%
\pgfpathlineto{\pgfqpoint{7.386114in}{13.729148in}}%
\pgfpathlineto{\pgfqpoint{7.401457in}{13.725663in}}%
\pgfpathlineto{\pgfqpoint{7.416633in}{13.728375in}}%
\pgfpathlineto{\pgfqpoint{7.431647in}{13.727352in}}%
\pgfpathlineto{\pgfqpoint{7.446502in}{13.730074in}}%
\pgfpathlineto{\pgfqpoint{7.461202in}{13.725173in}}%
\pgfpathlineto{\pgfqpoint{7.475749in}{13.722238in}}%
\pgfpathlineto{\pgfqpoint{7.490148in}{13.732176in}}%
\pgfpathlineto{\pgfqpoint{7.504399in}{13.724361in}}%
\pgfpathlineto{\pgfqpoint{7.518508in}{13.727216in}}%
\pgfusepath{stroke}%
\end{pgfscope}%
\begin{pgfscope}%
\pgfpathrectangle{\pgfqpoint{1.053750in}{13.104874in}}{\pgfqpoint{6.533250in}{3.078326in}}%
\pgfusepath{clip}%
\pgfsetrectcap%
\pgfsetroundjoin%
\pgfsetlinewidth{1.505625pt}%
\definecolor{currentstroke}{rgb}{1.000000,0.498039,0.054902}%
\pgfsetstrokecolor{currentstroke}%
\pgfsetdash{}{0pt}%
\pgfpathmoveto{\pgfqpoint{1.043750in}{14.466487in}}%
\pgfpathlineto{\pgfqpoint{1.053750in}{14.466482in}}%
\pgfpathlineto{\pgfqpoint{2.026793in}{14.249408in}}%
\pgfpathlineto{\pgfqpoint{2.595987in}{14.124231in}}%
\pgfpathlineto{\pgfqpoint{2.999836in}{14.142269in}}%
\pgfpathlineto{\pgfqpoint{3.313086in}{14.046464in}}%
\pgfpathlineto{\pgfqpoint{3.569030in}{13.997126in}}%
\pgfpathlineto{\pgfqpoint{3.785427in}{13.977050in}}%
\pgfpathlineto{\pgfqpoint{3.972879in}{13.989187in}}%
\pgfpathlineto{\pgfqpoint{4.138224in}{13.948632in}}%
\pgfpathlineto{\pgfqpoint{4.286129in}{13.986830in}}%
\pgfpathlineto{\pgfqpoint{4.419926in}{13.900301in}}%
\pgfpathlineto{\pgfqpoint{4.542073in}{13.887524in}}%
\pgfpathlineto{\pgfqpoint{4.654437in}{13.887533in}}%
\pgfpathlineto{\pgfqpoint{4.758470in}{13.974440in}}%
\pgfpathlineto{\pgfqpoint{4.855323in}{13.837011in}}%
\pgfpathlineto{\pgfqpoint{4.945922in}{13.897508in}}%
\pgfpathlineto{\pgfqpoint{5.031027in}{13.834651in}}%
\pgfpathlineto{\pgfqpoint{5.111267in}{13.895571in}}%
\pgfpathlineto{\pgfqpoint{5.187166in}{13.789134in}}%
\pgfpathlineto{\pgfqpoint{5.259172in}{13.811724in}}%
\pgfpathlineto{\pgfqpoint{5.327664in}{13.799627in}}%
\pgfpathlineto{\pgfqpoint{5.392969in}{13.800542in}}%
\pgfpathlineto{\pgfqpoint{5.455371in}{13.838366in}}%
\pgfpathlineto{\pgfqpoint{5.515116in}{13.758504in}}%
\pgfpathlineto{\pgfqpoint{5.572422in}{13.750309in}}%
\pgfpathlineto{\pgfqpoint{5.627480in}{13.753697in}}%
\pgfpathlineto{\pgfqpoint{5.680460in}{13.866869in}}%
\pgfpathlineto{\pgfqpoint{5.731513in}{13.771636in}}%
\pgfpathlineto{\pgfqpoint{5.780775in}{13.751210in}}%
\pgfpathlineto{\pgfqpoint{5.828366in}{14.040861in}}%
\pgfpathlineto{\pgfqpoint{5.874396in}{13.721999in}}%
\pgfpathlineto{\pgfqpoint{5.918965in}{13.793624in}}%
\pgfpathlineto{\pgfqpoint{5.962163in}{13.736080in}}%
\pgfpathlineto{\pgfqpoint{6.004070in}{13.731793in}}%
\pgfpathlineto{\pgfqpoint{6.044763in}{13.711731in}}%
\pgfpathlineto{\pgfqpoint{6.084310in}{13.747047in}}%
\pgfpathlineto{\pgfqpoint{6.122772in}{13.721604in}}%
\pgfpathlineto{\pgfqpoint{6.160209in}{13.731511in}}%
\pgfpathlineto{\pgfqpoint{6.196674in}{13.728507in}}%
\pgfpathlineto{\pgfqpoint{6.232215in}{13.750739in}}%
\pgfpathlineto{\pgfqpoint{6.266879in}{13.723556in}}%
\pgfpathlineto{\pgfqpoint{6.300707in}{13.717569in}}%
\pgfpathlineto{\pgfqpoint{6.333739in}{13.692370in}}%
\pgfpathlineto{\pgfqpoint{6.366012in}{13.773247in}}%
\pgfpathlineto{\pgfqpoint{6.397560in}{13.713948in}}%
\pgfpathlineto{\pgfqpoint{6.428414in}{13.732359in}}%
\pgfpathlineto{\pgfqpoint{6.458604in}{13.720245in}}%
\pgfpathlineto{\pgfqpoint{6.488159in}{13.694728in}}%
\pgfpathlineto{\pgfqpoint{6.517104in}{13.785864in}}%
\pgfpathlineto{\pgfqpoint{6.545465in}{13.681186in}}%
\pgfpathlineto{\pgfqpoint{6.573264in}{13.750737in}}%
\pgfpathlineto{\pgfqpoint{6.600523in}{13.854592in}}%
\pgfpathlineto{\pgfqpoint{6.627263in}{13.728390in}}%
\pgfpathlineto{\pgfqpoint{6.653503in}{13.706973in}}%
\pgfpathlineto{\pgfqpoint{6.679262in}{13.708523in}}%
\pgfpathlineto{\pgfqpoint{6.704556in}{13.739504in}}%
\pgfpathlineto{\pgfqpoint{6.729403in}{13.730545in}}%
\pgfpathlineto{\pgfqpoint{6.753818in}{13.727676in}}%
\pgfpathlineto{\pgfqpoint{6.777815in}{13.722369in}}%
\pgfpathlineto{\pgfqpoint{6.801409in}{13.702546in}}%
\pgfpathlineto{\pgfqpoint{6.824613in}{13.717024in}}%
\pgfpathlineto{\pgfqpoint{6.847439in}{13.745017in}}%
\pgfpathlineto{\pgfqpoint{6.869901in}{13.706849in}}%
\pgfpathlineto{\pgfqpoint{6.892008in}{13.741721in}}%
\pgfpathlineto{\pgfqpoint{6.913773in}{13.737361in}}%
\pgfpathlineto{\pgfqpoint{6.935206in}{13.716396in}}%
\pgfpathlineto{\pgfqpoint{6.956316in}{13.739695in}}%
\pgfpathlineto{\pgfqpoint{6.977113in}{13.751541in}}%
\pgfpathlineto{\pgfqpoint{6.997607in}{13.747937in}}%
\pgfpathlineto{\pgfqpoint{7.017806in}{13.757510in}}%
\pgfpathlineto{\pgfqpoint{7.037719in}{13.735992in}}%
\pgfpathlineto{\pgfqpoint{7.057353in}{13.750420in}}%
\pgfpathlineto{\pgfqpoint{7.076716in}{13.733108in}}%
\pgfpathlineto{\pgfqpoint{7.095816in}{13.738131in}}%
\pgfpathlineto{\pgfqpoint{7.114659in}{13.776994in}}%
\pgfpathlineto{\pgfqpoint{7.133253in}{13.739636in}}%
\pgfpathlineto{\pgfqpoint{7.151603in}{13.797421in}}%
\pgfpathlineto{\pgfqpoint{7.169717in}{13.862936in}}%
\pgfpathlineto{\pgfqpoint{7.187600in}{13.740754in}}%
\pgfpathlineto{\pgfqpoint{7.205258in}{13.805589in}}%
\pgfpathlineto{\pgfqpoint{7.222697in}{13.749826in}}%
\pgfpathlineto{\pgfqpoint{7.239922in}{13.792793in}}%
\pgfpathlineto{\pgfqpoint{7.256938in}{13.745465in}}%
\pgfpathlineto{\pgfqpoint{7.273750in}{13.766098in}}%
\pgfpathlineto{\pgfqpoint{7.290363in}{13.772429in}}%
\pgfpathlineto{\pgfqpoint{7.306782in}{13.744835in}}%
\pgfpathlineto{\pgfqpoint{7.323011in}{13.777279in}}%
\pgfpathlineto{\pgfqpoint{7.339055in}{13.765907in}}%
\pgfpathlineto{\pgfqpoint{7.354917in}{13.791128in}}%
\pgfpathlineto{\pgfqpoint{7.370603in}{13.761891in}}%
\pgfpathlineto{\pgfqpoint{7.386114in}{13.769122in}}%
\pgfpathlineto{\pgfqpoint{7.401457in}{13.773651in}}%
\pgfpathlineto{\pgfqpoint{7.416633in}{13.762757in}}%
\pgfpathlineto{\pgfqpoint{7.431647in}{13.782071in}}%
\pgfpathlineto{\pgfqpoint{7.446502in}{13.782824in}}%
\pgfpathlineto{\pgfqpoint{7.461202in}{13.782126in}}%
\pgfpathlineto{\pgfqpoint{7.475749in}{13.806776in}}%
\pgfpathlineto{\pgfqpoint{7.490148in}{13.797421in}}%
\pgfpathlineto{\pgfqpoint{7.504399in}{13.774733in}}%
\pgfpathlineto{\pgfqpoint{7.518508in}{13.783809in}}%
\pgfusepath{stroke}%
\end{pgfscope}%
\begin{pgfscope}%
\pgfsetrectcap%
\pgfsetmiterjoin%
\pgfsetlinewidth{0.803000pt}%
\definecolor{currentstroke}{rgb}{0.000000,0.000000,0.000000}%
\pgfsetstrokecolor{currentstroke}%
\pgfsetdash{}{0pt}%
\pgfpathmoveto{\pgfqpoint{1.053750in}{13.104874in}}%
\pgfpathlineto{\pgfqpoint{1.053750in}{16.183200in}}%
\pgfusepath{stroke}%
\end{pgfscope}%
\begin{pgfscope}%
\pgfsetrectcap%
\pgfsetmiterjoin%
\pgfsetlinewidth{0.803000pt}%
\definecolor{currentstroke}{rgb}{0.000000,0.000000,0.000000}%
\pgfsetstrokecolor{currentstroke}%
\pgfsetdash{}{0pt}%
\pgfpathmoveto{\pgfqpoint{7.587000in}{13.104874in}}%
\pgfpathlineto{\pgfqpoint{7.587000in}{16.183200in}}%
\pgfusepath{stroke}%
\end{pgfscope}%
\begin{pgfscope}%
\pgfsetrectcap%
\pgfsetmiterjoin%
\pgfsetlinewidth{0.803000pt}%
\definecolor{currentstroke}{rgb}{0.000000,0.000000,0.000000}%
\pgfsetstrokecolor{currentstroke}%
\pgfsetdash{}{0pt}%
\pgfpathmoveto{\pgfqpoint{1.053750in}{13.104874in}}%
\pgfpathlineto{\pgfqpoint{7.587000in}{13.104874in}}%
\pgfusepath{stroke}%
\end{pgfscope}%
\begin{pgfscope}%
\pgfsetrectcap%
\pgfsetmiterjoin%
\pgfsetlinewidth{0.803000pt}%
\definecolor{currentstroke}{rgb}{0.000000,0.000000,0.000000}%
\pgfsetstrokecolor{currentstroke}%
\pgfsetdash{}{0pt}%
\pgfpathmoveto{\pgfqpoint{1.053750in}{16.183200in}}%
\pgfpathlineto{\pgfqpoint{7.587000in}{16.183200in}}%
\pgfusepath{stroke}%
\end{pgfscope}%
\begin{pgfscope}%
\definecolor{textcolor}{rgb}{0.000000,0.000000,0.000000}%
\pgfsetstrokecolor{textcolor}%
\pgfsetfillcolor{textcolor}%
\pgftext[x=4.320375in,y=16.266533in,,base]{\color{textcolor}\sffamily\fontsize{12.000000}{14.400000}\selectfont fmnist\_2c2d}%
\end{pgfscope}%
\begin{pgfscope}%
\pgfsetbuttcap%
\pgfsetmiterjoin%
\definecolor{currentfill}{rgb}{1.000000,1.000000,1.000000}%
\pgfsetfillcolor{currentfill}%
\pgfsetfillopacity{0.800000}%
\pgfsetlinewidth{1.003750pt}%
\definecolor{currentstroke}{rgb}{0.800000,0.800000,0.800000}%
\pgfsetstrokecolor{currentstroke}%
\pgfsetstrokeopacity{0.800000}%
\pgfsetdash{}{0pt}%
\pgfpathmoveto{\pgfqpoint{5.329822in}{15.560609in}}%
\pgfpathlineto{\pgfqpoint{7.470333in}{15.560609in}}%
\pgfpathquadraticcurveto{\pgfqpoint{7.503667in}{15.560609in}}{\pgfqpoint{7.503667in}{15.593943in}}%
\pgfpathlineto{\pgfqpoint{7.503667in}{16.066533in}}%
\pgfpathquadraticcurveto{\pgfqpoint{7.503667in}{16.099867in}}{\pgfqpoint{7.470333in}{16.099867in}}%
\pgfpathlineto{\pgfqpoint{5.329822in}{16.099867in}}%
\pgfpathquadraticcurveto{\pgfqpoint{5.296489in}{16.099867in}}{\pgfqpoint{5.296489in}{16.066533in}}%
\pgfpathlineto{\pgfqpoint{5.296489in}{15.593943in}}%
\pgfpathquadraticcurveto{\pgfqpoint{5.296489in}{15.560609in}}{\pgfqpoint{5.329822in}{15.560609in}}%
\pgfpathclose%
\pgfusepath{stroke,fill}%
\end{pgfscope}%
\begin{pgfscope}%
\pgfsetrectcap%
\pgfsetroundjoin%
\pgfsetlinewidth{1.505625pt}%
\definecolor{currentstroke}{rgb}{0.121569,0.466667,0.705882}%
\pgfsetstrokecolor{currentstroke}%
\pgfsetdash{}{0pt}%
\pgfpathmoveto{\pgfqpoint{5.363156in}{15.964906in}}%
\pgfpathlineto{\pgfqpoint{5.696489in}{15.964906in}}%
\pgfusepath{stroke}%
\end{pgfscope}%
\begin{pgfscope}%
\definecolor{textcolor}{rgb}{0.000000,0.000000,0.000000}%
\pgfsetstrokecolor{textcolor}%
\pgfsetfillcolor{textcolor}%
\pgftext[x=5.829822in,y=15.906572in,left,base]{\color{textcolor}\sffamily\fontsize{12.000000}{14.400000}\selectfont PreconditionedSGD}%
\end{pgfscope}%
\begin{pgfscope}%
\pgfsetrectcap%
\pgfsetroundjoin%
\pgfsetlinewidth{1.505625pt}%
\definecolor{currentstroke}{rgb}{1.000000,0.498039,0.054902}%
\pgfsetstrokecolor{currentstroke}%
\pgfsetdash{}{0pt}%
\pgfpathmoveto{\pgfqpoint{5.363156in}{15.720277in}}%
\pgfpathlineto{\pgfqpoint{5.696489in}{15.720277in}}%
\pgfusepath{stroke}%
\end{pgfscope}%
\begin{pgfscope}%
\definecolor{textcolor}{rgb}{0.000000,0.000000,0.000000}%
\pgfsetstrokecolor{textcolor}%
\pgfsetfillcolor{textcolor}%
\pgftext[x=5.829822in,y=15.661944in,left,base]{\color{textcolor}\sffamily\fontsize{12.000000}{14.400000}\selectfont AdaptiveSGD}%
\end{pgfscope}%
\begin{pgfscope}%
\pgfsetbuttcap%
\pgfsetmiterjoin%
\definecolor{currentfill}{rgb}{1.000000,1.000000,1.000000}%
\pgfsetfillcolor{currentfill}%
\pgfsetlinewidth{0.000000pt}%
\definecolor{currentstroke}{rgb}{0.000000,0.000000,0.000000}%
\pgfsetstrokecolor{currentstroke}%
\pgfsetstrokeopacity{0.000000}%
\pgfsetdash{}{0pt}%
\pgfpathmoveto{\pgfqpoint{1.053750in}{9.410883in}}%
\pgfpathlineto{\pgfqpoint{7.587000in}{9.410883in}}%
\pgfpathlineto{\pgfqpoint{7.587000in}{12.489209in}}%
\pgfpathlineto{\pgfqpoint{1.053750in}{12.489209in}}%
\pgfpathclose%
\pgfusepath{fill}%
\end{pgfscope}%
\begin{pgfscope}%
\pgfpathrectangle{\pgfqpoint{1.053750in}{9.410883in}}{\pgfqpoint{6.533250in}{3.078326in}}%
\pgfusepath{clip}%
\pgfsetbuttcap%
\pgfsetroundjoin%
\definecolor{currentfill}{rgb}{0.121569,0.466667,0.705882}%
\pgfsetfillcolor{currentfill}%
\pgfsetfillopacity{0.300000}%
\pgfsetlinewidth{0.000000pt}%
\definecolor{currentstroke}{rgb}{0.000000,0.000000,0.000000}%
\pgfsetstrokecolor{currentstroke}%
\pgfsetdash{}{0pt}%
\pgfpathmoveto{\pgfqpoint{-3231.325327in}{12.343495in}}%
\pgfpathlineto{\pgfqpoint{-3231.325327in}{12.343495in}}%
\pgfpathlineto{\pgfqpoint{1.053750in}{9.944809in}}%
\pgfpathlineto{\pgfqpoint{2.026793in}{9.840428in}}%
\pgfpathlineto{\pgfqpoint{2.595987in}{9.735962in}}%
\pgfpathlineto{\pgfqpoint{2.999836in}{9.769284in}}%
\pgfpathlineto{\pgfqpoint{3.313086in}{9.716391in}}%
\pgfpathlineto{\pgfqpoint{3.569030in}{9.719145in}}%
\pgfpathlineto{\pgfqpoint{3.785427in}{9.696848in}}%
\pgfpathlineto{\pgfqpoint{3.972879in}{9.677552in}}%
\pgfpathlineto{\pgfqpoint{4.138224in}{9.693759in}}%
\pgfpathlineto{\pgfqpoint{4.286129in}{9.672629in}}%
\pgfpathlineto{\pgfqpoint{4.419926in}{9.678337in}}%
\pgfpathlineto{\pgfqpoint{4.542073in}{9.650661in}}%
\pgfpathlineto{\pgfqpoint{4.654437in}{9.652328in}}%
\pgfpathlineto{\pgfqpoint{4.758470in}{9.669748in}}%
\pgfpathlineto{\pgfqpoint{4.855323in}{9.633836in}}%
\pgfpathlineto{\pgfqpoint{4.945922in}{9.649615in}}%
\pgfpathlineto{\pgfqpoint{5.031027in}{9.637636in}}%
\pgfpathlineto{\pgfqpoint{5.111267in}{9.633983in}}%
\pgfpathlineto{\pgfqpoint{5.187166in}{9.633934in}}%
\pgfpathlineto{\pgfqpoint{5.259172in}{9.616556in}}%
\pgfpathlineto{\pgfqpoint{5.327664in}{9.638170in}}%
\pgfpathlineto{\pgfqpoint{5.392969in}{9.627082in}}%
\pgfpathlineto{\pgfqpoint{5.455371in}{9.631392in}}%
\pgfpathlineto{\pgfqpoint{5.515116in}{9.621441in}}%
\pgfpathlineto{\pgfqpoint{5.572422in}{9.602430in}}%
\pgfpathlineto{\pgfqpoint{5.627480in}{9.617916in}}%
\pgfpathlineto{\pgfqpoint{5.680460in}{9.604315in}}%
\pgfpathlineto{\pgfqpoint{5.731513in}{9.597631in}}%
\pgfpathlineto{\pgfqpoint{5.780775in}{9.613249in}}%
\pgfpathlineto{\pgfqpoint{5.828366in}{9.615851in}}%
\pgfpathlineto{\pgfqpoint{5.874396in}{9.602847in}}%
\pgfpathlineto{\pgfqpoint{5.918965in}{9.602438in}}%
\pgfpathlineto{\pgfqpoint{5.962163in}{9.609095in}}%
\pgfpathlineto{\pgfqpoint{6.004070in}{9.594005in}}%
\pgfpathlineto{\pgfqpoint{6.044763in}{9.593257in}}%
\pgfpathlineto{\pgfqpoint{6.084310in}{9.597558in}}%
\pgfpathlineto{\pgfqpoint{6.122772in}{9.592725in}}%
\pgfpathlineto{\pgfqpoint{6.160209in}{9.583759in}}%
\pgfpathlineto{\pgfqpoint{6.196674in}{9.588108in}}%
\pgfpathlineto{\pgfqpoint{6.232215in}{9.600423in}}%
\pgfpathlineto{\pgfqpoint{6.266879in}{9.591261in}}%
\pgfpathlineto{\pgfqpoint{6.300707in}{9.601181in}}%
\pgfpathlineto{\pgfqpoint{6.333739in}{9.581341in}}%
\pgfpathlineto{\pgfqpoint{6.366012in}{9.583077in}}%
\pgfpathlineto{\pgfqpoint{6.397560in}{9.586008in}}%
\pgfpathlineto{\pgfqpoint{6.428414in}{9.569952in}}%
\pgfpathlineto{\pgfqpoint{6.458604in}{9.583962in}}%
\pgfpathlineto{\pgfqpoint{6.488159in}{9.557540in}}%
\pgfpathlineto{\pgfqpoint{6.517104in}{9.583034in}}%
\pgfpathlineto{\pgfqpoint{6.545465in}{9.588564in}}%
\pgfpathlineto{\pgfqpoint{6.573264in}{9.576872in}}%
\pgfpathlineto{\pgfqpoint{6.600523in}{9.570133in}}%
\pgfpathlineto{\pgfqpoint{6.627263in}{9.569768in}}%
\pgfpathlineto{\pgfqpoint{6.653503in}{9.560396in}}%
\pgfpathlineto{\pgfqpoint{6.679262in}{9.564247in}}%
\pgfpathlineto{\pgfqpoint{6.704556in}{9.571104in}}%
\pgfpathlineto{\pgfqpoint{6.729403in}{9.561079in}}%
\pgfpathlineto{\pgfqpoint{6.753818in}{9.561255in}}%
\pgfpathlineto{\pgfqpoint{6.777815in}{9.560196in}}%
\pgfpathlineto{\pgfqpoint{6.801409in}{9.553259in}}%
\pgfpathlineto{\pgfqpoint{6.824613in}{9.567719in}}%
\pgfpathlineto{\pgfqpoint{6.847439in}{9.547377in}}%
\pgfpathlineto{\pgfqpoint{6.869901in}{9.549214in}}%
\pgfpathlineto{\pgfqpoint{6.892008in}{9.556721in}}%
\pgfpathlineto{\pgfqpoint{6.913773in}{9.562780in}}%
\pgfpathlineto{\pgfqpoint{6.935206in}{9.548756in}}%
\pgfpathlineto{\pgfqpoint{6.956316in}{9.549631in}}%
\pgfpathlineto{\pgfqpoint{6.977113in}{9.553052in}}%
\pgfpathlineto{\pgfqpoint{6.997607in}{9.557816in}}%
\pgfpathlineto{\pgfqpoint{7.017806in}{9.554052in}}%
\pgfpathlineto{\pgfqpoint{7.037719in}{9.548522in}}%
\pgfpathlineto{\pgfqpoint{7.057353in}{9.556158in}}%
\pgfpathlineto{\pgfqpoint{7.076716in}{9.556790in}}%
\pgfpathlineto{\pgfqpoint{7.095816in}{9.540149in}}%
\pgfpathlineto{\pgfqpoint{7.114659in}{9.548066in}}%
\pgfpathlineto{\pgfqpoint{7.133253in}{9.558214in}}%
\pgfpathlineto{\pgfqpoint{7.151603in}{9.541294in}}%
\pgfpathlineto{\pgfqpoint{7.169717in}{9.571030in}}%
\pgfpathlineto{\pgfqpoint{7.187600in}{9.545709in}}%
\pgfpathlineto{\pgfqpoint{7.205258in}{9.547216in}}%
\pgfpathlineto{\pgfqpoint{7.222697in}{9.555695in}}%
\pgfpathlineto{\pgfqpoint{7.239922in}{9.541849in}}%
\pgfpathlineto{\pgfqpoint{7.256938in}{9.537807in}}%
\pgfpathlineto{\pgfqpoint{7.273750in}{9.536034in}}%
\pgfpathlineto{\pgfqpoint{7.290363in}{9.533695in}}%
\pgfpathlineto{\pgfqpoint{7.306782in}{9.545084in}}%
\pgfpathlineto{\pgfqpoint{7.323011in}{9.534066in}}%
\pgfpathlineto{\pgfqpoint{7.339055in}{9.523415in}}%
\pgfpathlineto{\pgfqpoint{7.354917in}{9.544005in}}%
\pgfpathlineto{\pgfqpoint{7.370603in}{9.528367in}}%
\pgfpathlineto{\pgfqpoint{7.386114in}{9.535312in}}%
\pgfpathlineto{\pgfqpoint{7.401457in}{9.533888in}}%
\pgfpathlineto{\pgfqpoint{7.416633in}{9.532387in}}%
\pgfpathlineto{\pgfqpoint{7.431647in}{9.520151in}}%
\pgfpathlineto{\pgfqpoint{7.446502in}{9.532402in}}%
\pgfpathlineto{\pgfqpoint{7.461202in}{9.541107in}}%
\pgfpathlineto{\pgfqpoint{7.475749in}{9.525121in}}%
\pgfpathlineto{\pgfqpoint{7.490148in}{9.537370in}}%
\pgfpathlineto{\pgfqpoint{7.504399in}{9.519571in}}%
\pgfpathlineto{\pgfqpoint{7.518508in}{9.520081in}}%
\pgfpathlineto{\pgfqpoint{7.518508in}{9.520081in}}%
\pgfpathlineto{\pgfqpoint{7.518508in}{9.520081in}}%
\pgfpathlineto{\pgfqpoint{7.504399in}{9.519571in}}%
\pgfpathlineto{\pgfqpoint{7.490148in}{9.537370in}}%
\pgfpathlineto{\pgfqpoint{7.475749in}{9.525121in}}%
\pgfpathlineto{\pgfqpoint{7.461202in}{9.541107in}}%
\pgfpathlineto{\pgfqpoint{7.446502in}{9.532402in}}%
\pgfpathlineto{\pgfqpoint{7.431647in}{9.520151in}}%
\pgfpathlineto{\pgfqpoint{7.416633in}{9.532387in}}%
\pgfpathlineto{\pgfqpoint{7.401457in}{9.533888in}}%
\pgfpathlineto{\pgfqpoint{7.386114in}{9.535312in}}%
\pgfpathlineto{\pgfqpoint{7.370603in}{9.528367in}}%
\pgfpathlineto{\pgfqpoint{7.354917in}{9.544005in}}%
\pgfpathlineto{\pgfqpoint{7.339055in}{9.523415in}}%
\pgfpathlineto{\pgfqpoint{7.323011in}{9.534066in}}%
\pgfpathlineto{\pgfqpoint{7.306782in}{9.545084in}}%
\pgfpathlineto{\pgfqpoint{7.290363in}{9.533695in}}%
\pgfpathlineto{\pgfqpoint{7.273750in}{9.536034in}}%
\pgfpathlineto{\pgfqpoint{7.256938in}{9.537807in}}%
\pgfpathlineto{\pgfqpoint{7.239922in}{9.541849in}}%
\pgfpathlineto{\pgfqpoint{7.222697in}{9.555695in}}%
\pgfpathlineto{\pgfqpoint{7.205258in}{9.547216in}}%
\pgfpathlineto{\pgfqpoint{7.187600in}{9.545709in}}%
\pgfpathlineto{\pgfqpoint{7.169717in}{9.571030in}}%
\pgfpathlineto{\pgfqpoint{7.151603in}{9.541294in}}%
\pgfpathlineto{\pgfqpoint{7.133253in}{9.558214in}}%
\pgfpathlineto{\pgfqpoint{7.114659in}{9.548066in}}%
\pgfpathlineto{\pgfqpoint{7.095816in}{9.540149in}}%
\pgfpathlineto{\pgfqpoint{7.076716in}{9.556790in}}%
\pgfpathlineto{\pgfqpoint{7.057353in}{9.556158in}}%
\pgfpathlineto{\pgfqpoint{7.037719in}{9.548522in}}%
\pgfpathlineto{\pgfqpoint{7.017806in}{9.554052in}}%
\pgfpathlineto{\pgfqpoint{6.997607in}{9.557816in}}%
\pgfpathlineto{\pgfqpoint{6.977113in}{9.553052in}}%
\pgfpathlineto{\pgfqpoint{6.956316in}{9.549631in}}%
\pgfpathlineto{\pgfqpoint{6.935206in}{9.548756in}}%
\pgfpathlineto{\pgfqpoint{6.913773in}{9.562780in}}%
\pgfpathlineto{\pgfqpoint{6.892008in}{9.556721in}}%
\pgfpathlineto{\pgfqpoint{6.869901in}{9.549214in}}%
\pgfpathlineto{\pgfqpoint{6.847439in}{9.547377in}}%
\pgfpathlineto{\pgfqpoint{6.824613in}{9.567719in}}%
\pgfpathlineto{\pgfqpoint{6.801409in}{9.553259in}}%
\pgfpathlineto{\pgfqpoint{6.777815in}{9.560196in}}%
\pgfpathlineto{\pgfqpoint{6.753818in}{9.561255in}}%
\pgfpathlineto{\pgfqpoint{6.729403in}{9.561079in}}%
\pgfpathlineto{\pgfqpoint{6.704556in}{9.571104in}}%
\pgfpathlineto{\pgfqpoint{6.679262in}{9.564247in}}%
\pgfpathlineto{\pgfqpoint{6.653503in}{9.560396in}}%
\pgfpathlineto{\pgfqpoint{6.627263in}{9.569768in}}%
\pgfpathlineto{\pgfqpoint{6.600523in}{9.570133in}}%
\pgfpathlineto{\pgfqpoint{6.573264in}{9.576872in}}%
\pgfpathlineto{\pgfqpoint{6.545465in}{9.588564in}}%
\pgfpathlineto{\pgfqpoint{6.517104in}{9.583034in}}%
\pgfpathlineto{\pgfqpoint{6.488159in}{9.557540in}}%
\pgfpathlineto{\pgfqpoint{6.458604in}{9.583962in}}%
\pgfpathlineto{\pgfqpoint{6.428414in}{9.569952in}}%
\pgfpathlineto{\pgfqpoint{6.397560in}{9.586008in}}%
\pgfpathlineto{\pgfqpoint{6.366012in}{9.583077in}}%
\pgfpathlineto{\pgfqpoint{6.333739in}{9.581341in}}%
\pgfpathlineto{\pgfqpoint{6.300707in}{9.601181in}}%
\pgfpathlineto{\pgfqpoint{6.266879in}{9.591261in}}%
\pgfpathlineto{\pgfqpoint{6.232215in}{9.600423in}}%
\pgfpathlineto{\pgfqpoint{6.196674in}{9.588108in}}%
\pgfpathlineto{\pgfqpoint{6.160209in}{9.583759in}}%
\pgfpathlineto{\pgfqpoint{6.122772in}{9.592725in}}%
\pgfpathlineto{\pgfqpoint{6.084310in}{9.597558in}}%
\pgfpathlineto{\pgfqpoint{6.044763in}{9.593257in}}%
\pgfpathlineto{\pgfqpoint{6.004070in}{9.594005in}}%
\pgfpathlineto{\pgfqpoint{5.962163in}{9.609095in}}%
\pgfpathlineto{\pgfqpoint{5.918965in}{9.602438in}}%
\pgfpathlineto{\pgfqpoint{5.874396in}{9.602847in}}%
\pgfpathlineto{\pgfqpoint{5.828366in}{9.615851in}}%
\pgfpathlineto{\pgfqpoint{5.780775in}{9.613249in}}%
\pgfpathlineto{\pgfqpoint{5.731513in}{9.597631in}}%
\pgfpathlineto{\pgfqpoint{5.680460in}{9.604315in}}%
\pgfpathlineto{\pgfqpoint{5.627480in}{9.617916in}}%
\pgfpathlineto{\pgfqpoint{5.572422in}{9.602430in}}%
\pgfpathlineto{\pgfqpoint{5.515116in}{9.621441in}}%
\pgfpathlineto{\pgfqpoint{5.455371in}{9.631392in}}%
\pgfpathlineto{\pgfqpoint{5.392969in}{9.627082in}}%
\pgfpathlineto{\pgfqpoint{5.327664in}{9.638170in}}%
\pgfpathlineto{\pgfqpoint{5.259172in}{9.616556in}}%
\pgfpathlineto{\pgfqpoint{5.187166in}{9.633934in}}%
\pgfpathlineto{\pgfqpoint{5.111267in}{9.633983in}}%
\pgfpathlineto{\pgfqpoint{5.031027in}{9.637636in}}%
\pgfpathlineto{\pgfqpoint{4.945922in}{9.649615in}}%
\pgfpathlineto{\pgfqpoint{4.855323in}{9.633836in}}%
\pgfpathlineto{\pgfqpoint{4.758470in}{9.669748in}}%
\pgfpathlineto{\pgfqpoint{4.654437in}{9.652328in}}%
\pgfpathlineto{\pgfqpoint{4.542073in}{9.650661in}}%
\pgfpathlineto{\pgfqpoint{4.419926in}{9.678337in}}%
\pgfpathlineto{\pgfqpoint{4.286129in}{9.672629in}}%
\pgfpathlineto{\pgfqpoint{4.138224in}{9.693759in}}%
\pgfpathlineto{\pgfqpoint{3.972879in}{9.677552in}}%
\pgfpathlineto{\pgfqpoint{3.785427in}{9.696848in}}%
\pgfpathlineto{\pgfqpoint{3.569030in}{9.719145in}}%
\pgfpathlineto{\pgfqpoint{3.313086in}{9.716391in}}%
\pgfpathlineto{\pgfqpoint{2.999836in}{9.769284in}}%
\pgfpathlineto{\pgfqpoint{2.595987in}{9.735962in}}%
\pgfpathlineto{\pgfqpoint{2.026793in}{9.840428in}}%
\pgfpathlineto{\pgfqpoint{1.053750in}{9.944809in}}%
\pgfpathlineto{\pgfqpoint{-3231.325327in}{12.343495in}}%
\pgfpathclose%
\pgfusepath{fill}%
\end{pgfscope}%
\begin{pgfscope}%
\pgfpathrectangle{\pgfqpoint{1.053750in}{9.410883in}}{\pgfqpoint{6.533250in}{3.078326in}}%
\pgfusepath{clip}%
\pgfsetbuttcap%
\pgfsetroundjoin%
\definecolor{currentfill}{rgb}{1.000000,0.498039,0.054902}%
\pgfsetfillcolor{currentfill}%
\pgfsetfillopacity{0.300000}%
\pgfsetlinewidth{0.000000pt}%
\definecolor{currentstroke}{rgb}{0.000000,0.000000,0.000000}%
\pgfsetstrokecolor{currentstroke}%
\pgfsetdash{}{0pt}%
\pgfpathmoveto{\pgfqpoint{-3231.325327in}{12.343495in}}%
\pgfpathlineto{\pgfqpoint{-3231.325327in}{12.343495in}}%
\pgfpathlineto{\pgfqpoint{1.053750in}{9.973369in}}%
\pgfpathlineto{\pgfqpoint{2.026793in}{9.832830in}}%
\pgfpathlineto{\pgfqpoint{2.595987in}{9.741821in}}%
\pgfpathlineto{\pgfqpoint{2.999836in}{9.790302in}}%
\pgfpathlineto{\pgfqpoint{3.313086in}{9.723886in}}%
\pgfpathlineto{\pgfqpoint{3.569030in}{9.719228in}}%
\pgfpathlineto{\pgfqpoint{3.785427in}{9.689323in}}%
\pgfpathlineto{\pgfqpoint{3.972879in}{9.676864in}}%
\pgfpathlineto{\pgfqpoint{4.138224in}{9.679293in}}%
\pgfpathlineto{\pgfqpoint{4.286129in}{9.689122in}}%
\pgfpathlineto{\pgfqpoint{4.419926in}{9.663988in}}%
\pgfpathlineto{\pgfqpoint{4.542073in}{9.638943in}}%
\pgfpathlineto{\pgfqpoint{4.654437in}{9.640369in}}%
\pgfpathlineto{\pgfqpoint{4.758470in}{9.684242in}}%
\pgfpathlineto{\pgfqpoint{4.855323in}{9.612005in}}%
\pgfpathlineto{\pgfqpoint{4.945922in}{9.639899in}}%
\pgfpathlineto{\pgfqpoint{5.031027in}{9.614074in}}%
\pgfpathlineto{\pgfqpoint{5.111267in}{9.632239in}}%
\pgfpathlineto{\pgfqpoint{5.187166in}{9.597787in}}%
\pgfpathlineto{\pgfqpoint{5.259172in}{9.592206in}}%
\pgfpathlineto{\pgfqpoint{5.327664in}{9.605838in}}%
\pgfpathlineto{\pgfqpoint{5.392969in}{9.596824in}}%
\pgfpathlineto{\pgfqpoint{5.455371in}{9.615126in}}%
\pgfpathlineto{\pgfqpoint{5.515116in}{9.581070in}}%
\pgfpathlineto{\pgfqpoint{5.572422in}{9.563488in}}%
\pgfpathlineto{\pgfqpoint{5.627480in}{9.573841in}}%
\pgfpathlineto{\pgfqpoint{5.680460in}{9.595843in}}%
\pgfpathlineto{\pgfqpoint{5.731513in}{9.553392in}}%
\pgfpathlineto{\pgfqpoint{5.780775in}{9.573263in}}%
\pgfpathlineto{\pgfqpoint{5.828366in}{9.662032in}}%
\pgfpathlineto{\pgfqpoint{5.874396in}{9.550554in}}%
\pgfpathlineto{\pgfqpoint{5.918965in}{9.568999in}}%
\pgfpathlineto{\pgfqpoint{5.962163in}{9.552038in}}%
\pgfpathlineto{\pgfqpoint{6.004070in}{9.542490in}}%
\pgfpathlineto{\pgfqpoint{6.044763in}{9.539180in}}%
\pgfpathlineto{\pgfqpoint{6.084310in}{9.551607in}}%
\pgfpathlineto{\pgfqpoint{6.122772in}{9.542073in}}%
\pgfpathlineto{\pgfqpoint{6.160209in}{9.532268in}}%
\pgfpathlineto{\pgfqpoint{6.196674in}{9.530824in}}%
\pgfpathlineto{\pgfqpoint{6.232215in}{9.554233in}}%
\pgfpathlineto{\pgfqpoint{6.266879in}{9.534773in}}%
\pgfpathlineto{\pgfqpoint{6.300707in}{9.539905in}}%
\pgfpathlineto{\pgfqpoint{6.333739in}{9.516920in}}%
\pgfpathlineto{\pgfqpoint{6.366012in}{9.543655in}}%
\pgfpathlineto{\pgfqpoint{6.397560in}{9.517919in}}%
\pgfpathlineto{\pgfqpoint{6.428414in}{9.509006in}}%
\pgfpathlineto{\pgfqpoint{6.458604in}{9.510280in}}%
\pgfpathlineto{\pgfqpoint{6.488159in}{9.488254in}}%
\pgfpathlineto{\pgfqpoint{6.517104in}{9.534240in}}%
\pgfpathlineto{\pgfqpoint{6.545465in}{9.509992in}}%
\pgfpathlineto{\pgfqpoint{6.573264in}{9.519010in}}%
\pgfpathlineto{\pgfqpoint{6.600523in}{9.547689in}}%
\pgfpathlineto{\pgfqpoint{6.627263in}{9.494975in}}%
\pgfpathlineto{\pgfqpoint{6.653503in}{9.488026in}}%
\pgfpathlineto{\pgfqpoint{6.679262in}{9.489569in}}%
\pgfpathlineto{\pgfqpoint{6.704556in}{9.497787in}}%
\pgfpathlineto{\pgfqpoint{6.729403in}{9.487321in}}%
\pgfpathlineto{\pgfqpoint{6.753818in}{9.487983in}}%
\pgfpathlineto{\pgfqpoint{6.777815in}{9.484844in}}%
\pgfpathlineto{\pgfqpoint{6.801409in}{9.479394in}}%
\pgfpathlineto{\pgfqpoint{6.824613in}{9.482058in}}%
\pgfpathlineto{\pgfqpoint{6.847439in}{9.474199in}}%
\pgfpathlineto{\pgfqpoint{6.869901in}{9.466215in}}%
\pgfpathlineto{\pgfqpoint{6.892008in}{9.481151in}}%
\pgfpathlineto{\pgfqpoint{6.913773in}{9.481944in}}%
\pgfpathlineto{\pgfqpoint{6.935206in}{9.464022in}}%
\pgfpathlineto{\pgfqpoint{6.956316in}{9.475714in}}%
\pgfpathlineto{\pgfqpoint{6.977113in}{9.471868in}}%
\pgfpathlineto{\pgfqpoint{6.997607in}{9.475191in}}%
\pgfpathlineto{\pgfqpoint{7.017806in}{9.478535in}}%
\pgfpathlineto{\pgfqpoint{7.037719in}{9.466401in}}%
\pgfpathlineto{\pgfqpoint{7.057353in}{9.468471in}}%
\pgfpathlineto{\pgfqpoint{7.076716in}{9.462250in}}%
\pgfpathlineto{\pgfqpoint{7.095816in}{9.459056in}}%
\pgfpathlineto{\pgfqpoint{7.114659in}{9.468008in}}%
\pgfpathlineto{\pgfqpoint{7.133253in}{9.473181in}}%
\pgfpathlineto{\pgfqpoint{7.151603in}{9.477179in}}%
\pgfpathlineto{\pgfqpoint{7.169717in}{9.501946in}}%
\pgfpathlineto{\pgfqpoint{7.187600in}{9.452682in}}%
\pgfpathlineto{\pgfqpoint{7.205258in}{9.476035in}}%
\pgfpathlineto{\pgfqpoint{7.222697in}{9.465383in}}%
\pgfpathlineto{\pgfqpoint{7.239922in}{9.462386in}}%
\pgfpathlineto{\pgfqpoint{7.256938in}{9.453721in}}%
\pgfpathlineto{\pgfqpoint{7.273750in}{9.456837in}}%
\pgfpathlineto{\pgfqpoint{7.290363in}{9.446518in}}%
\pgfpathlineto{\pgfqpoint{7.306782in}{9.451845in}}%
\pgfpathlineto{\pgfqpoint{7.323011in}{9.446773in}}%
\pgfpathlineto{\pgfqpoint{7.339055in}{9.438497in}}%
\pgfpathlineto{\pgfqpoint{7.354917in}{9.456178in}}%
\pgfpathlineto{\pgfqpoint{7.370603in}{9.439544in}}%
\pgfpathlineto{\pgfqpoint{7.386114in}{9.449664in}}%
\pgfpathlineto{\pgfqpoint{7.401457in}{9.443929in}}%
\pgfpathlineto{\pgfqpoint{7.416633in}{9.438407in}}%
\pgfpathlineto{\pgfqpoint{7.431647in}{9.442674in}}%
\pgfpathlineto{\pgfqpoint{7.446502in}{9.447582in}}%
\pgfpathlineto{\pgfqpoint{7.461202in}{9.453649in}}%
\pgfpathlineto{\pgfqpoint{7.475749in}{9.448711in}}%
\pgfpathlineto{\pgfqpoint{7.490148in}{9.443186in}}%
\pgfpathlineto{\pgfqpoint{7.504399in}{9.432660in}}%
\pgfpathlineto{\pgfqpoint{7.518508in}{9.429222in}}%
\pgfpathlineto{\pgfqpoint{7.518508in}{9.429222in}}%
\pgfpathlineto{\pgfqpoint{7.518508in}{9.429222in}}%
\pgfpathlineto{\pgfqpoint{7.504399in}{9.432660in}}%
\pgfpathlineto{\pgfqpoint{7.490148in}{9.443186in}}%
\pgfpathlineto{\pgfqpoint{7.475749in}{9.448711in}}%
\pgfpathlineto{\pgfqpoint{7.461202in}{9.453649in}}%
\pgfpathlineto{\pgfqpoint{7.446502in}{9.447582in}}%
\pgfpathlineto{\pgfqpoint{7.431647in}{9.442674in}}%
\pgfpathlineto{\pgfqpoint{7.416633in}{9.438407in}}%
\pgfpathlineto{\pgfqpoint{7.401457in}{9.443929in}}%
\pgfpathlineto{\pgfqpoint{7.386114in}{9.449664in}}%
\pgfpathlineto{\pgfqpoint{7.370603in}{9.439544in}}%
\pgfpathlineto{\pgfqpoint{7.354917in}{9.456178in}}%
\pgfpathlineto{\pgfqpoint{7.339055in}{9.438497in}}%
\pgfpathlineto{\pgfqpoint{7.323011in}{9.446773in}}%
\pgfpathlineto{\pgfqpoint{7.306782in}{9.451845in}}%
\pgfpathlineto{\pgfqpoint{7.290363in}{9.446518in}}%
\pgfpathlineto{\pgfqpoint{7.273750in}{9.456837in}}%
\pgfpathlineto{\pgfqpoint{7.256938in}{9.453721in}}%
\pgfpathlineto{\pgfqpoint{7.239922in}{9.462386in}}%
\pgfpathlineto{\pgfqpoint{7.222697in}{9.465383in}}%
\pgfpathlineto{\pgfqpoint{7.205258in}{9.476035in}}%
\pgfpathlineto{\pgfqpoint{7.187600in}{9.452682in}}%
\pgfpathlineto{\pgfqpoint{7.169717in}{9.501946in}}%
\pgfpathlineto{\pgfqpoint{7.151603in}{9.477179in}}%
\pgfpathlineto{\pgfqpoint{7.133253in}{9.473181in}}%
\pgfpathlineto{\pgfqpoint{7.114659in}{9.468008in}}%
\pgfpathlineto{\pgfqpoint{7.095816in}{9.459056in}}%
\pgfpathlineto{\pgfqpoint{7.076716in}{9.462250in}}%
\pgfpathlineto{\pgfqpoint{7.057353in}{9.468471in}}%
\pgfpathlineto{\pgfqpoint{7.037719in}{9.466401in}}%
\pgfpathlineto{\pgfqpoint{7.017806in}{9.478535in}}%
\pgfpathlineto{\pgfqpoint{6.997607in}{9.475191in}}%
\pgfpathlineto{\pgfqpoint{6.977113in}{9.471868in}}%
\pgfpathlineto{\pgfqpoint{6.956316in}{9.475714in}}%
\pgfpathlineto{\pgfqpoint{6.935206in}{9.464022in}}%
\pgfpathlineto{\pgfqpoint{6.913773in}{9.481944in}}%
\pgfpathlineto{\pgfqpoint{6.892008in}{9.481151in}}%
\pgfpathlineto{\pgfqpoint{6.869901in}{9.466215in}}%
\pgfpathlineto{\pgfqpoint{6.847439in}{9.474199in}}%
\pgfpathlineto{\pgfqpoint{6.824613in}{9.482058in}}%
\pgfpathlineto{\pgfqpoint{6.801409in}{9.479394in}}%
\pgfpathlineto{\pgfqpoint{6.777815in}{9.484844in}}%
\pgfpathlineto{\pgfqpoint{6.753818in}{9.487983in}}%
\pgfpathlineto{\pgfqpoint{6.729403in}{9.487321in}}%
\pgfpathlineto{\pgfqpoint{6.704556in}{9.497787in}}%
\pgfpathlineto{\pgfqpoint{6.679262in}{9.489569in}}%
\pgfpathlineto{\pgfqpoint{6.653503in}{9.488026in}}%
\pgfpathlineto{\pgfqpoint{6.627263in}{9.494975in}}%
\pgfpathlineto{\pgfqpoint{6.600523in}{9.547689in}}%
\pgfpathlineto{\pgfqpoint{6.573264in}{9.519010in}}%
\pgfpathlineto{\pgfqpoint{6.545465in}{9.509992in}}%
\pgfpathlineto{\pgfqpoint{6.517104in}{9.534240in}}%
\pgfpathlineto{\pgfqpoint{6.488159in}{9.488254in}}%
\pgfpathlineto{\pgfqpoint{6.458604in}{9.510280in}}%
\pgfpathlineto{\pgfqpoint{6.428414in}{9.509006in}}%
\pgfpathlineto{\pgfqpoint{6.397560in}{9.517919in}}%
\pgfpathlineto{\pgfqpoint{6.366012in}{9.543655in}}%
\pgfpathlineto{\pgfqpoint{6.333739in}{9.516920in}}%
\pgfpathlineto{\pgfqpoint{6.300707in}{9.539905in}}%
\pgfpathlineto{\pgfqpoint{6.266879in}{9.534773in}}%
\pgfpathlineto{\pgfqpoint{6.232215in}{9.554233in}}%
\pgfpathlineto{\pgfqpoint{6.196674in}{9.530824in}}%
\pgfpathlineto{\pgfqpoint{6.160209in}{9.532268in}}%
\pgfpathlineto{\pgfqpoint{6.122772in}{9.542073in}}%
\pgfpathlineto{\pgfqpoint{6.084310in}{9.551607in}}%
\pgfpathlineto{\pgfqpoint{6.044763in}{9.539180in}}%
\pgfpathlineto{\pgfqpoint{6.004070in}{9.542490in}}%
\pgfpathlineto{\pgfqpoint{5.962163in}{9.552038in}}%
\pgfpathlineto{\pgfqpoint{5.918965in}{9.568999in}}%
\pgfpathlineto{\pgfqpoint{5.874396in}{9.550554in}}%
\pgfpathlineto{\pgfqpoint{5.828366in}{9.662032in}}%
\pgfpathlineto{\pgfqpoint{5.780775in}{9.573263in}}%
\pgfpathlineto{\pgfqpoint{5.731513in}{9.553392in}}%
\pgfpathlineto{\pgfqpoint{5.680460in}{9.595843in}}%
\pgfpathlineto{\pgfqpoint{5.627480in}{9.573841in}}%
\pgfpathlineto{\pgfqpoint{5.572422in}{9.563488in}}%
\pgfpathlineto{\pgfqpoint{5.515116in}{9.581070in}}%
\pgfpathlineto{\pgfqpoint{5.455371in}{9.615126in}}%
\pgfpathlineto{\pgfqpoint{5.392969in}{9.596824in}}%
\pgfpathlineto{\pgfqpoint{5.327664in}{9.605838in}}%
\pgfpathlineto{\pgfqpoint{5.259172in}{9.592206in}}%
\pgfpathlineto{\pgfqpoint{5.187166in}{9.597787in}}%
\pgfpathlineto{\pgfqpoint{5.111267in}{9.632239in}}%
\pgfpathlineto{\pgfqpoint{5.031027in}{9.614074in}}%
\pgfpathlineto{\pgfqpoint{4.945922in}{9.639899in}}%
\pgfpathlineto{\pgfqpoint{4.855323in}{9.612005in}}%
\pgfpathlineto{\pgfqpoint{4.758470in}{9.684242in}}%
\pgfpathlineto{\pgfqpoint{4.654437in}{9.640369in}}%
\pgfpathlineto{\pgfqpoint{4.542073in}{9.638943in}}%
\pgfpathlineto{\pgfqpoint{4.419926in}{9.663988in}}%
\pgfpathlineto{\pgfqpoint{4.286129in}{9.689122in}}%
\pgfpathlineto{\pgfqpoint{4.138224in}{9.679293in}}%
\pgfpathlineto{\pgfqpoint{3.972879in}{9.676864in}}%
\pgfpathlineto{\pgfqpoint{3.785427in}{9.689323in}}%
\pgfpathlineto{\pgfqpoint{3.569030in}{9.719228in}}%
\pgfpathlineto{\pgfqpoint{3.313086in}{9.723886in}}%
\pgfpathlineto{\pgfqpoint{2.999836in}{9.790302in}}%
\pgfpathlineto{\pgfqpoint{2.595987in}{9.741821in}}%
\pgfpathlineto{\pgfqpoint{2.026793in}{9.832830in}}%
\pgfpathlineto{\pgfqpoint{1.053750in}{9.973369in}}%
\pgfpathlineto{\pgfqpoint{-3231.325327in}{12.343495in}}%
\pgfpathclose%
\pgfusepath{fill}%
\end{pgfscope}%
\begin{pgfscope}%
\pgfsetbuttcap%
\pgfsetroundjoin%
\definecolor{currentfill}{rgb}{0.000000,0.000000,0.000000}%
\pgfsetfillcolor{currentfill}%
\pgfsetlinewidth{0.803000pt}%
\definecolor{currentstroke}{rgb}{0.000000,0.000000,0.000000}%
\pgfsetstrokecolor{currentstroke}%
\pgfsetdash{}{0pt}%
\pgfsys@defobject{currentmarker}{\pgfqpoint{0.000000in}{-0.048611in}}{\pgfqpoint{0.000000in}{0.000000in}}{%
\pgfpathmoveto{\pgfqpoint{0.000000in}{0.000000in}}%
\pgfpathlineto{\pgfqpoint{0.000000in}{-0.048611in}}%
\pgfusepath{stroke,fill}%
}%
\begin{pgfscope}%
\pgfsys@transformshift{1.053750in}{9.410883in}%
\pgfsys@useobject{currentmarker}{}%
\end{pgfscope}%
\end{pgfscope}%
\begin{pgfscope}%
\pgfsetbuttcap%
\pgfsetroundjoin%
\definecolor{currentfill}{rgb}{0.000000,0.000000,0.000000}%
\pgfsetfillcolor{currentfill}%
\pgfsetlinewidth{0.803000pt}%
\definecolor{currentstroke}{rgb}{0.000000,0.000000,0.000000}%
\pgfsetstrokecolor{currentstroke}%
\pgfsetdash{}{0pt}%
\pgfsys@defobject{currentmarker}{\pgfqpoint{0.000000in}{-0.048611in}}{\pgfqpoint{0.000000in}{0.000000in}}{%
\pgfpathmoveto{\pgfqpoint{0.000000in}{0.000000in}}%
\pgfpathlineto{\pgfqpoint{0.000000in}{-0.048611in}}%
\pgfusepath{stroke,fill}%
}%
\begin{pgfscope}%
\pgfsys@transformshift{4.286129in}{9.410883in}%
\pgfsys@useobject{currentmarker}{}%
\end{pgfscope}%
\end{pgfscope}%
\begin{pgfscope}%
\pgfsetbuttcap%
\pgfsetroundjoin%
\definecolor{currentfill}{rgb}{0.000000,0.000000,0.000000}%
\pgfsetfillcolor{currentfill}%
\pgfsetlinewidth{0.803000pt}%
\definecolor{currentstroke}{rgb}{0.000000,0.000000,0.000000}%
\pgfsetstrokecolor{currentstroke}%
\pgfsetdash{}{0pt}%
\pgfsys@defobject{currentmarker}{\pgfqpoint{0.000000in}{-0.048611in}}{\pgfqpoint{0.000000in}{0.000000in}}{%
\pgfpathmoveto{\pgfqpoint{0.000000in}{0.000000in}}%
\pgfpathlineto{\pgfqpoint{0.000000in}{-0.048611in}}%
\pgfusepath{stroke,fill}%
}%
\begin{pgfscope}%
\pgfsys@transformshift{7.518508in}{9.410883in}%
\pgfsys@useobject{currentmarker}{}%
\end{pgfscope}%
\end{pgfscope}%
\begin{pgfscope}%
\pgfsetbuttcap%
\pgfsetroundjoin%
\definecolor{currentfill}{rgb}{0.000000,0.000000,0.000000}%
\pgfsetfillcolor{currentfill}%
\pgfsetlinewidth{0.602250pt}%
\definecolor{currentstroke}{rgb}{0.000000,0.000000,0.000000}%
\pgfsetstrokecolor{currentstroke}%
\pgfsetdash{}{0pt}%
\pgfsys@defobject{currentmarker}{\pgfqpoint{0.000000in}{-0.027778in}}{\pgfqpoint{0.000000in}{0.000000in}}{%
\pgfpathmoveto{\pgfqpoint{0.000000in}{0.000000in}}%
\pgfpathlineto{\pgfqpoint{0.000000in}{-0.027778in}}%
\pgfusepath{stroke,fill}%
}%
\begin{pgfscope}%
\pgfsys@transformshift{2.026793in}{9.410883in}%
\pgfsys@useobject{currentmarker}{}%
\end{pgfscope}%
\end{pgfscope}%
\begin{pgfscope}%
\pgfsetbuttcap%
\pgfsetroundjoin%
\definecolor{currentfill}{rgb}{0.000000,0.000000,0.000000}%
\pgfsetfillcolor{currentfill}%
\pgfsetlinewidth{0.602250pt}%
\definecolor{currentstroke}{rgb}{0.000000,0.000000,0.000000}%
\pgfsetstrokecolor{currentstroke}%
\pgfsetdash{}{0pt}%
\pgfsys@defobject{currentmarker}{\pgfqpoint{0.000000in}{-0.027778in}}{\pgfqpoint{0.000000in}{0.000000in}}{%
\pgfpathmoveto{\pgfqpoint{0.000000in}{0.000000in}}%
\pgfpathlineto{\pgfqpoint{0.000000in}{-0.027778in}}%
\pgfusepath{stroke,fill}%
}%
\begin{pgfscope}%
\pgfsys@transformshift{2.595987in}{9.410883in}%
\pgfsys@useobject{currentmarker}{}%
\end{pgfscope}%
\end{pgfscope}%
\begin{pgfscope}%
\pgfsetbuttcap%
\pgfsetroundjoin%
\definecolor{currentfill}{rgb}{0.000000,0.000000,0.000000}%
\pgfsetfillcolor{currentfill}%
\pgfsetlinewidth{0.602250pt}%
\definecolor{currentstroke}{rgb}{0.000000,0.000000,0.000000}%
\pgfsetstrokecolor{currentstroke}%
\pgfsetdash{}{0pt}%
\pgfsys@defobject{currentmarker}{\pgfqpoint{0.000000in}{-0.027778in}}{\pgfqpoint{0.000000in}{0.000000in}}{%
\pgfpathmoveto{\pgfqpoint{0.000000in}{0.000000in}}%
\pgfpathlineto{\pgfqpoint{0.000000in}{-0.027778in}}%
\pgfusepath{stroke,fill}%
}%
\begin{pgfscope}%
\pgfsys@transformshift{2.999836in}{9.410883in}%
\pgfsys@useobject{currentmarker}{}%
\end{pgfscope}%
\end{pgfscope}%
\begin{pgfscope}%
\pgfsetbuttcap%
\pgfsetroundjoin%
\definecolor{currentfill}{rgb}{0.000000,0.000000,0.000000}%
\pgfsetfillcolor{currentfill}%
\pgfsetlinewidth{0.602250pt}%
\definecolor{currentstroke}{rgb}{0.000000,0.000000,0.000000}%
\pgfsetstrokecolor{currentstroke}%
\pgfsetdash{}{0pt}%
\pgfsys@defobject{currentmarker}{\pgfqpoint{0.000000in}{-0.027778in}}{\pgfqpoint{0.000000in}{0.000000in}}{%
\pgfpathmoveto{\pgfqpoint{0.000000in}{0.000000in}}%
\pgfpathlineto{\pgfqpoint{0.000000in}{-0.027778in}}%
\pgfusepath{stroke,fill}%
}%
\begin{pgfscope}%
\pgfsys@transformshift{3.313086in}{9.410883in}%
\pgfsys@useobject{currentmarker}{}%
\end{pgfscope}%
\end{pgfscope}%
\begin{pgfscope}%
\pgfsetbuttcap%
\pgfsetroundjoin%
\definecolor{currentfill}{rgb}{0.000000,0.000000,0.000000}%
\pgfsetfillcolor{currentfill}%
\pgfsetlinewidth{0.602250pt}%
\definecolor{currentstroke}{rgb}{0.000000,0.000000,0.000000}%
\pgfsetstrokecolor{currentstroke}%
\pgfsetdash{}{0pt}%
\pgfsys@defobject{currentmarker}{\pgfqpoint{0.000000in}{-0.027778in}}{\pgfqpoint{0.000000in}{0.000000in}}{%
\pgfpathmoveto{\pgfqpoint{0.000000in}{0.000000in}}%
\pgfpathlineto{\pgfqpoint{0.000000in}{-0.027778in}}%
\pgfusepath{stroke,fill}%
}%
\begin{pgfscope}%
\pgfsys@transformshift{3.569030in}{9.410883in}%
\pgfsys@useobject{currentmarker}{}%
\end{pgfscope}%
\end{pgfscope}%
\begin{pgfscope}%
\pgfsetbuttcap%
\pgfsetroundjoin%
\definecolor{currentfill}{rgb}{0.000000,0.000000,0.000000}%
\pgfsetfillcolor{currentfill}%
\pgfsetlinewidth{0.602250pt}%
\definecolor{currentstroke}{rgb}{0.000000,0.000000,0.000000}%
\pgfsetstrokecolor{currentstroke}%
\pgfsetdash{}{0pt}%
\pgfsys@defobject{currentmarker}{\pgfqpoint{0.000000in}{-0.027778in}}{\pgfqpoint{0.000000in}{0.000000in}}{%
\pgfpathmoveto{\pgfqpoint{0.000000in}{0.000000in}}%
\pgfpathlineto{\pgfqpoint{0.000000in}{-0.027778in}}%
\pgfusepath{stroke,fill}%
}%
\begin{pgfscope}%
\pgfsys@transformshift{3.785427in}{9.410883in}%
\pgfsys@useobject{currentmarker}{}%
\end{pgfscope}%
\end{pgfscope}%
\begin{pgfscope}%
\pgfsetbuttcap%
\pgfsetroundjoin%
\definecolor{currentfill}{rgb}{0.000000,0.000000,0.000000}%
\pgfsetfillcolor{currentfill}%
\pgfsetlinewidth{0.602250pt}%
\definecolor{currentstroke}{rgb}{0.000000,0.000000,0.000000}%
\pgfsetstrokecolor{currentstroke}%
\pgfsetdash{}{0pt}%
\pgfsys@defobject{currentmarker}{\pgfqpoint{0.000000in}{-0.027778in}}{\pgfqpoint{0.000000in}{0.000000in}}{%
\pgfpathmoveto{\pgfqpoint{0.000000in}{0.000000in}}%
\pgfpathlineto{\pgfqpoint{0.000000in}{-0.027778in}}%
\pgfusepath{stroke,fill}%
}%
\begin{pgfscope}%
\pgfsys@transformshift{3.972879in}{9.410883in}%
\pgfsys@useobject{currentmarker}{}%
\end{pgfscope}%
\end{pgfscope}%
\begin{pgfscope}%
\pgfsetbuttcap%
\pgfsetroundjoin%
\definecolor{currentfill}{rgb}{0.000000,0.000000,0.000000}%
\pgfsetfillcolor{currentfill}%
\pgfsetlinewidth{0.602250pt}%
\definecolor{currentstroke}{rgb}{0.000000,0.000000,0.000000}%
\pgfsetstrokecolor{currentstroke}%
\pgfsetdash{}{0pt}%
\pgfsys@defobject{currentmarker}{\pgfqpoint{0.000000in}{-0.027778in}}{\pgfqpoint{0.000000in}{0.000000in}}{%
\pgfpathmoveto{\pgfqpoint{0.000000in}{0.000000in}}%
\pgfpathlineto{\pgfqpoint{0.000000in}{-0.027778in}}%
\pgfusepath{stroke,fill}%
}%
\begin{pgfscope}%
\pgfsys@transformshift{4.138224in}{9.410883in}%
\pgfsys@useobject{currentmarker}{}%
\end{pgfscope}%
\end{pgfscope}%
\begin{pgfscope}%
\pgfsetbuttcap%
\pgfsetroundjoin%
\definecolor{currentfill}{rgb}{0.000000,0.000000,0.000000}%
\pgfsetfillcolor{currentfill}%
\pgfsetlinewidth{0.602250pt}%
\definecolor{currentstroke}{rgb}{0.000000,0.000000,0.000000}%
\pgfsetstrokecolor{currentstroke}%
\pgfsetdash{}{0pt}%
\pgfsys@defobject{currentmarker}{\pgfqpoint{0.000000in}{-0.027778in}}{\pgfqpoint{0.000000in}{0.000000in}}{%
\pgfpathmoveto{\pgfqpoint{0.000000in}{0.000000in}}%
\pgfpathlineto{\pgfqpoint{0.000000in}{-0.027778in}}%
\pgfusepath{stroke,fill}%
}%
\begin{pgfscope}%
\pgfsys@transformshift{5.259172in}{9.410883in}%
\pgfsys@useobject{currentmarker}{}%
\end{pgfscope}%
\end{pgfscope}%
\begin{pgfscope}%
\pgfsetbuttcap%
\pgfsetroundjoin%
\definecolor{currentfill}{rgb}{0.000000,0.000000,0.000000}%
\pgfsetfillcolor{currentfill}%
\pgfsetlinewidth{0.602250pt}%
\definecolor{currentstroke}{rgb}{0.000000,0.000000,0.000000}%
\pgfsetstrokecolor{currentstroke}%
\pgfsetdash{}{0pt}%
\pgfsys@defobject{currentmarker}{\pgfqpoint{0.000000in}{-0.027778in}}{\pgfqpoint{0.000000in}{0.000000in}}{%
\pgfpathmoveto{\pgfqpoint{0.000000in}{0.000000in}}%
\pgfpathlineto{\pgfqpoint{0.000000in}{-0.027778in}}%
\pgfusepath{stroke,fill}%
}%
\begin{pgfscope}%
\pgfsys@transformshift{5.828366in}{9.410883in}%
\pgfsys@useobject{currentmarker}{}%
\end{pgfscope}%
\end{pgfscope}%
\begin{pgfscope}%
\pgfsetbuttcap%
\pgfsetroundjoin%
\definecolor{currentfill}{rgb}{0.000000,0.000000,0.000000}%
\pgfsetfillcolor{currentfill}%
\pgfsetlinewidth{0.602250pt}%
\definecolor{currentstroke}{rgb}{0.000000,0.000000,0.000000}%
\pgfsetstrokecolor{currentstroke}%
\pgfsetdash{}{0pt}%
\pgfsys@defobject{currentmarker}{\pgfqpoint{0.000000in}{-0.027778in}}{\pgfqpoint{0.000000in}{0.000000in}}{%
\pgfpathmoveto{\pgfqpoint{0.000000in}{0.000000in}}%
\pgfpathlineto{\pgfqpoint{0.000000in}{-0.027778in}}%
\pgfusepath{stroke,fill}%
}%
\begin{pgfscope}%
\pgfsys@transformshift{6.232215in}{9.410883in}%
\pgfsys@useobject{currentmarker}{}%
\end{pgfscope}%
\end{pgfscope}%
\begin{pgfscope}%
\pgfsetbuttcap%
\pgfsetroundjoin%
\definecolor{currentfill}{rgb}{0.000000,0.000000,0.000000}%
\pgfsetfillcolor{currentfill}%
\pgfsetlinewidth{0.602250pt}%
\definecolor{currentstroke}{rgb}{0.000000,0.000000,0.000000}%
\pgfsetstrokecolor{currentstroke}%
\pgfsetdash{}{0pt}%
\pgfsys@defobject{currentmarker}{\pgfqpoint{0.000000in}{-0.027778in}}{\pgfqpoint{0.000000in}{0.000000in}}{%
\pgfpathmoveto{\pgfqpoint{0.000000in}{0.000000in}}%
\pgfpathlineto{\pgfqpoint{0.000000in}{-0.027778in}}%
\pgfusepath{stroke,fill}%
}%
\begin{pgfscope}%
\pgfsys@transformshift{6.545465in}{9.410883in}%
\pgfsys@useobject{currentmarker}{}%
\end{pgfscope}%
\end{pgfscope}%
\begin{pgfscope}%
\pgfsetbuttcap%
\pgfsetroundjoin%
\definecolor{currentfill}{rgb}{0.000000,0.000000,0.000000}%
\pgfsetfillcolor{currentfill}%
\pgfsetlinewidth{0.602250pt}%
\definecolor{currentstroke}{rgb}{0.000000,0.000000,0.000000}%
\pgfsetstrokecolor{currentstroke}%
\pgfsetdash{}{0pt}%
\pgfsys@defobject{currentmarker}{\pgfqpoint{0.000000in}{-0.027778in}}{\pgfqpoint{0.000000in}{0.000000in}}{%
\pgfpathmoveto{\pgfqpoint{0.000000in}{0.000000in}}%
\pgfpathlineto{\pgfqpoint{0.000000in}{-0.027778in}}%
\pgfusepath{stroke,fill}%
}%
\begin{pgfscope}%
\pgfsys@transformshift{6.801409in}{9.410883in}%
\pgfsys@useobject{currentmarker}{}%
\end{pgfscope}%
\end{pgfscope}%
\begin{pgfscope}%
\pgfsetbuttcap%
\pgfsetroundjoin%
\definecolor{currentfill}{rgb}{0.000000,0.000000,0.000000}%
\pgfsetfillcolor{currentfill}%
\pgfsetlinewidth{0.602250pt}%
\definecolor{currentstroke}{rgb}{0.000000,0.000000,0.000000}%
\pgfsetstrokecolor{currentstroke}%
\pgfsetdash{}{0pt}%
\pgfsys@defobject{currentmarker}{\pgfqpoint{0.000000in}{-0.027778in}}{\pgfqpoint{0.000000in}{0.000000in}}{%
\pgfpathmoveto{\pgfqpoint{0.000000in}{0.000000in}}%
\pgfpathlineto{\pgfqpoint{0.000000in}{-0.027778in}}%
\pgfusepath{stroke,fill}%
}%
\begin{pgfscope}%
\pgfsys@transformshift{7.017806in}{9.410883in}%
\pgfsys@useobject{currentmarker}{}%
\end{pgfscope}%
\end{pgfscope}%
\begin{pgfscope}%
\pgfsetbuttcap%
\pgfsetroundjoin%
\definecolor{currentfill}{rgb}{0.000000,0.000000,0.000000}%
\pgfsetfillcolor{currentfill}%
\pgfsetlinewidth{0.602250pt}%
\definecolor{currentstroke}{rgb}{0.000000,0.000000,0.000000}%
\pgfsetstrokecolor{currentstroke}%
\pgfsetdash{}{0pt}%
\pgfsys@defobject{currentmarker}{\pgfqpoint{0.000000in}{-0.027778in}}{\pgfqpoint{0.000000in}{0.000000in}}{%
\pgfpathmoveto{\pgfqpoint{0.000000in}{0.000000in}}%
\pgfpathlineto{\pgfqpoint{0.000000in}{-0.027778in}}%
\pgfusepath{stroke,fill}%
}%
\begin{pgfscope}%
\pgfsys@transformshift{7.205258in}{9.410883in}%
\pgfsys@useobject{currentmarker}{}%
\end{pgfscope}%
\end{pgfscope}%
\begin{pgfscope}%
\pgfsetbuttcap%
\pgfsetroundjoin%
\definecolor{currentfill}{rgb}{0.000000,0.000000,0.000000}%
\pgfsetfillcolor{currentfill}%
\pgfsetlinewidth{0.602250pt}%
\definecolor{currentstroke}{rgb}{0.000000,0.000000,0.000000}%
\pgfsetstrokecolor{currentstroke}%
\pgfsetdash{}{0pt}%
\pgfsys@defobject{currentmarker}{\pgfqpoint{0.000000in}{-0.027778in}}{\pgfqpoint{0.000000in}{0.000000in}}{%
\pgfpathmoveto{\pgfqpoint{0.000000in}{0.000000in}}%
\pgfpathlineto{\pgfqpoint{0.000000in}{-0.027778in}}%
\pgfusepath{stroke,fill}%
}%
\begin{pgfscope}%
\pgfsys@transformshift{7.370603in}{9.410883in}%
\pgfsys@useobject{currentmarker}{}%
\end{pgfscope}%
\end{pgfscope}%
\begin{pgfscope}%
\pgfsetbuttcap%
\pgfsetroundjoin%
\definecolor{currentfill}{rgb}{0.000000,0.000000,0.000000}%
\pgfsetfillcolor{currentfill}%
\pgfsetlinewidth{0.803000pt}%
\definecolor{currentstroke}{rgb}{0.000000,0.000000,0.000000}%
\pgfsetstrokecolor{currentstroke}%
\pgfsetdash{}{0pt}%
\pgfsys@defobject{currentmarker}{\pgfqpoint{-0.048611in}{0.000000in}}{\pgfqpoint{0.000000in}{0.000000in}}{%
\pgfpathmoveto{\pgfqpoint{0.000000in}{0.000000in}}%
\pgfpathlineto{\pgfqpoint{-0.048611in}{0.000000in}}%
\pgfusepath{stroke,fill}%
}%
\begin{pgfscope}%
\pgfsys@transformshift{1.053750in}{9.953287in}%
\pgfsys@useobject{currentmarker}{}%
\end{pgfscope}%
\end{pgfscope}%
\begin{pgfscope}%
\definecolor{textcolor}{rgb}{0.000000,0.000000,0.000000}%
\pgfsetstrokecolor{textcolor}%
\pgfsetfillcolor{textcolor}%
\pgftext[x=0.691472in,y=9.889973in,left,base]{\color{textcolor}\sffamily\fontsize{12.000000}{14.400000}\selectfont 0.5}%
\end{pgfscope}%
\begin{pgfscope}%
\pgfsetbuttcap%
\pgfsetroundjoin%
\definecolor{currentfill}{rgb}{0.000000,0.000000,0.000000}%
\pgfsetfillcolor{currentfill}%
\pgfsetlinewidth{0.803000pt}%
\definecolor{currentstroke}{rgb}{0.000000,0.000000,0.000000}%
\pgfsetstrokecolor{currentstroke}%
\pgfsetdash{}{0pt}%
\pgfsys@defobject{currentmarker}{\pgfqpoint{-0.048611in}{0.000000in}}{\pgfqpoint{0.000000in}{0.000000in}}{%
\pgfpathmoveto{\pgfqpoint{0.000000in}{0.000000in}}%
\pgfpathlineto{\pgfqpoint{-0.048611in}{0.000000in}}%
\pgfusepath{stroke,fill}%
}%
\begin{pgfscope}%
\pgfsys@transformshift{1.053750in}{10.601461in}%
\pgfsys@useobject{currentmarker}{}%
\end{pgfscope}%
\end{pgfscope}%
\begin{pgfscope}%
\definecolor{textcolor}{rgb}{0.000000,0.000000,0.000000}%
\pgfsetstrokecolor{textcolor}%
\pgfsetfillcolor{textcolor}%
\pgftext[x=0.691472in,y=10.538147in,left,base]{\color{textcolor}\sffamily\fontsize{12.000000}{14.400000}\selectfont 1.0}%
\end{pgfscope}%
\begin{pgfscope}%
\pgfsetbuttcap%
\pgfsetroundjoin%
\definecolor{currentfill}{rgb}{0.000000,0.000000,0.000000}%
\pgfsetfillcolor{currentfill}%
\pgfsetlinewidth{0.803000pt}%
\definecolor{currentstroke}{rgb}{0.000000,0.000000,0.000000}%
\pgfsetstrokecolor{currentstroke}%
\pgfsetdash{}{0pt}%
\pgfsys@defobject{currentmarker}{\pgfqpoint{-0.048611in}{0.000000in}}{\pgfqpoint{0.000000in}{0.000000in}}{%
\pgfpathmoveto{\pgfqpoint{0.000000in}{0.000000in}}%
\pgfpathlineto{\pgfqpoint{-0.048611in}{0.000000in}}%
\pgfusepath{stroke,fill}%
}%
\begin{pgfscope}%
\pgfsys@transformshift{1.053750in}{11.249634in}%
\pgfsys@useobject{currentmarker}{}%
\end{pgfscope}%
\end{pgfscope}%
\begin{pgfscope}%
\definecolor{textcolor}{rgb}{0.000000,0.000000,0.000000}%
\pgfsetstrokecolor{textcolor}%
\pgfsetfillcolor{textcolor}%
\pgftext[x=0.691472in,y=11.186320in,left,base]{\color{textcolor}\sffamily\fontsize{12.000000}{14.400000}\selectfont 1.5}%
\end{pgfscope}%
\begin{pgfscope}%
\pgfsetbuttcap%
\pgfsetroundjoin%
\definecolor{currentfill}{rgb}{0.000000,0.000000,0.000000}%
\pgfsetfillcolor{currentfill}%
\pgfsetlinewidth{0.803000pt}%
\definecolor{currentstroke}{rgb}{0.000000,0.000000,0.000000}%
\pgfsetstrokecolor{currentstroke}%
\pgfsetdash{}{0pt}%
\pgfsys@defobject{currentmarker}{\pgfqpoint{-0.048611in}{0.000000in}}{\pgfqpoint{0.000000in}{0.000000in}}{%
\pgfpathmoveto{\pgfqpoint{0.000000in}{0.000000in}}%
\pgfpathlineto{\pgfqpoint{-0.048611in}{0.000000in}}%
\pgfusepath{stroke,fill}%
}%
\begin{pgfscope}%
\pgfsys@transformshift{1.053750in}{11.897807in}%
\pgfsys@useobject{currentmarker}{}%
\end{pgfscope}%
\end{pgfscope}%
\begin{pgfscope}%
\definecolor{textcolor}{rgb}{0.000000,0.000000,0.000000}%
\pgfsetstrokecolor{textcolor}%
\pgfsetfillcolor{textcolor}%
\pgftext[x=0.691472in,y=11.834494in,left,base]{\color{textcolor}\sffamily\fontsize{12.000000}{14.400000}\selectfont 2.0}%
\end{pgfscope}%
\begin{pgfscope}%
\definecolor{textcolor}{rgb}{0.000000,0.000000,0.000000}%
\pgfsetstrokecolor{textcolor}%
\pgfsetfillcolor{textcolor}%
\pgftext[x=0.635917in,y=10.950046in,,bottom,rotate=90.000000]{\color{textcolor}\sffamily\fontsize{14.000000}{16.800000}\selectfont train\_losses}%
\end{pgfscope}%
\begin{pgfscope}%
\pgfpathrectangle{\pgfqpoint{1.053750in}{9.410883in}}{\pgfqpoint{6.533250in}{3.078326in}}%
\pgfusepath{clip}%
\pgfsetrectcap%
\pgfsetroundjoin%
\pgfsetlinewidth{1.505625pt}%
\definecolor{currentstroke}{rgb}{0.121569,0.466667,0.705882}%
\pgfsetstrokecolor{currentstroke}%
\pgfsetdash{}{0pt}%
\pgfpathmoveto{\pgfqpoint{1.043750in}{9.944816in}}%
\pgfpathlineto{\pgfqpoint{1.053750in}{9.944809in}}%
\pgfpathlineto{\pgfqpoint{2.026793in}{9.840428in}}%
\pgfpathlineto{\pgfqpoint{2.595987in}{9.735962in}}%
\pgfpathlineto{\pgfqpoint{2.999836in}{9.769284in}}%
\pgfpathlineto{\pgfqpoint{3.313086in}{9.716391in}}%
\pgfpathlineto{\pgfqpoint{3.569030in}{9.719145in}}%
\pgfpathlineto{\pgfqpoint{3.785427in}{9.696848in}}%
\pgfpathlineto{\pgfqpoint{3.972879in}{9.677552in}}%
\pgfpathlineto{\pgfqpoint{4.138224in}{9.693759in}}%
\pgfpathlineto{\pgfqpoint{4.286129in}{9.672629in}}%
\pgfpathlineto{\pgfqpoint{4.419926in}{9.678337in}}%
\pgfpathlineto{\pgfqpoint{4.542073in}{9.650661in}}%
\pgfpathlineto{\pgfqpoint{4.654437in}{9.652328in}}%
\pgfpathlineto{\pgfqpoint{4.758470in}{9.669748in}}%
\pgfpathlineto{\pgfqpoint{4.855323in}{9.633836in}}%
\pgfpathlineto{\pgfqpoint{4.945922in}{9.649615in}}%
\pgfpathlineto{\pgfqpoint{5.031027in}{9.637636in}}%
\pgfpathlineto{\pgfqpoint{5.111267in}{9.633983in}}%
\pgfpathlineto{\pgfqpoint{5.187166in}{9.633934in}}%
\pgfpathlineto{\pgfqpoint{5.259172in}{9.616556in}}%
\pgfpathlineto{\pgfqpoint{5.327664in}{9.638170in}}%
\pgfpathlineto{\pgfqpoint{5.392969in}{9.627082in}}%
\pgfpathlineto{\pgfqpoint{5.455371in}{9.631392in}}%
\pgfpathlineto{\pgfqpoint{5.515116in}{9.621441in}}%
\pgfpathlineto{\pgfqpoint{5.572422in}{9.602430in}}%
\pgfpathlineto{\pgfqpoint{5.627480in}{9.617916in}}%
\pgfpathlineto{\pgfqpoint{5.680460in}{9.604315in}}%
\pgfpathlineto{\pgfqpoint{5.731513in}{9.597631in}}%
\pgfpathlineto{\pgfqpoint{5.780775in}{9.613249in}}%
\pgfpathlineto{\pgfqpoint{5.828366in}{9.615851in}}%
\pgfpathlineto{\pgfqpoint{5.874396in}{9.602847in}}%
\pgfpathlineto{\pgfqpoint{5.918965in}{9.602438in}}%
\pgfpathlineto{\pgfqpoint{5.962163in}{9.609095in}}%
\pgfpathlineto{\pgfqpoint{6.004070in}{9.594005in}}%
\pgfpathlineto{\pgfqpoint{6.044763in}{9.593257in}}%
\pgfpathlineto{\pgfqpoint{6.084310in}{9.597558in}}%
\pgfpathlineto{\pgfqpoint{6.122772in}{9.592725in}}%
\pgfpathlineto{\pgfqpoint{6.160209in}{9.583759in}}%
\pgfpathlineto{\pgfqpoint{6.196674in}{9.588108in}}%
\pgfpathlineto{\pgfqpoint{6.232215in}{9.600423in}}%
\pgfpathlineto{\pgfqpoint{6.266879in}{9.591261in}}%
\pgfpathlineto{\pgfqpoint{6.300707in}{9.601181in}}%
\pgfpathlineto{\pgfqpoint{6.333739in}{9.581341in}}%
\pgfpathlineto{\pgfqpoint{6.366012in}{9.583077in}}%
\pgfpathlineto{\pgfqpoint{6.397560in}{9.586008in}}%
\pgfpathlineto{\pgfqpoint{6.428414in}{9.569952in}}%
\pgfpathlineto{\pgfqpoint{6.458604in}{9.583962in}}%
\pgfpathlineto{\pgfqpoint{6.488159in}{9.557540in}}%
\pgfpathlineto{\pgfqpoint{6.517104in}{9.583034in}}%
\pgfpathlineto{\pgfqpoint{6.545465in}{9.588564in}}%
\pgfpathlineto{\pgfqpoint{6.573264in}{9.576872in}}%
\pgfpathlineto{\pgfqpoint{6.600523in}{9.570133in}}%
\pgfpathlineto{\pgfqpoint{6.627263in}{9.569768in}}%
\pgfpathlineto{\pgfqpoint{6.653503in}{9.560396in}}%
\pgfpathlineto{\pgfqpoint{6.679262in}{9.564247in}}%
\pgfpathlineto{\pgfqpoint{6.704556in}{9.571104in}}%
\pgfpathlineto{\pgfqpoint{6.729403in}{9.561079in}}%
\pgfpathlineto{\pgfqpoint{6.753818in}{9.561255in}}%
\pgfpathlineto{\pgfqpoint{6.777815in}{9.560196in}}%
\pgfpathlineto{\pgfqpoint{6.801409in}{9.553259in}}%
\pgfpathlineto{\pgfqpoint{6.824613in}{9.567719in}}%
\pgfpathlineto{\pgfqpoint{6.847439in}{9.547377in}}%
\pgfpathlineto{\pgfqpoint{6.869901in}{9.549214in}}%
\pgfpathlineto{\pgfqpoint{6.892008in}{9.556721in}}%
\pgfpathlineto{\pgfqpoint{6.913773in}{9.562780in}}%
\pgfpathlineto{\pgfqpoint{6.935206in}{9.548756in}}%
\pgfpathlineto{\pgfqpoint{6.956316in}{9.549631in}}%
\pgfpathlineto{\pgfqpoint{6.977113in}{9.553052in}}%
\pgfpathlineto{\pgfqpoint{6.997607in}{9.557816in}}%
\pgfpathlineto{\pgfqpoint{7.017806in}{9.554052in}}%
\pgfpathlineto{\pgfqpoint{7.037719in}{9.548522in}}%
\pgfpathlineto{\pgfqpoint{7.057353in}{9.556158in}}%
\pgfpathlineto{\pgfqpoint{7.076716in}{9.556790in}}%
\pgfpathlineto{\pgfqpoint{7.095816in}{9.540149in}}%
\pgfpathlineto{\pgfqpoint{7.114659in}{9.548066in}}%
\pgfpathlineto{\pgfqpoint{7.133253in}{9.558214in}}%
\pgfpathlineto{\pgfqpoint{7.151603in}{9.541294in}}%
\pgfpathlineto{\pgfqpoint{7.169717in}{9.571030in}}%
\pgfpathlineto{\pgfqpoint{7.187600in}{9.545709in}}%
\pgfpathlineto{\pgfqpoint{7.205258in}{9.547216in}}%
\pgfpathlineto{\pgfqpoint{7.222697in}{9.555695in}}%
\pgfpathlineto{\pgfqpoint{7.239922in}{9.541849in}}%
\pgfpathlineto{\pgfqpoint{7.256938in}{9.537807in}}%
\pgfpathlineto{\pgfqpoint{7.273750in}{9.536034in}}%
\pgfpathlineto{\pgfqpoint{7.290363in}{9.533695in}}%
\pgfpathlineto{\pgfqpoint{7.306782in}{9.545084in}}%
\pgfpathlineto{\pgfqpoint{7.323011in}{9.534066in}}%
\pgfpathlineto{\pgfqpoint{7.339055in}{9.523415in}}%
\pgfpathlineto{\pgfqpoint{7.354917in}{9.544005in}}%
\pgfpathlineto{\pgfqpoint{7.370603in}{9.528367in}}%
\pgfpathlineto{\pgfqpoint{7.386114in}{9.535312in}}%
\pgfpathlineto{\pgfqpoint{7.401457in}{9.533888in}}%
\pgfpathlineto{\pgfqpoint{7.416633in}{9.532387in}}%
\pgfpathlineto{\pgfqpoint{7.431647in}{9.520151in}}%
\pgfpathlineto{\pgfqpoint{7.446502in}{9.532402in}}%
\pgfpathlineto{\pgfqpoint{7.461202in}{9.541107in}}%
\pgfpathlineto{\pgfqpoint{7.475749in}{9.525121in}}%
\pgfpathlineto{\pgfqpoint{7.490148in}{9.537370in}}%
\pgfpathlineto{\pgfqpoint{7.504399in}{9.519571in}}%
\pgfpathlineto{\pgfqpoint{7.518508in}{9.520081in}}%
\pgfusepath{stroke}%
\end{pgfscope}%
\begin{pgfscope}%
\pgfpathrectangle{\pgfqpoint{1.053750in}{9.410883in}}{\pgfqpoint{6.533250in}{3.078326in}}%
\pgfusepath{clip}%
\pgfsetrectcap%
\pgfsetroundjoin%
\pgfsetlinewidth{1.505625pt}%
\definecolor{currentstroke}{rgb}{1.000000,0.498039,0.054902}%
\pgfsetstrokecolor{currentstroke}%
\pgfsetdash{}{0pt}%
\pgfpathmoveto{\pgfqpoint{1.043750in}{9.973376in}}%
\pgfpathlineto{\pgfqpoint{1.053750in}{9.973369in}}%
\pgfpathlineto{\pgfqpoint{2.026793in}{9.832830in}}%
\pgfpathlineto{\pgfqpoint{2.595987in}{9.741821in}}%
\pgfpathlineto{\pgfqpoint{2.999836in}{9.790302in}}%
\pgfpathlineto{\pgfqpoint{3.313086in}{9.723886in}}%
\pgfpathlineto{\pgfqpoint{3.569030in}{9.719228in}}%
\pgfpathlineto{\pgfqpoint{3.785427in}{9.689323in}}%
\pgfpathlineto{\pgfqpoint{3.972879in}{9.676864in}}%
\pgfpathlineto{\pgfqpoint{4.138224in}{9.679293in}}%
\pgfpathlineto{\pgfqpoint{4.286129in}{9.689122in}}%
\pgfpathlineto{\pgfqpoint{4.419926in}{9.663988in}}%
\pgfpathlineto{\pgfqpoint{4.542073in}{9.638943in}}%
\pgfpathlineto{\pgfqpoint{4.654437in}{9.640369in}}%
\pgfpathlineto{\pgfqpoint{4.758470in}{9.684242in}}%
\pgfpathlineto{\pgfqpoint{4.855323in}{9.612005in}}%
\pgfpathlineto{\pgfqpoint{4.945922in}{9.639899in}}%
\pgfpathlineto{\pgfqpoint{5.031027in}{9.614074in}}%
\pgfpathlineto{\pgfqpoint{5.111267in}{9.632239in}}%
\pgfpathlineto{\pgfqpoint{5.187166in}{9.597787in}}%
\pgfpathlineto{\pgfqpoint{5.259172in}{9.592206in}}%
\pgfpathlineto{\pgfqpoint{5.327664in}{9.605838in}}%
\pgfpathlineto{\pgfqpoint{5.392969in}{9.596824in}}%
\pgfpathlineto{\pgfqpoint{5.455371in}{9.615126in}}%
\pgfpathlineto{\pgfqpoint{5.515116in}{9.581070in}}%
\pgfpathlineto{\pgfqpoint{5.572422in}{9.563488in}}%
\pgfpathlineto{\pgfqpoint{5.627480in}{9.573841in}}%
\pgfpathlineto{\pgfqpoint{5.680460in}{9.595843in}}%
\pgfpathlineto{\pgfqpoint{5.731513in}{9.553392in}}%
\pgfpathlineto{\pgfqpoint{5.780775in}{9.573263in}}%
\pgfpathlineto{\pgfqpoint{5.828366in}{9.662032in}}%
\pgfpathlineto{\pgfqpoint{5.874396in}{9.550554in}}%
\pgfpathlineto{\pgfqpoint{5.918965in}{9.568999in}}%
\pgfpathlineto{\pgfqpoint{5.962163in}{9.552038in}}%
\pgfpathlineto{\pgfqpoint{6.004070in}{9.542490in}}%
\pgfpathlineto{\pgfqpoint{6.044763in}{9.539180in}}%
\pgfpathlineto{\pgfqpoint{6.084310in}{9.551607in}}%
\pgfpathlineto{\pgfqpoint{6.122772in}{9.542073in}}%
\pgfpathlineto{\pgfqpoint{6.160209in}{9.532268in}}%
\pgfpathlineto{\pgfqpoint{6.196674in}{9.530824in}}%
\pgfpathlineto{\pgfqpoint{6.232215in}{9.554233in}}%
\pgfpathlineto{\pgfqpoint{6.266879in}{9.534773in}}%
\pgfpathlineto{\pgfqpoint{6.300707in}{9.539905in}}%
\pgfpathlineto{\pgfqpoint{6.333739in}{9.516920in}}%
\pgfpathlineto{\pgfqpoint{6.366012in}{9.543655in}}%
\pgfpathlineto{\pgfqpoint{6.397560in}{9.517919in}}%
\pgfpathlineto{\pgfqpoint{6.428414in}{9.509006in}}%
\pgfpathlineto{\pgfqpoint{6.458604in}{9.510280in}}%
\pgfpathlineto{\pgfqpoint{6.488159in}{9.488254in}}%
\pgfpathlineto{\pgfqpoint{6.517104in}{9.534240in}}%
\pgfpathlineto{\pgfqpoint{6.545465in}{9.509992in}}%
\pgfpathlineto{\pgfqpoint{6.573264in}{9.519010in}}%
\pgfpathlineto{\pgfqpoint{6.600523in}{9.547689in}}%
\pgfpathlineto{\pgfqpoint{6.627263in}{9.494975in}}%
\pgfpathlineto{\pgfqpoint{6.653503in}{9.488026in}}%
\pgfpathlineto{\pgfqpoint{6.679262in}{9.489569in}}%
\pgfpathlineto{\pgfqpoint{6.704556in}{9.497787in}}%
\pgfpathlineto{\pgfqpoint{6.729403in}{9.487321in}}%
\pgfpathlineto{\pgfqpoint{6.753818in}{9.487983in}}%
\pgfpathlineto{\pgfqpoint{6.777815in}{9.484844in}}%
\pgfpathlineto{\pgfqpoint{6.801409in}{9.479394in}}%
\pgfpathlineto{\pgfqpoint{6.824613in}{9.482058in}}%
\pgfpathlineto{\pgfqpoint{6.847439in}{9.474199in}}%
\pgfpathlineto{\pgfqpoint{6.869901in}{9.466215in}}%
\pgfpathlineto{\pgfqpoint{6.892008in}{9.481151in}}%
\pgfpathlineto{\pgfqpoint{6.913773in}{9.481944in}}%
\pgfpathlineto{\pgfqpoint{6.935206in}{9.464022in}}%
\pgfpathlineto{\pgfqpoint{6.956316in}{9.475714in}}%
\pgfpathlineto{\pgfqpoint{6.977113in}{9.471868in}}%
\pgfpathlineto{\pgfqpoint{6.997607in}{9.475191in}}%
\pgfpathlineto{\pgfqpoint{7.017806in}{9.478535in}}%
\pgfpathlineto{\pgfqpoint{7.037719in}{9.466401in}}%
\pgfpathlineto{\pgfqpoint{7.057353in}{9.468471in}}%
\pgfpathlineto{\pgfqpoint{7.076716in}{9.462250in}}%
\pgfpathlineto{\pgfqpoint{7.095816in}{9.459056in}}%
\pgfpathlineto{\pgfqpoint{7.114659in}{9.468008in}}%
\pgfpathlineto{\pgfqpoint{7.133253in}{9.473181in}}%
\pgfpathlineto{\pgfqpoint{7.151603in}{9.477179in}}%
\pgfpathlineto{\pgfqpoint{7.169717in}{9.501946in}}%
\pgfpathlineto{\pgfqpoint{7.187600in}{9.452682in}}%
\pgfpathlineto{\pgfqpoint{7.205258in}{9.476035in}}%
\pgfpathlineto{\pgfqpoint{7.222697in}{9.465383in}}%
\pgfpathlineto{\pgfqpoint{7.239922in}{9.462386in}}%
\pgfpathlineto{\pgfqpoint{7.256938in}{9.453721in}}%
\pgfpathlineto{\pgfqpoint{7.273750in}{9.456837in}}%
\pgfpathlineto{\pgfqpoint{7.290363in}{9.446518in}}%
\pgfpathlineto{\pgfqpoint{7.306782in}{9.451845in}}%
\pgfpathlineto{\pgfqpoint{7.323011in}{9.446773in}}%
\pgfpathlineto{\pgfqpoint{7.339055in}{9.438497in}}%
\pgfpathlineto{\pgfqpoint{7.354917in}{9.456178in}}%
\pgfpathlineto{\pgfqpoint{7.370603in}{9.439544in}}%
\pgfpathlineto{\pgfqpoint{7.386114in}{9.449664in}}%
\pgfpathlineto{\pgfqpoint{7.401457in}{9.443929in}}%
\pgfpathlineto{\pgfqpoint{7.416633in}{9.438407in}}%
\pgfpathlineto{\pgfqpoint{7.431647in}{9.442674in}}%
\pgfpathlineto{\pgfqpoint{7.446502in}{9.447582in}}%
\pgfpathlineto{\pgfqpoint{7.461202in}{9.453649in}}%
\pgfpathlineto{\pgfqpoint{7.475749in}{9.448711in}}%
\pgfpathlineto{\pgfqpoint{7.490148in}{9.443186in}}%
\pgfpathlineto{\pgfqpoint{7.504399in}{9.432660in}}%
\pgfpathlineto{\pgfqpoint{7.518508in}{9.429222in}}%
\pgfusepath{stroke}%
\end{pgfscope}%
\begin{pgfscope}%
\pgfsetrectcap%
\pgfsetmiterjoin%
\pgfsetlinewidth{0.803000pt}%
\definecolor{currentstroke}{rgb}{0.000000,0.000000,0.000000}%
\pgfsetstrokecolor{currentstroke}%
\pgfsetdash{}{0pt}%
\pgfpathmoveto{\pgfqpoint{1.053750in}{9.410883in}}%
\pgfpathlineto{\pgfqpoint{1.053750in}{12.489209in}}%
\pgfusepath{stroke}%
\end{pgfscope}%
\begin{pgfscope}%
\pgfsetrectcap%
\pgfsetmiterjoin%
\pgfsetlinewidth{0.803000pt}%
\definecolor{currentstroke}{rgb}{0.000000,0.000000,0.000000}%
\pgfsetstrokecolor{currentstroke}%
\pgfsetdash{}{0pt}%
\pgfpathmoveto{\pgfqpoint{7.587000in}{9.410883in}}%
\pgfpathlineto{\pgfqpoint{7.587000in}{12.489209in}}%
\pgfusepath{stroke}%
\end{pgfscope}%
\begin{pgfscope}%
\pgfsetrectcap%
\pgfsetmiterjoin%
\pgfsetlinewidth{0.803000pt}%
\definecolor{currentstroke}{rgb}{0.000000,0.000000,0.000000}%
\pgfsetstrokecolor{currentstroke}%
\pgfsetdash{}{0pt}%
\pgfpathmoveto{\pgfqpoint{1.053750in}{9.410883in}}%
\pgfpathlineto{\pgfqpoint{7.587000in}{9.410883in}}%
\pgfusepath{stroke}%
\end{pgfscope}%
\begin{pgfscope}%
\pgfsetrectcap%
\pgfsetmiterjoin%
\pgfsetlinewidth{0.803000pt}%
\definecolor{currentstroke}{rgb}{0.000000,0.000000,0.000000}%
\pgfsetstrokecolor{currentstroke}%
\pgfsetdash{}{0pt}%
\pgfpathmoveto{\pgfqpoint{1.053750in}{12.489209in}}%
\pgfpathlineto{\pgfqpoint{7.587000in}{12.489209in}}%
\pgfusepath{stroke}%
\end{pgfscope}%
\begin{pgfscope}%
\pgfsetbuttcap%
\pgfsetmiterjoin%
\definecolor{currentfill}{rgb}{1.000000,1.000000,1.000000}%
\pgfsetfillcolor{currentfill}%
\pgfsetlinewidth{0.000000pt}%
\definecolor{currentstroke}{rgb}{0.000000,0.000000,0.000000}%
\pgfsetstrokecolor{currentstroke}%
\pgfsetstrokeopacity{0.000000}%
\pgfsetdash{}{0pt}%
\pgfpathmoveto{\pgfqpoint{1.053750in}{5.716891in}}%
\pgfpathlineto{\pgfqpoint{7.587000in}{5.716891in}}%
\pgfpathlineto{\pgfqpoint{7.587000in}{8.795217in}}%
\pgfpathlineto{\pgfqpoint{1.053750in}{8.795217in}}%
\pgfpathclose%
\pgfusepath{fill}%
\end{pgfscope}%
\begin{pgfscope}%
\pgfpathrectangle{\pgfqpoint{1.053750in}{5.716891in}}{\pgfqpoint{6.533250in}{3.078326in}}%
\pgfusepath{clip}%
\pgfsetbuttcap%
\pgfsetroundjoin%
\definecolor{currentfill}{rgb}{0.121569,0.466667,0.705882}%
\pgfsetfillcolor{currentfill}%
\pgfsetfillopacity{0.300000}%
\pgfsetlinewidth{0.000000pt}%
\definecolor{currentstroke}{rgb}{0.000000,0.000000,0.000000}%
\pgfsetstrokecolor{currentstroke}%
\pgfsetdash{}{0pt}%
\pgfpathmoveto{\pgfqpoint{-3231.325327in}{1.870504in}}%
\pgfpathlineto{\pgfqpoint{-3231.325327in}{1.870504in}}%
\pgfpathlineto{\pgfqpoint{1.053750in}{7.704137in}}%
\pgfpathlineto{\pgfqpoint{2.026793in}{7.877646in}}%
\pgfpathlineto{\pgfqpoint{2.595987in}{8.073858in}}%
\pgfpathlineto{\pgfqpoint{2.999836in}{8.079533in}}%
\pgfpathlineto{\pgfqpoint{3.313086in}{8.138721in}}%
\pgfpathlineto{\pgfqpoint{3.569030in}{8.193855in}}%
\pgfpathlineto{\pgfqpoint{3.785427in}{8.183314in}}%
\pgfpathlineto{\pgfqpoint{3.972879in}{8.201962in}}%
\pgfpathlineto{\pgfqpoint{4.138224in}{8.221421in}}%
\pgfpathlineto{\pgfqpoint{4.286129in}{8.254664in}}%
\pgfpathlineto{\pgfqpoint{4.419926in}{8.243313in}}%
\pgfpathlineto{\pgfqpoint{4.542073in}{8.262772in}}%
\pgfpathlineto{\pgfqpoint{4.654437in}{8.264393in}}%
\pgfpathlineto{\pgfqpoint{4.758470in}{8.255475in}}%
\pgfpathlineto{\pgfqpoint{4.855323in}{8.290339in}}%
\pgfpathlineto{\pgfqpoint{4.945922in}{8.272501in}}%
\pgfpathlineto{\pgfqpoint{5.031027in}{8.278987in}}%
\pgfpathlineto{\pgfqpoint{5.111267in}{8.282231in}}%
\pgfpathlineto{\pgfqpoint{5.187166in}{8.297636in}}%
\pgfpathlineto{\pgfqpoint{5.259172in}{8.315473in}}%
\pgfpathlineto{\pgfqpoint{5.327664in}{8.306554in}}%
\pgfpathlineto{\pgfqpoint{5.392969in}{8.314662in}}%
\pgfpathlineto{\pgfqpoint{5.455371in}{8.330878in}}%
\pgfpathlineto{\pgfqpoint{5.515116in}{8.328446in}}%
\pgfpathlineto{\pgfqpoint{5.572422in}{8.338986in}}%
\pgfpathlineto{\pgfqpoint{5.627480in}{8.321959in}}%
\pgfpathlineto{\pgfqpoint{5.680460in}{8.329256in}}%
\pgfpathlineto{\pgfqpoint{5.731513in}{8.322770in}}%
\pgfpathlineto{\pgfqpoint{5.780775in}{8.347905in}}%
\pgfpathlineto{\pgfqpoint{5.828366in}{8.314662in}}%
\pgfpathlineto{\pgfqpoint{5.874396in}{8.353580in}}%
\pgfpathlineto{\pgfqpoint{5.918965in}{8.346283in}}%
\pgfpathlineto{\pgfqpoint{5.962163in}{8.341418in}}%
\pgfpathlineto{\pgfqpoint{6.004070in}{8.357634in}}%
\pgfpathlineto{\pgfqpoint{6.044763in}{8.361688in}}%
\pgfpathlineto{\pgfqpoint{6.084310in}{8.358445in}}%
\pgfpathlineto{\pgfqpoint{6.122772in}{8.373039in}}%
\pgfpathlineto{\pgfqpoint{6.160209in}{8.380336in}}%
\pgfpathlineto{\pgfqpoint{6.196674in}{8.354391in}}%
\pgfpathlineto{\pgfqpoint{6.232215in}{8.353580in}}%
\pgfpathlineto{\pgfqpoint{6.266879in}{8.377904in}}%
\pgfpathlineto{\pgfqpoint{6.300707in}{8.376282in}}%
\pgfpathlineto{\pgfqpoint{6.333739in}{8.360066in}}%
\pgfpathlineto{\pgfqpoint{6.366012in}{8.381147in}}%
\pgfpathlineto{\pgfqpoint{6.397560in}{8.370607in}}%
\pgfpathlineto{\pgfqpoint{6.428414in}{8.389255in}}%
\pgfpathlineto{\pgfqpoint{6.458604in}{8.376282in}}%
\pgfpathlineto{\pgfqpoint{6.488159in}{8.388444in}}%
\pgfpathlineto{\pgfqpoint{6.517104in}{8.380336in}}%
\pgfpathlineto{\pgfqpoint{6.545465in}{8.394930in}}%
\pgfpathlineto{\pgfqpoint{6.573264in}{8.380336in}}%
\pgfpathlineto{\pgfqpoint{6.600523in}{8.385201in}}%
\pgfpathlineto{\pgfqpoint{6.627263in}{8.381147in}}%
\pgfpathlineto{\pgfqpoint{6.653503in}{8.390066in}}%
\pgfpathlineto{\pgfqpoint{6.679262in}{8.377904in}}%
\pgfpathlineto{\pgfqpoint{6.704556in}{8.376282in}}%
\pgfpathlineto{\pgfqpoint{6.729403in}{8.394120in}}%
\pgfpathlineto{\pgfqpoint{6.753818in}{8.405471in}}%
\pgfpathlineto{\pgfqpoint{6.777815in}{8.389255in}}%
\pgfpathlineto{\pgfqpoint{6.801409in}{8.401417in}}%
\pgfpathlineto{\pgfqpoint{6.824613in}{8.390876in}}%
\pgfpathlineto{\pgfqpoint{6.847439in}{8.390876in}}%
\pgfpathlineto{\pgfqpoint{6.869901in}{8.399795in}}%
\pgfpathlineto{\pgfqpoint{6.892008in}{8.391687in}}%
\pgfpathlineto{\pgfqpoint{6.913773in}{8.397363in}}%
\pgfpathlineto{\pgfqpoint{6.935206in}{8.403849in}}%
\pgfpathlineto{\pgfqpoint{6.956316in}{8.403038in}}%
\pgfpathlineto{\pgfqpoint{6.977113in}{8.398984in}}%
\pgfpathlineto{\pgfqpoint{6.997607in}{8.397363in}}%
\pgfpathlineto{\pgfqpoint{7.017806in}{8.392498in}}%
\pgfpathlineto{\pgfqpoint{7.037719in}{8.411146in}}%
\pgfpathlineto{\pgfqpoint{7.057353in}{8.380336in}}%
\pgfpathlineto{\pgfqpoint{7.076716in}{8.377904in}}%
\pgfpathlineto{\pgfqpoint{7.095816in}{8.393309in}}%
\pgfpathlineto{\pgfqpoint{7.114659in}{8.403038in}}%
\pgfpathlineto{\pgfqpoint{7.133253in}{8.399795in}}%
\pgfpathlineto{\pgfqpoint{7.151603in}{8.398984in}}%
\pgfpathlineto{\pgfqpoint{7.169717in}{8.342229in}}%
\pgfpathlineto{\pgfqpoint{7.187600in}{8.416822in}}%
\pgfpathlineto{\pgfqpoint{7.205258in}{8.406281in}}%
\pgfpathlineto{\pgfqpoint{7.222697in}{8.414389in}}%
\pgfpathlineto{\pgfqpoint{7.239922in}{8.409525in}}%
\pgfpathlineto{\pgfqpoint{7.256938in}{8.411146in}}%
\pgfpathlineto{\pgfqpoint{7.273750in}{8.401417in}}%
\pgfpathlineto{\pgfqpoint{7.290363in}{8.403038in}}%
\pgfpathlineto{\pgfqpoint{7.306782in}{8.400606in}}%
\pgfpathlineto{\pgfqpoint{7.323011in}{8.417633in}}%
\pgfpathlineto{\pgfqpoint{7.339055in}{8.404660in}}%
\pgfpathlineto{\pgfqpoint{7.354917in}{8.420876in}}%
\pgfpathlineto{\pgfqpoint{7.370603in}{8.413579in}}%
\pgfpathlineto{\pgfqpoint{7.386114in}{8.416011in}}%
\pgfpathlineto{\pgfqpoint{7.401457in}{8.430605in}}%
\pgfpathlineto{\pgfqpoint{7.416633in}{8.416822in}}%
\pgfpathlineto{\pgfqpoint{7.431647in}{8.415200in}}%
\pgfpathlineto{\pgfqpoint{7.446502in}{8.411957in}}%
\pgfpathlineto{\pgfqpoint{7.461202in}{8.417633in}}%
\pgfpathlineto{\pgfqpoint{7.475749in}{8.418443in}}%
\pgfpathlineto{\pgfqpoint{7.490148in}{8.403038in}}%
\pgfpathlineto{\pgfqpoint{7.504399in}{8.423308in}}%
\pgfpathlineto{\pgfqpoint{7.518508in}{8.406281in}}%
\pgfpathlineto{\pgfqpoint{7.518508in}{8.406281in}}%
\pgfpathlineto{\pgfqpoint{7.518508in}{8.406281in}}%
\pgfpathlineto{\pgfqpoint{7.504399in}{8.423308in}}%
\pgfpathlineto{\pgfqpoint{7.490148in}{8.403038in}}%
\pgfpathlineto{\pgfqpoint{7.475749in}{8.418443in}}%
\pgfpathlineto{\pgfqpoint{7.461202in}{8.417633in}}%
\pgfpathlineto{\pgfqpoint{7.446502in}{8.411957in}}%
\pgfpathlineto{\pgfqpoint{7.431647in}{8.415200in}}%
\pgfpathlineto{\pgfqpoint{7.416633in}{8.416822in}}%
\pgfpathlineto{\pgfqpoint{7.401457in}{8.430605in}}%
\pgfpathlineto{\pgfqpoint{7.386114in}{8.416011in}}%
\pgfpathlineto{\pgfqpoint{7.370603in}{8.413579in}}%
\pgfpathlineto{\pgfqpoint{7.354917in}{8.420876in}}%
\pgfpathlineto{\pgfqpoint{7.339055in}{8.404660in}}%
\pgfpathlineto{\pgfqpoint{7.323011in}{8.417633in}}%
\pgfpathlineto{\pgfqpoint{7.306782in}{8.400606in}}%
\pgfpathlineto{\pgfqpoint{7.290363in}{8.403038in}}%
\pgfpathlineto{\pgfqpoint{7.273750in}{8.401417in}}%
\pgfpathlineto{\pgfqpoint{7.256938in}{8.411146in}}%
\pgfpathlineto{\pgfqpoint{7.239922in}{8.409525in}}%
\pgfpathlineto{\pgfqpoint{7.222697in}{8.414389in}}%
\pgfpathlineto{\pgfqpoint{7.205258in}{8.406281in}}%
\pgfpathlineto{\pgfqpoint{7.187600in}{8.416822in}}%
\pgfpathlineto{\pgfqpoint{7.169717in}{8.342229in}}%
\pgfpathlineto{\pgfqpoint{7.151603in}{8.398984in}}%
\pgfpathlineto{\pgfqpoint{7.133253in}{8.399795in}}%
\pgfpathlineto{\pgfqpoint{7.114659in}{8.403038in}}%
\pgfpathlineto{\pgfqpoint{7.095816in}{8.393309in}}%
\pgfpathlineto{\pgfqpoint{7.076716in}{8.377904in}}%
\pgfpathlineto{\pgfqpoint{7.057353in}{8.380336in}}%
\pgfpathlineto{\pgfqpoint{7.037719in}{8.411146in}}%
\pgfpathlineto{\pgfqpoint{7.017806in}{8.392498in}}%
\pgfpathlineto{\pgfqpoint{6.997607in}{8.397363in}}%
\pgfpathlineto{\pgfqpoint{6.977113in}{8.398984in}}%
\pgfpathlineto{\pgfqpoint{6.956316in}{8.403038in}}%
\pgfpathlineto{\pgfqpoint{6.935206in}{8.403849in}}%
\pgfpathlineto{\pgfqpoint{6.913773in}{8.397363in}}%
\pgfpathlineto{\pgfqpoint{6.892008in}{8.391687in}}%
\pgfpathlineto{\pgfqpoint{6.869901in}{8.399795in}}%
\pgfpathlineto{\pgfqpoint{6.847439in}{8.390876in}}%
\pgfpathlineto{\pgfqpoint{6.824613in}{8.390876in}}%
\pgfpathlineto{\pgfqpoint{6.801409in}{8.401417in}}%
\pgfpathlineto{\pgfqpoint{6.777815in}{8.389255in}}%
\pgfpathlineto{\pgfqpoint{6.753818in}{8.405471in}}%
\pgfpathlineto{\pgfqpoint{6.729403in}{8.394120in}}%
\pgfpathlineto{\pgfqpoint{6.704556in}{8.376282in}}%
\pgfpathlineto{\pgfqpoint{6.679262in}{8.377904in}}%
\pgfpathlineto{\pgfqpoint{6.653503in}{8.390066in}}%
\pgfpathlineto{\pgfqpoint{6.627263in}{8.381147in}}%
\pgfpathlineto{\pgfqpoint{6.600523in}{8.385201in}}%
\pgfpathlineto{\pgfqpoint{6.573264in}{8.380336in}}%
\pgfpathlineto{\pgfqpoint{6.545465in}{8.394930in}}%
\pgfpathlineto{\pgfqpoint{6.517104in}{8.380336in}}%
\pgfpathlineto{\pgfqpoint{6.488159in}{8.388444in}}%
\pgfpathlineto{\pgfqpoint{6.458604in}{8.376282in}}%
\pgfpathlineto{\pgfqpoint{6.428414in}{8.389255in}}%
\pgfpathlineto{\pgfqpoint{6.397560in}{8.370607in}}%
\pgfpathlineto{\pgfqpoint{6.366012in}{8.381147in}}%
\pgfpathlineto{\pgfqpoint{6.333739in}{8.360066in}}%
\pgfpathlineto{\pgfqpoint{6.300707in}{8.376282in}}%
\pgfpathlineto{\pgfqpoint{6.266879in}{8.377904in}}%
\pgfpathlineto{\pgfqpoint{6.232215in}{8.353580in}}%
\pgfpathlineto{\pgfqpoint{6.196674in}{8.354391in}}%
\pgfpathlineto{\pgfqpoint{6.160209in}{8.380336in}}%
\pgfpathlineto{\pgfqpoint{6.122772in}{8.373039in}}%
\pgfpathlineto{\pgfqpoint{6.084310in}{8.358445in}}%
\pgfpathlineto{\pgfqpoint{6.044763in}{8.361688in}}%
\pgfpathlineto{\pgfqpoint{6.004070in}{8.357634in}}%
\pgfpathlineto{\pgfqpoint{5.962163in}{8.341418in}}%
\pgfpathlineto{\pgfqpoint{5.918965in}{8.346283in}}%
\pgfpathlineto{\pgfqpoint{5.874396in}{8.353580in}}%
\pgfpathlineto{\pgfqpoint{5.828366in}{8.314662in}}%
\pgfpathlineto{\pgfqpoint{5.780775in}{8.347905in}}%
\pgfpathlineto{\pgfqpoint{5.731513in}{8.322770in}}%
\pgfpathlineto{\pgfqpoint{5.680460in}{8.329256in}}%
\pgfpathlineto{\pgfqpoint{5.627480in}{8.321959in}}%
\pgfpathlineto{\pgfqpoint{5.572422in}{8.338986in}}%
\pgfpathlineto{\pgfqpoint{5.515116in}{8.328446in}}%
\pgfpathlineto{\pgfqpoint{5.455371in}{8.330878in}}%
\pgfpathlineto{\pgfqpoint{5.392969in}{8.314662in}}%
\pgfpathlineto{\pgfqpoint{5.327664in}{8.306554in}}%
\pgfpathlineto{\pgfqpoint{5.259172in}{8.315473in}}%
\pgfpathlineto{\pgfqpoint{5.187166in}{8.297636in}}%
\pgfpathlineto{\pgfqpoint{5.111267in}{8.282231in}}%
\pgfpathlineto{\pgfqpoint{5.031027in}{8.278987in}}%
\pgfpathlineto{\pgfqpoint{4.945922in}{8.272501in}}%
\pgfpathlineto{\pgfqpoint{4.855323in}{8.290339in}}%
\pgfpathlineto{\pgfqpoint{4.758470in}{8.255475in}}%
\pgfpathlineto{\pgfqpoint{4.654437in}{8.264393in}}%
\pgfpathlineto{\pgfqpoint{4.542073in}{8.262772in}}%
\pgfpathlineto{\pgfqpoint{4.419926in}{8.243313in}}%
\pgfpathlineto{\pgfqpoint{4.286129in}{8.254664in}}%
\pgfpathlineto{\pgfqpoint{4.138224in}{8.221421in}}%
\pgfpathlineto{\pgfqpoint{3.972879in}{8.201962in}}%
\pgfpathlineto{\pgfqpoint{3.785427in}{8.183314in}}%
\pgfpathlineto{\pgfqpoint{3.569030in}{8.193855in}}%
\pgfpathlineto{\pgfqpoint{3.313086in}{8.138721in}}%
\pgfpathlineto{\pgfqpoint{2.999836in}{8.079533in}}%
\pgfpathlineto{\pgfqpoint{2.595987in}{8.073858in}}%
\pgfpathlineto{\pgfqpoint{2.026793in}{7.877646in}}%
\pgfpathlineto{\pgfqpoint{1.053750in}{7.704137in}}%
\pgfpathlineto{\pgfqpoint{-3231.325327in}{1.870504in}}%
\pgfpathclose%
\pgfusepath{fill}%
\end{pgfscope}%
\begin{pgfscope}%
\pgfpathrectangle{\pgfqpoint{1.053750in}{5.716891in}}{\pgfqpoint{6.533250in}{3.078326in}}%
\pgfusepath{clip}%
\pgfsetbuttcap%
\pgfsetroundjoin%
\definecolor{currentfill}{rgb}{1.000000,0.498039,0.054902}%
\pgfsetfillcolor{currentfill}%
\pgfsetfillopacity{0.300000}%
\pgfsetlinewidth{0.000000pt}%
\definecolor{currentstroke}{rgb}{0.000000,0.000000,0.000000}%
\pgfsetstrokecolor{currentstroke}%
\pgfsetdash{}{0pt}%
\pgfpathmoveto{\pgfqpoint{-3231.325327in}{1.870504in}}%
\pgfpathlineto{\pgfqpoint{-3231.325327in}{1.870504in}}%
\pgfpathlineto{\pgfqpoint{1.053750in}{7.656301in}}%
\pgfpathlineto{\pgfqpoint{2.026793in}{7.884133in}}%
\pgfpathlineto{\pgfqpoint{2.595987in}{8.054399in}}%
\pgfpathlineto{\pgfqpoint{2.999836in}{8.017102in}}%
\pgfpathlineto{\pgfqpoint{3.313086in}{8.120073in}}%
\pgfpathlineto{\pgfqpoint{3.569030in}{8.194665in}}%
\pgfpathlineto{\pgfqpoint{3.785427in}{8.199530in}}%
\pgfpathlineto{\pgfqpoint{3.972879in}{8.182503in}}%
\pgfpathlineto{\pgfqpoint{4.138224in}{8.222232in}}%
\pgfpathlineto{\pgfqpoint{4.286129in}{8.153315in}}%
\pgfpathlineto{\pgfqpoint{4.419926in}{8.254664in}}%
\pgfpathlineto{\pgfqpoint{4.542073in}{8.270880in}}%
\pgfpathlineto{\pgfqpoint{4.654437in}{8.266015in}}%
\pgfpathlineto{\pgfqpoint{4.758470in}{8.175206in}}%
\pgfpathlineto{\pgfqpoint{4.855323in}{8.322770in}}%
\pgfpathlineto{\pgfqpoint{4.945922in}{8.257907in}}%
\pgfpathlineto{\pgfqpoint{5.031027in}{8.314662in}}%
\pgfpathlineto{\pgfqpoint{5.111267in}{8.248177in}}%
\pgfpathlineto{\pgfqpoint{5.187166in}{8.346283in}}%
\pgfpathlineto{\pgfqpoint{5.259172in}{8.315473in}}%
\pgfpathlineto{\pgfqpoint{5.327664in}{8.342229in}}%
\pgfpathlineto{\pgfqpoint{5.392969in}{8.341418in}}%
\pgfpathlineto{\pgfqpoint{5.455371in}{8.321149in}}%
\pgfpathlineto{\pgfqpoint{5.515116in}{8.386823in}}%
\pgfpathlineto{\pgfqpoint{5.572422in}{8.380336in}}%
\pgfpathlineto{\pgfqpoint{5.627480in}{8.380336in}}%
\pgfpathlineto{\pgfqpoint{5.680460in}{8.255475in}}%
\pgfpathlineto{\pgfqpoint{5.731513in}{8.362499in}}%
\pgfpathlineto{\pgfqpoint{5.780775in}{8.373850in}}%
\pgfpathlineto{\pgfqpoint{5.828366in}{8.103046in}}%
\pgfpathlineto{\pgfqpoint{5.874396in}{8.417633in}}%
\pgfpathlineto{\pgfqpoint{5.918965in}{8.368174in}}%
\pgfpathlineto{\pgfqpoint{5.962163in}{8.396552in}}%
\pgfpathlineto{\pgfqpoint{6.004070in}{8.392498in}}%
\pgfpathlineto{\pgfqpoint{6.044763in}{8.438713in}}%
\pgfpathlineto{\pgfqpoint{6.084310in}{8.393309in}}%
\pgfpathlineto{\pgfqpoint{6.122772in}{8.403038in}}%
\pgfpathlineto{\pgfqpoint{6.160209in}{8.398984in}}%
\pgfpathlineto{\pgfqpoint{6.196674in}{8.411957in}}%
\pgfpathlineto{\pgfqpoint{6.232215in}{8.390876in}}%
\pgfpathlineto{\pgfqpoint{6.266879in}{8.419254in}}%
\pgfpathlineto{\pgfqpoint{6.300707in}{8.424119in}}%
\pgfpathlineto{\pgfqpoint{6.333739in}{8.434659in}}%
\pgfpathlineto{\pgfqpoint{6.366012in}{8.373850in}}%
\pgfpathlineto{\pgfqpoint{6.397560in}{8.433038in}}%
\pgfpathlineto{\pgfqpoint{6.428414in}{8.399795in}}%
\pgfpathlineto{\pgfqpoint{6.458604in}{8.433038in}}%
\pgfpathlineto{\pgfqpoint{6.488159in}{8.433038in}}%
\pgfpathlineto{\pgfqpoint{6.517104in}{8.364120in}}%
\pgfpathlineto{\pgfqpoint{6.545465in}{8.454929in}}%
\pgfpathlineto{\pgfqpoint{6.573264in}{8.416011in}}%
\pgfpathlineto{\pgfqpoint{6.600523in}{8.334121in}}%
\pgfpathlineto{\pgfqpoint{6.627263in}{8.441145in}}%
\pgfpathlineto{\pgfqpoint{6.653503in}{8.454118in}}%
\pgfpathlineto{\pgfqpoint{6.679262in}{8.443578in}}%
\pgfpathlineto{\pgfqpoint{6.704556in}{8.425740in}}%
\pgfpathlineto{\pgfqpoint{6.729403in}{8.435470in}}%
\pgfpathlineto{\pgfqpoint{6.753818in}{8.437091in}}%
\pgfpathlineto{\pgfqpoint{6.777815in}{8.436281in}}%
\pgfpathlineto{\pgfqpoint{6.801409in}{8.446010in}}%
\pgfpathlineto{\pgfqpoint{6.824613in}{8.439524in}}%
\pgfpathlineto{\pgfqpoint{6.847439in}{8.426551in}}%
\pgfpathlineto{\pgfqpoint{6.869901in}{8.456550in}}%
\pgfpathlineto{\pgfqpoint{6.892008in}{8.443578in}}%
\pgfpathlineto{\pgfqpoint{6.913773in}{8.429794in}}%
\pgfpathlineto{\pgfqpoint{6.935206in}{8.451686in}}%
\pgfpathlineto{\pgfqpoint{6.956316in}{8.438713in}}%
\pgfpathlineto{\pgfqpoint{6.977113in}{8.428173in}}%
\pgfpathlineto{\pgfqpoint{6.997607in}{8.444389in}}%
\pgfpathlineto{\pgfqpoint{7.017806in}{8.411957in}}%
\pgfpathlineto{\pgfqpoint{7.037719in}{8.461415in}}%
\pgfpathlineto{\pgfqpoint{7.057353in}{8.451686in}}%
\pgfpathlineto{\pgfqpoint{7.076716in}{8.433848in}}%
\pgfpathlineto{\pgfqpoint{7.095816in}{8.436281in}}%
\pgfpathlineto{\pgfqpoint{7.114659in}{8.424930in}}%
\pgfpathlineto{\pgfqpoint{7.133253in}{8.441145in}}%
\pgfpathlineto{\pgfqpoint{7.151603in}{8.414389in}}%
\pgfpathlineto{\pgfqpoint{7.169717in}{8.359256in}}%
\pgfpathlineto{\pgfqpoint{7.187600in}{8.463037in}}%
\pgfpathlineto{\pgfqpoint{7.205258in}{8.416011in}}%
\pgfpathlineto{\pgfqpoint{7.222697in}{8.458983in}}%
\pgfpathlineto{\pgfqpoint{7.239922in}{8.420876in}}%
\pgfpathlineto{\pgfqpoint{7.256938in}{8.462226in}}%
\pgfpathlineto{\pgfqpoint{7.273750in}{8.463848in}}%
\pgfpathlineto{\pgfqpoint{7.290363in}{8.441956in}}%
\pgfpathlineto{\pgfqpoint{7.306782in}{8.465469in}}%
\pgfpathlineto{\pgfqpoint{7.323011in}{8.463848in}}%
\pgfpathlineto{\pgfqpoint{7.339055in}{8.450064in}}%
\pgfpathlineto{\pgfqpoint{7.354917in}{8.447632in}}%
\pgfpathlineto{\pgfqpoint{7.370603in}{8.458983in}}%
\pgfpathlineto{\pgfqpoint{7.386114in}{8.463037in}}%
\pgfpathlineto{\pgfqpoint{7.401457in}{8.460604in}}%
\pgfpathlineto{\pgfqpoint{7.416633in}{8.451686in}}%
\pgfpathlineto{\pgfqpoint{7.431647in}{8.439524in}}%
\pgfpathlineto{\pgfqpoint{7.446502in}{8.457361in}}%
\pgfpathlineto{\pgfqpoint{7.461202in}{8.454118in}}%
\pgfpathlineto{\pgfqpoint{7.475749in}{8.430605in}}%
\pgfpathlineto{\pgfqpoint{7.490148in}{8.444389in}}%
\pgfpathlineto{\pgfqpoint{7.504399in}{8.458172in}}%
\pgfpathlineto{\pgfqpoint{7.518508in}{8.462226in}}%
\pgfpathlineto{\pgfqpoint{7.518508in}{8.462226in}}%
\pgfpathlineto{\pgfqpoint{7.518508in}{8.462226in}}%
\pgfpathlineto{\pgfqpoint{7.504399in}{8.458172in}}%
\pgfpathlineto{\pgfqpoint{7.490148in}{8.444389in}}%
\pgfpathlineto{\pgfqpoint{7.475749in}{8.430605in}}%
\pgfpathlineto{\pgfqpoint{7.461202in}{8.454118in}}%
\pgfpathlineto{\pgfqpoint{7.446502in}{8.457361in}}%
\pgfpathlineto{\pgfqpoint{7.431647in}{8.439524in}}%
\pgfpathlineto{\pgfqpoint{7.416633in}{8.451686in}}%
\pgfpathlineto{\pgfqpoint{7.401457in}{8.460604in}}%
\pgfpathlineto{\pgfqpoint{7.386114in}{8.463037in}}%
\pgfpathlineto{\pgfqpoint{7.370603in}{8.458983in}}%
\pgfpathlineto{\pgfqpoint{7.354917in}{8.447632in}}%
\pgfpathlineto{\pgfqpoint{7.339055in}{8.450064in}}%
\pgfpathlineto{\pgfqpoint{7.323011in}{8.463848in}}%
\pgfpathlineto{\pgfqpoint{7.306782in}{8.465469in}}%
\pgfpathlineto{\pgfqpoint{7.290363in}{8.441956in}}%
\pgfpathlineto{\pgfqpoint{7.273750in}{8.463848in}}%
\pgfpathlineto{\pgfqpoint{7.256938in}{8.462226in}}%
\pgfpathlineto{\pgfqpoint{7.239922in}{8.420876in}}%
\pgfpathlineto{\pgfqpoint{7.222697in}{8.458983in}}%
\pgfpathlineto{\pgfqpoint{7.205258in}{8.416011in}}%
\pgfpathlineto{\pgfqpoint{7.187600in}{8.463037in}}%
\pgfpathlineto{\pgfqpoint{7.169717in}{8.359256in}}%
\pgfpathlineto{\pgfqpoint{7.151603in}{8.414389in}}%
\pgfpathlineto{\pgfqpoint{7.133253in}{8.441145in}}%
\pgfpathlineto{\pgfqpoint{7.114659in}{8.424930in}}%
\pgfpathlineto{\pgfqpoint{7.095816in}{8.436281in}}%
\pgfpathlineto{\pgfqpoint{7.076716in}{8.433848in}}%
\pgfpathlineto{\pgfqpoint{7.057353in}{8.451686in}}%
\pgfpathlineto{\pgfqpoint{7.037719in}{8.461415in}}%
\pgfpathlineto{\pgfqpoint{7.017806in}{8.411957in}}%
\pgfpathlineto{\pgfqpoint{6.997607in}{8.444389in}}%
\pgfpathlineto{\pgfqpoint{6.977113in}{8.428173in}}%
\pgfpathlineto{\pgfqpoint{6.956316in}{8.438713in}}%
\pgfpathlineto{\pgfqpoint{6.935206in}{8.451686in}}%
\pgfpathlineto{\pgfqpoint{6.913773in}{8.429794in}}%
\pgfpathlineto{\pgfqpoint{6.892008in}{8.443578in}}%
\pgfpathlineto{\pgfqpoint{6.869901in}{8.456550in}}%
\pgfpathlineto{\pgfqpoint{6.847439in}{8.426551in}}%
\pgfpathlineto{\pgfqpoint{6.824613in}{8.439524in}}%
\pgfpathlineto{\pgfqpoint{6.801409in}{8.446010in}}%
\pgfpathlineto{\pgfqpoint{6.777815in}{8.436281in}}%
\pgfpathlineto{\pgfqpoint{6.753818in}{8.437091in}}%
\pgfpathlineto{\pgfqpoint{6.729403in}{8.435470in}}%
\pgfpathlineto{\pgfqpoint{6.704556in}{8.425740in}}%
\pgfpathlineto{\pgfqpoint{6.679262in}{8.443578in}}%
\pgfpathlineto{\pgfqpoint{6.653503in}{8.454118in}}%
\pgfpathlineto{\pgfqpoint{6.627263in}{8.441145in}}%
\pgfpathlineto{\pgfqpoint{6.600523in}{8.334121in}}%
\pgfpathlineto{\pgfqpoint{6.573264in}{8.416011in}}%
\pgfpathlineto{\pgfqpoint{6.545465in}{8.454929in}}%
\pgfpathlineto{\pgfqpoint{6.517104in}{8.364120in}}%
\pgfpathlineto{\pgfqpoint{6.488159in}{8.433038in}}%
\pgfpathlineto{\pgfqpoint{6.458604in}{8.433038in}}%
\pgfpathlineto{\pgfqpoint{6.428414in}{8.399795in}}%
\pgfpathlineto{\pgfqpoint{6.397560in}{8.433038in}}%
\pgfpathlineto{\pgfqpoint{6.366012in}{8.373850in}}%
\pgfpathlineto{\pgfqpoint{6.333739in}{8.434659in}}%
\pgfpathlineto{\pgfqpoint{6.300707in}{8.424119in}}%
\pgfpathlineto{\pgfqpoint{6.266879in}{8.419254in}}%
\pgfpathlineto{\pgfqpoint{6.232215in}{8.390876in}}%
\pgfpathlineto{\pgfqpoint{6.196674in}{8.411957in}}%
\pgfpathlineto{\pgfqpoint{6.160209in}{8.398984in}}%
\pgfpathlineto{\pgfqpoint{6.122772in}{8.403038in}}%
\pgfpathlineto{\pgfqpoint{6.084310in}{8.393309in}}%
\pgfpathlineto{\pgfqpoint{6.044763in}{8.438713in}}%
\pgfpathlineto{\pgfqpoint{6.004070in}{8.392498in}}%
\pgfpathlineto{\pgfqpoint{5.962163in}{8.396552in}}%
\pgfpathlineto{\pgfqpoint{5.918965in}{8.368174in}}%
\pgfpathlineto{\pgfqpoint{5.874396in}{8.417633in}}%
\pgfpathlineto{\pgfqpoint{5.828366in}{8.103046in}}%
\pgfpathlineto{\pgfqpoint{5.780775in}{8.373850in}}%
\pgfpathlineto{\pgfqpoint{5.731513in}{8.362499in}}%
\pgfpathlineto{\pgfqpoint{5.680460in}{8.255475in}}%
\pgfpathlineto{\pgfqpoint{5.627480in}{8.380336in}}%
\pgfpathlineto{\pgfqpoint{5.572422in}{8.380336in}}%
\pgfpathlineto{\pgfqpoint{5.515116in}{8.386823in}}%
\pgfpathlineto{\pgfqpoint{5.455371in}{8.321149in}}%
\pgfpathlineto{\pgfqpoint{5.392969in}{8.341418in}}%
\pgfpathlineto{\pgfqpoint{5.327664in}{8.342229in}}%
\pgfpathlineto{\pgfqpoint{5.259172in}{8.315473in}}%
\pgfpathlineto{\pgfqpoint{5.187166in}{8.346283in}}%
\pgfpathlineto{\pgfqpoint{5.111267in}{8.248177in}}%
\pgfpathlineto{\pgfqpoint{5.031027in}{8.314662in}}%
\pgfpathlineto{\pgfqpoint{4.945922in}{8.257907in}}%
\pgfpathlineto{\pgfqpoint{4.855323in}{8.322770in}}%
\pgfpathlineto{\pgfqpoint{4.758470in}{8.175206in}}%
\pgfpathlineto{\pgfqpoint{4.654437in}{8.266015in}}%
\pgfpathlineto{\pgfqpoint{4.542073in}{8.270880in}}%
\pgfpathlineto{\pgfqpoint{4.419926in}{8.254664in}}%
\pgfpathlineto{\pgfqpoint{4.286129in}{8.153315in}}%
\pgfpathlineto{\pgfqpoint{4.138224in}{8.222232in}}%
\pgfpathlineto{\pgfqpoint{3.972879in}{8.182503in}}%
\pgfpathlineto{\pgfqpoint{3.785427in}{8.199530in}}%
\pgfpathlineto{\pgfqpoint{3.569030in}{8.194665in}}%
\pgfpathlineto{\pgfqpoint{3.313086in}{8.120073in}}%
\pgfpathlineto{\pgfqpoint{2.999836in}{8.017102in}}%
\pgfpathlineto{\pgfqpoint{2.595987in}{8.054399in}}%
\pgfpathlineto{\pgfqpoint{2.026793in}{7.884133in}}%
\pgfpathlineto{\pgfqpoint{1.053750in}{7.656301in}}%
\pgfpathlineto{\pgfqpoint{-3231.325327in}{1.870504in}}%
\pgfpathclose%
\pgfusepath{fill}%
\end{pgfscope}%
\begin{pgfscope}%
\pgfsetbuttcap%
\pgfsetroundjoin%
\definecolor{currentfill}{rgb}{0.000000,0.000000,0.000000}%
\pgfsetfillcolor{currentfill}%
\pgfsetlinewidth{0.803000pt}%
\definecolor{currentstroke}{rgb}{0.000000,0.000000,0.000000}%
\pgfsetstrokecolor{currentstroke}%
\pgfsetdash{}{0pt}%
\pgfsys@defobject{currentmarker}{\pgfqpoint{0.000000in}{-0.048611in}}{\pgfqpoint{0.000000in}{0.000000in}}{%
\pgfpathmoveto{\pgfqpoint{0.000000in}{0.000000in}}%
\pgfpathlineto{\pgfqpoint{0.000000in}{-0.048611in}}%
\pgfusepath{stroke,fill}%
}%
\begin{pgfscope}%
\pgfsys@transformshift{1.053750in}{5.716891in}%
\pgfsys@useobject{currentmarker}{}%
\end{pgfscope}%
\end{pgfscope}%
\begin{pgfscope}%
\pgfsetbuttcap%
\pgfsetroundjoin%
\definecolor{currentfill}{rgb}{0.000000,0.000000,0.000000}%
\pgfsetfillcolor{currentfill}%
\pgfsetlinewidth{0.803000pt}%
\definecolor{currentstroke}{rgb}{0.000000,0.000000,0.000000}%
\pgfsetstrokecolor{currentstroke}%
\pgfsetdash{}{0pt}%
\pgfsys@defobject{currentmarker}{\pgfqpoint{0.000000in}{-0.048611in}}{\pgfqpoint{0.000000in}{0.000000in}}{%
\pgfpathmoveto{\pgfqpoint{0.000000in}{0.000000in}}%
\pgfpathlineto{\pgfqpoint{0.000000in}{-0.048611in}}%
\pgfusepath{stroke,fill}%
}%
\begin{pgfscope}%
\pgfsys@transformshift{4.286129in}{5.716891in}%
\pgfsys@useobject{currentmarker}{}%
\end{pgfscope}%
\end{pgfscope}%
\begin{pgfscope}%
\pgfsetbuttcap%
\pgfsetroundjoin%
\definecolor{currentfill}{rgb}{0.000000,0.000000,0.000000}%
\pgfsetfillcolor{currentfill}%
\pgfsetlinewidth{0.803000pt}%
\definecolor{currentstroke}{rgb}{0.000000,0.000000,0.000000}%
\pgfsetstrokecolor{currentstroke}%
\pgfsetdash{}{0pt}%
\pgfsys@defobject{currentmarker}{\pgfqpoint{0.000000in}{-0.048611in}}{\pgfqpoint{0.000000in}{0.000000in}}{%
\pgfpathmoveto{\pgfqpoint{0.000000in}{0.000000in}}%
\pgfpathlineto{\pgfqpoint{0.000000in}{-0.048611in}}%
\pgfusepath{stroke,fill}%
}%
\begin{pgfscope}%
\pgfsys@transformshift{7.518508in}{5.716891in}%
\pgfsys@useobject{currentmarker}{}%
\end{pgfscope}%
\end{pgfscope}%
\begin{pgfscope}%
\pgfsetbuttcap%
\pgfsetroundjoin%
\definecolor{currentfill}{rgb}{0.000000,0.000000,0.000000}%
\pgfsetfillcolor{currentfill}%
\pgfsetlinewidth{0.602250pt}%
\definecolor{currentstroke}{rgb}{0.000000,0.000000,0.000000}%
\pgfsetstrokecolor{currentstroke}%
\pgfsetdash{}{0pt}%
\pgfsys@defobject{currentmarker}{\pgfqpoint{0.000000in}{-0.027778in}}{\pgfqpoint{0.000000in}{0.000000in}}{%
\pgfpathmoveto{\pgfqpoint{0.000000in}{0.000000in}}%
\pgfpathlineto{\pgfqpoint{0.000000in}{-0.027778in}}%
\pgfusepath{stroke,fill}%
}%
\begin{pgfscope}%
\pgfsys@transformshift{2.026793in}{5.716891in}%
\pgfsys@useobject{currentmarker}{}%
\end{pgfscope}%
\end{pgfscope}%
\begin{pgfscope}%
\pgfsetbuttcap%
\pgfsetroundjoin%
\definecolor{currentfill}{rgb}{0.000000,0.000000,0.000000}%
\pgfsetfillcolor{currentfill}%
\pgfsetlinewidth{0.602250pt}%
\definecolor{currentstroke}{rgb}{0.000000,0.000000,0.000000}%
\pgfsetstrokecolor{currentstroke}%
\pgfsetdash{}{0pt}%
\pgfsys@defobject{currentmarker}{\pgfqpoint{0.000000in}{-0.027778in}}{\pgfqpoint{0.000000in}{0.000000in}}{%
\pgfpathmoveto{\pgfqpoint{0.000000in}{0.000000in}}%
\pgfpathlineto{\pgfqpoint{0.000000in}{-0.027778in}}%
\pgfusepath{stroke,fill}%
}%
\begin{pgfscope}%
\pgfsys@transformshift{2.595987in}{5.716891in}%
\pgfsys@useobject{currentmarker}{}%
\end{pgfscope}%
\end{pgfscope}%
\begin{pgfscope}%
\pgfsetbuttcap%
\pgfsetroundjoin%
\definecolor{currentfill}{rgb}{0.000000,0.000000,0.000000}%
\pgfsetfillcolor{currentfill}%
\pgfsetlinewidth{0.602250pt}%
\definecolor{currentstroke}{rgb}{0.000000,0.000000,0.000000}%
\pgfsetstrokecolor{currentstroke}%
\pgfsetdash{}{0pt}%
\pgfsys@defobject{currentmarker}{\pgfqpoint{0.000000in}{-0.027778in}}{\pgfqpoint{0.000000in}{0.000000in}}{%
\pgfpathmoveto{\pgfqpoint{0.000000in}{0.000000in}}%
\pgfpathlineto{\pgfqpoint{0.000000in}{-0.027778in}}%
\pgfusepath{stroke,fill}%
}%
\begin{pgfscope}%
\pgfsys@transformshift{2.999836in}{5.716891in}%
\pgfsys@useobject{currentmarker}{}%
\end{pgfscope}%
\end{pgfscope}%
\begin{pgfscope}%
\pgfsetbuttcap%
\pgfsetroundjoin%
\definecolor{currentfill}{rgb}{0.000000,0.000000,0.000000}%
\pgfsetfillcolor{currentfill}%
\pgfsetlinewidth{0.602250pt}%
\definecolor{currentstroke}{rgb}{0.000000,0.000000,0.000000}%
\pgfsetstrokecolor{currentstroke}%
\pgfsetdash{}{0pt}%
\pgfsys@defobject{currentmarker}{\pgfqpoint{0.000000in}{-0.027778in}}{\pgfqpoint{0.000000in}{0.000000in}}{%
\pgfpathmoveto{\pgfqpoint{0.000000in}{0.000000in}}%
\pgfpathlineto{\pgfqpoint{0.000000in}{-0.027778in}}%
\pgfusepath{stroke,fill}%
}%
\begin{pgfscope}%
\pgfsys@transformshift{3.313086in}{5.716891in}%
\pgfsys@useobject{currentmarker}{}%
\end{pgfscope}%
\end{pgfscope}%
\begin{pgfscope}%
\pgfsetbuttcap%
\pgfsetroundjoin%
\definecolor{currentfill}{rgb}{0.000000,0.000000,0.000000}%
\pgfsetfillcolor{currentfill}%
\pgfsetlinewidth{0.602250pt}%
\definecolor{currentstroke}{rgb}{0.000000,0.000000,0.000000}%
\pgfsetstrokecolor{currentstroke}%
\pgfsetdash{}{0pt}%
\pgfsys@defobject{currentmarker}{\pgfqpoint{0.000000in}{-0.027778in}}{\pgfqpoint{0.000000in}{0.000000in}}{%
\pgfpathmoveto{\pgfqpoint{0.000000in}{0.000000in}}%
\pgfpathlineto{\pgfqpoint{0.000000in}{-0.027778in}}%
\pgfusepath{stroke,fill}%
}%
\begin{pgfscope}%
\pgfsys@transformshift{3.569030in}{5.716891in}%
\pgfsys@useobject{currentmarker}{}%
\end{pgfscope}%
\end{pgfscope}%
\begin{pgfscope}%
\pgfsetbuttcap%
\pgfsetroundjoin%
\definecolor{currentfill}{rgb}{0.000000,0.000000,0.000000}%
\pgfsetfillcolor{currentfill}%
\pgfsetlinewidth{0.602250pt}%
\definecolor{currentstroke}{rgb}{0.000000,0.000000,0.000000}%
\pgfsetstrokecolor{currentstroke}%
\pgfsetdash{}{0pt}%
\pgfsys@defobject{currentmarker}{\pgfqpoint{0.000000in}{-0.027778in}}{\pgfqpoint{0.000000in}{0.000000in}}{%
\pgfpathmoveto{\pgfqpoint{0.000000in}{0.000000in}}%
\pgfpathlineto{\pgfqpoint{0.000000in}{-0.027778in}}%
\pgfusepath{stroke,fill}%
}%
\begin{pgfscope}%
\pgfsys@transformshift{3.785427in}{5.716891in}%
\pgfsys@useobject{currentmarker}{}%
\end{pgfscope}%
\end{pgfscope}%
\begin{pgfscope}%
\pgfsetbuttcap%
\pgfsetroundjoin%
\definecolor{currentfill}{rgb}{0.000000,0.000000,0.000000}%
\pgfsetfillcolor{currentfill}%
\pgfsetlinewidth{0.602250pt}%
\definecolor{currentstroke}{rgb}{0.000000,0.000000,0.000000}%
\pgfsetstrokecolor{currentstroke}%
\pgfsetdash{}{0pt}%
\pgfsys@defobject{currentmarker}{\pgfqpoint{0.000000in}{-0.027778in}}{\pgfqpoint{0.000000in}{0.000000in}}{%
\pgfpathmoveto{\pgfqpoint{0.000000in}{0.000000in}}%
\pgfpathlineto{\pgfqpoint{0.000000in}{-0.027778in}}%
\pgfusepath{stroke,fill}%
}%
\begin{pgfscope}%
\pgfsys@transformshift{3.972879in}{5.716891in}%
\pgfsys@useobject{currentmarker}{}%
\end{pgfscope}%
\end{pgfscope}%
\begin{pgfscope}%
\pgfsetbuttcap%
\pgfsetroundjoin%
\definecolor{currentfill}{rgb}{0.000000,0.000000,0.000000}%
\pgfsetfillcolor{currentfill}%
\pgfsetlinewidth{0.602250pt}%
\definecolor{currentstroke}{rgb}{0.000000,0.000000,0.000000}%
\pgfsetstrokecolor{currentstroke}%
\pgfsetdash{}{0pt}%
\pgfsys@defobject{currentmarker}{\pgfqpoint{0.000000in}{-0.027778in}}{\pgfqpoint{0.000000in}{0.000000in}}{%
\pgfpathmoveto{\pgfqpoint{0.000000in}{0.000000in}}%
\pgfpathlineto{\pgfqpoint{0.000000in}{-0.027778in}}%
\pgfusepath{stroke,fill}%
}%
\begin{pgfscope}%
\pgfsys@transformshift{4.138224in}{5.716891in}%
\pgfsys@useobject{currentmarker}{}%
\end{pgfscope}%
\end{pgfscope}%
\begin{pgfscope}%
\pgfsetbuttcap%
\pgfsetroundjoin%
\definecolor{currentfill}{rgb}{0.000000,0.000000,0.000000}%
\pgfsetfillcolor{currentfill}%
\pgfsetlinewidth{0.602250pt}%
\definecolor{currentstroke}{rgb}{0.000000,0.000000,0.000000}%
\pgfsetstrokecolor{currentstroke}%
\pgfsetdash{}{0pt}%
\pgfsys@defobject{currentmarker}{\pgfqpoint{0.000000in}{-0.027778in}}{\pgfqpoint{0.000000in}{0.000000in}}{%
\pgfpathmoveto{\pgfqpoint{0.000000in}{0.000000in}}%
\pgfpathlineto{\pgfqpoint{0.000000in}{-0.027778in}}%
\pgfusepath{stroke,fill}%
}%
\begin{pgfscope}%
\pgfsys@transformshift{5.259172in}{5.716891in}%
\pgfsys@useobject{currentmarker}{}%
\end{pgfscope}%
\end{pgfscope}%
\begin{pgfscope}%
\pgfsetbuttcap%
\pgfsetroundjoin%
\definecolor{currentfill}{rgb}{0.000000,0.000000,0.000000}%
\pgfsetfillcolor{currentfill}%
\pgfsetlinewidth{0.602250pt}%
\definecolor{currentstroke}{rgb}{0.000000,0.000000,0.000000}%
\pgfsetstrokecolor{currentstroke}%
\pgfsetdash{}{0pt}%
\pgfsys@defobject{currentmarker}{\pgfqpoint{0.000000in}{-0.027778in}}{\pgfqpoint{0.000000in}{0.000000in}}{%
\pgfpathmoveto{\pgfqpoint{0.000000in}{0.000000in}}%
\pgfpathlineto{\pgfqpoint{0.000000in}{-0.027778in}}%
\pgfusepath{stroke,fill}%
}%
\begin{pgfscope}%
\pgfsys@transformshift{5.828366in}{5.716891in}%
\pgfsys@useobject{currentmarker}{}%
\end{pgfscope}%
\end{pgfscope}%
\begin{pgfscope}%
\pgfsetbuttcap%
\pgfsetroundjoin%
\definecolor{currentfill}{rgb}{0.000000,0.000000,0.000000}%
\pgfsetfillcolor{currentfill}%
\pgfsetlinewidth{0.602250pt}%
\definecolor{currentstroke}{rgb}{0.000000,0.000000,0.000000}%
\pgfsetstrokecolor{currentstroke}%
\pgfsetdash{}{0pt}%
\pgfsys@defobject{currentmarker}{\pgfqpoint{0.000000in}{-0.027778in}}{\pgfqpoint{0.000000in}{0.000000in}}{%
\pgfpathmoveto{\pgfqpoint{0.000000in}{0.000000in}}%
\pgfpathlineto{\pgfqpoint{0.000000in}{-0.027778in}}%
\pgfusepath{stroke,fill}%
}%
\begin{pgfscope}%
\pgfsys@transformshift{6.232215in}{5.716891in}%
\pgfsys@useobject{currentmarker}{}%
\end{pgfscope}%
\end{pgfscope}%
\begin{pgfscope}%
\pgfsetbuttcap%
\pgfsetroundjoin%
\definecolor{currentfill}{rgb}{0.000000,0.000000,0.000000}%
\pgfsetfillcolor{currentfill}%
\pgfsetlinewidth{0.602250pt}%
\definecolor{currentstroke}{rgb}{0.000000,0.000000,0.000000}%
\pgfsetstrokecolor{currentstroke}%
\pgfsetdash{}{0pt}%
\pgfsys@defobject{currentmarker}{\pgfqpoint{0.000000in}{-0.027778in}}{\pgfqpoint{0.000000in}{0.000000in}}{%
\pgfpathmoveto{\pgfqpoint{0.000000in}{0.000000in}}%
\pgfpathlineto{\pgfqpoint{0.000000in}{-0.027778in}}%
\pgfusepath{stroke,fill}%
}%
\begin{pgfscope}%
\pgfsys@transformshift{6.545465in}{5.716891in}%
\pgfsys@useobject{currentmarker}{}%
\end{pgfscope}%
\end{pgfscope}%
\begin{pgfscope}%
\pgfsetbuttcap%
\pgfsetroundjoin%
\definecolor{currentfill}{rgb}{0.000000,0.000000,0.000000}%
\pgfsetfillcolor{currentfill}%
\pgfsetlinewidth{0.602250pt}%
\definecolor{currentstroke}{rgb}{0.000000,0.000000,0.000000}%
\pgfsetstrokecolor{currentstroke}%
\pgfsetdash{}{0pt}%
\pgfsys@defobject{currentmarker}{\pgfqpoint{0.000000in}{-0.027778in}}{\pgfqpoint{0.000000in}{0.000000in}}{%
\pgfpathmoveto{\pgfqpoint{0.000000in}{0.000000in}}%
\pgfpathlineto{\pgfqpoint{0.000000in}{-0.027778in}}%
\pgfusepath{stroke,fill}%
}%
\begin{pgfscope}%
\pgfsys@transformshift{6.801409in}{5.716891in}%
\pgfsys@useobject{currentmarker}{}%
\end{pgfscope}%
\end{pgfscope}%
\begin{pgfscope}%
\pgfsetbuttcap%
\pgfsetroundjoin%
\definecolor{currentfill}{rgb}{0.000000,0.000000,0.000000}%
\pgfsetfillcolor{currentfill}%
\pgfsetlinewidth{0.602250pt}%
\definecolor{currentstroke}{rgb}{0.000000,0.000000,0.000000}%
\pgfsetstrokecolor{currentstroke}%
\pgfsetdash{}{0pt}%
\pgfsys@defobject{currentmarker}{\pgfqpoint{0.000000in}{-0.027778in}}{\pgfqpoint{0.000000in}{0.000000in}}{%
\pgfpathmoveto{\pgfqpoint{0.000000in}{0.000000in}}%
\pgfpathlineto{\pgfqpoint{0.000000in}{-0.027778in}}%
\pgfusepath{stroke,fill}%
}%
\begin{pgfscope}%
\pgfsys@transformshift{7.017806in}{5.716891in}%
\pgfsys@useobject{currentmarker}{}%
\end{pgfscope}%
\end{pgfscope}%
\begin{pgfscope}%
\pgfsetbuttcap%
\pgfsetroundjoin%
\definecolor{currentfill}{rgb}{0.000000,0.000000,0.000000}%
\pgfsetfillcolor{currentfill}%
\pgfsetlinewidth{0.602250pt}%
\definecolor{currentstroke}{rgb}{0.000000,0.000000,0.000000}%
\pgfsetstrokecolor{currentstroke}%
\pgfsetdash{}{0pt}%
\pgfsys@defobject{currentmarker}{\pgfqpoint{0.000000in}{-0.027778in}}{\pgfqpoint{0.000000in}{0.000000in}}{%
\pgfpathmoveto{\pgfqpoint{0.000000in}{0.000000in}}%
\pgfpathlineto{\pgfqpoint{0.000000in}{-0.027778in}}%
\pgfusepath{stroke,fill}%
}%
\begin{pgfscope}%
\pgfsys@transformshift{7.205258in}{5.716891in}%
\pgfsys@useobject{currentmarker}{}%
\end{pgfscope}%
\end{pgfscope}%
\begin{pgfscope}%
\pgfsetbuttcap%
\pgfsetroundjoin%
\definecolor{currentfill}{rgb}{0.000000,0.000000,0.000000}%
\pgfsetfillcolor{currentfill}%
\pgfsetlinewidth{0.602250pt}%
\definecolor{currentstroke}{rgb}{0.000000,0.000000,0.000000}%
\pgfsetstrokecolor{currentstroke}%
\pgfsetdash{}{0pt}%
\pgfsys@defobject{currentmarker}{\pgfqpoint{0.000000in}{-0.027778in}}{\pgfqpoint{0.000000in}{0.000000in}}{%
\pgfpathmoveto{\pgfqpoint{0.000000in}{0.000000in}}%
\pgfpathlineto{\pgfqpoint{0.000000in}{-0.027778in}}%
\pgfusepath{stroke,fill}%
}%
\begin{pgfscope}%
\pgfsys@transformshift{7.370603in}{5.716891in}%
\pgfsys@useobject{currentmarker}{}%
\end{pgfscope}%
\end{pgfscope}%
\begin{pgfscope}%
\pgfsetbuttcap%
\pgfsetroundjoin%
\definecolor{currentfill}{rgb}{0.000000,0.000000,0.000000}%
\pgfsetfillcolor{currentfill}%
\pgfsetlinewidth{0.803000pt}%
\definecolor{currentstroke}{rgb}{0.000000,0.000000,0.000000}%
\pgfsetstrokecolor{currentstroke}%
\pgfsetdash{}{0pt}%
\pgfsys@defobject{currentmarker}{\pgfqpoint{-0.048611in}{0.000000in}}{\pgfqpoint{0.000000in}{0.000000in}}{%
\pgfpathmoveto{\pgfqpoint{0.000000in}{0.000000in}}%
\pgfpathlineto{\pgfqpoint{-0.048611in}{0.000000in}}%
\pgfusepath{stroke,fill}%
}%
\begin{pgfscope}%
\pgfsys@transformshift{1.053750in}{5.933697in}%
\pgfsys@useobject{currentmarker}{}%
\end{pgfscope}%
\end{pgfscope}%
\begin{pgfscope}%
\definecolor{textcolor}{rgb}{0.000000,0.000000,0.000000}%
\pgfsetstrokecolor{textcolor}%
\pgfsetfillcolor{textcolor}%
\pgftext[x=0.585434in,y=5.870383in,left,base]{\color{textcolor}\sffamily\fontsize{12.000000}{14.400000}\selectfont 0.60}%
\end{pgfscope}%
\begin{pgfscope}%
\pgfsetbuttcap%
\pgfsetroundjoin%
\definecolor{currentfill}{rgb}{0.000000,0.000000,0.000000}%
\pgfsetfillcolor{currentfill}%
\pgfsetlinewidth{0.803000pt}%
\definecolor{currentstroke}{rgb}{0.000000,0.000000,0.000000}%
\pgfsetstrokecolor{currentstroke}%
\pgfsetdash{}{0pt}%
\pgfsys@defobject{currentmarker}{\pgfqpoint{-0.048611in}{0.000000in}}{\pgfqpoint{0.000000in}{0.000000in}}{%
\pgfpathmoveto{\pgfqpoint{0.000000in}{0.000000in}}%
\pgfpathlineto{\pgfqpoint{-0.048611in}{0.000000in}}%
\pgfusepath{stroke,fill}%
}%
\begin{pgfscope}%
\pgfsys@transformshift{1.053750in}{6.338443in}%
\pgfsys@useobject{currentmarker}{}%
\end{pgfscope}%
\end{pgfscope}%
\begin{pgfscope}%
\definecolor{textcolor}{rgb}{0.000000,0.000000,0.000000}%
\pgfsetstrokecolor{textcolor}%
\pgfsetfillcolor{textcolor}%
\pgftext[x=0.585434in,y=6.275129in,left,base]{\color{textcolor}\sffamily\fontsize{12.000000}{14.400000}\selectfont 0.65}%
\end{pgfscope}%
\begin{pgfscope}%
\pgfsetbuttcap%
\pgfsetroundjoin%
\definecolor{currentfill}{rgb}{0.000000,0.000000,0.000000}%
\pgfsetfillcolor{currentfill}%
\pgfsetlinewidth{0.803000pt}%
\definecolor{currentstroke}{rgb}{0.000000,0.000000,0.000000}%
\pgfsetstrokecolor{currentstroke}%
\pgfsetdash{}{0pt}%
\pgfsys@defobject{currentmarker}{\pgfqpoint{-0.048611in}{0.000000in}}{\pgfqpoint{0.000000in}{0.000000in}}{%
\pgfpathmoveto{\pgfqpoint{0.000000in}{0.000000in}}%
\pgfpathlineto{\pgfqpoint{-0.048611in}{0.000000in}}%
\pgfusepath{stroke,fill}%
}%
\begin{pgfscope}%
\pgfsys@transformshift{1.053750in}{6.743189in}%
\pgfsys@useobject{currentmarker}{}%
\end{pgfscope}%
\end{pgfscope}%
\begin{pgfscope}%
\definecolor{textcolor}{rgb}{0.000000,0.000000,0.000000}%
\pgfsetstrokecolor{textcolor}%
\pgfsetfillcolor{textcolor}%
\pgftext[x=0.585434in,y=6.679875in,left,base]{\color{textcolor}\sffamily\fontsize{12.000000}{14.400000}\selectfont 0.70}%
\end{pgfscope}%
\begin{pgfscope}%
\pgfsetbuttcap%
\pgfsetroundjoin%
\definecolor{currentfill}{rgb}{0.000000,0.000000,0.000000}%
\pgfsetfillcolor{currentfill}%
\pgfsetlinewidth{0.803000pt}%
\definecolor{currentstroke}{rgb}{0.000000,0.000000,0.000000}%
\pgfsetstrokecolor{currentstroke}%
\pgfsetdash{}{0pt}%
\pgfsys@defobject{currentmarker}{\pgfqpoint{-0.048611in}{0.000000in}}{\pgfqpoint{0.000000in}{0.000000in}}{%
\pgfpathmoveto{\pgfqpoint{0.000000in}{0.000000in}}%
\pgfpathlineto{\pgfqpoint{-0.048611in}{0.000000in}}%
\pgfusepath{stroke,fill}%
}%
\begin{pgfscope}%
\pgfsys@transformshift{1.053750in}{7.147936in}%
\pgfsys@useobject{currentmarker}{}%
\end{pgfscope}%
\end{pgfscope}%
\begin{pgfscope}%
\definecolor{textcolor}{rgb}{0.000000,0.000000,0.000000}%
\pgfsetstrokecolor{textcolor}%
\pgfsetfillcolor{textcolor}%
\pgftext[x=0.585434in,y=7.084622in,left,base]{\color{textcolor}\sffamily\fontsize{12.000000}{14.400000}\selectfont 0.75}%
\end{pgfscope}%
\begin{pgfscope}%
\pgfsetbuttcap%
\pgfsetroundjoin%
\definecolor{currentfill}{rgb}{0.000000,0.000000,0.000000}%
\pgfsetfillcolor{currentfill}%
\pgfsetlinewidth{0.803000pt}%
\definecolor{currentstroke}{rgb}{0.000000,0.000000,0.000000}%
\pgfsetstrokecolor{currentstroke}%
\pgfsetdash{}{0pt}%
\pgfsys@defobject{currentmarker}{\pgfqpoint{-0.048611in}{0.000000in}}{\pgfqpoint{0.000000in}{0.000000in}}{%
\pgfpathmoveto{\pgfqpoint{0.000000in}{0.000000in}}%
\pgfpathlineto{\pgfqpoint{-0.048611in}{0.000000in}}%
\pgfusepath{stroke,fill}%
}%
\begin{pgfscope}%
\pgfsys@transformshift{1.053750in}{7.552682in}%
\pgfsys@useobject{currentmarker}{}%
\end{pgfscope}%
\end{pgfscope}%
\begin{pgfscope}%
\definecolor{textcolor}{rgb}{0.000000,0.000000,0.000000}%
\pgfsetstrokecolor{textcolor}%
\pgfsetfillcolor{textcolor}%
\pgftext[x=0.585434in,y=7.489368in,left,base]{\color{textcolor}\sffamily\fontsize{12.000000}{14.400000}\selectfont 0.80}%
\end{pgfscope}%
\begin{pgfscope}%
\pgfsetbuttcap%
\pgfsetroundjoin%
\definecolor{currentfill}{rgb}{0.000000,0.000000,0.000000}%
\pgfsetfillcolor{currentfill}%
\pgfsetlinewidth{0.803000pt}%
\definecolor{currentstroke}{rgb}{0.000000,0.000000,0.000000}%
\pgfsetstrokecolor{currentstroke}%
\pgfsetdash{}{0pt}%
\pgfsys@defobject{currentmarker}{\pgfqpoint{-0.048611in}{0.000000in}}{\pgfqpoint{0.000000in}{0.000000in}}{%
\pgfpathmoveto{\pgfqpoint{0.000000in}{0.000000in}}%
\pgfpathlineto{\pgfqpoint{-0.048611in}{0.000000in}}%
\pgfusepath{stroke,fill}%
}%
\begin{pgfscope}%
\pgfsys@transformshift{1.053750in}{7.957428in}%
\pgfsys@useobject{currentmarker}{}%
\end{pgfscope}%
\end{pgfscope}%
\begin{pgfscope}%
\definecolor{textcolor}{rgb}{0.000000,0.000000,0.000000}%
\pgfsetstrokecolor{textcolor}%
\pgfsetfillcolor{textcolor}%
\pgftext[x=0.585434in,y=7.894114in,left,base]{\color{textcolor}\sffamily\fontsize{12.000000}{14.400000}\selectfont 0.85}%
\end{pgfscope}%
\begin{pgfscope}%
\pgfsetbuttcap%
\pgfsetroundjoin%
\definecolor{currentfill}{rgb}{0.000000,0.000000,0.000000}%
\pgfsetfillcolor{currentfill}%
\pgfsetlinewidth{0.803000pt}%
\definecolor{currentstroke}{rgb}{0.000000,0.000000,0.000000}%
\pgfsetstrokecolor{currentstroke}%
\pgfsetdash{}{0pt}%
\pgfsys@defobject{currentmarker}{\pgfqpoint{-0.048611in}{0.000000in}}{\pgfqpoint{0.000000in}{0.000000in}}{%
\pgfpathmoveto{\pgfqpoint{0.000000in}{0.000000in}}%
\pgfpathlineto{\pgfqpoint{-0.048611in}{0.000000in}}%
\pgfusepath{stroke,fill}%
}%
\begin{pgfscope}%
\pgfsys@transformshift{1.053750in}{8.362174in}%
\pgfsys@useobject{currentmarker}{}%
\end{pgfscope}%
\end{pgfscope}%
\begin{pgfscope}%
\definecolor{textcolor}{rgb}{0.000000,0.000000,0.000000}%
\pgfsetstrokecolor{textcolor}%
\pgfsetfillcolor{textcolor}%
\pgftext[x=0.585434in,y=8.298861in,left,base]{\color{textcolor}\sffamily\fontsize{12.000000}{14.400000}\selectfont 0.90}%
\end{pgfscope}%
\begin{pgfscope}%
\pgfsetbuttcap%
\pgfsetroundjoin%
\definecolor{currentfill}{rgb}{0.000000,0.000000,0.000000}%
\pgfsetfillcolor{currentfill}%
\pgfsetlinewidth{0.803000pt}%
\definecolor{currentstroke}{rgb}{0.000000,0.000000,0.000000}%
\pgfsetstrokecolor{currentstroke}%
\pgfsetdash{}{0pt}%
\pgfsys@defobject{currentmarker}{\pgfqpoint{-0.048611in}{0.000000in}}{\pgfqpoint{0.000000in}{0.000000in}}{%
\pgfpathmoveto{\pgfqpoint{0.000000in}{0.000000in}}%
\pgfpathlineto{\pgfqpoint{-0.048611in}{0.000000in}}%
\pgfusepath{stroke,fill}%
}%
\begin{pgfscope}%
\pgfsys@transformshift{1.053750in}{8.766921in}%
\pgfsys@useobject{currentmarker}{}%
\end{pgfscope}%
\end{pgfscope}%
\begin{pgfscope}%
\definecolor{textcolor}{rgb}{0.000000,0.000000,0.000000}%
\pgfsetstrokecolor{textcolor}%
\pgfsetfillcolor{textcolor}%
\pgftext[x=0.585434in,y=8.703607in,left,base]{\color{textcolor}\sffamily\fontsize{12.000000}{14.400000}\selectfont 0.95}%
\end{pgfscope}%
\begin{pgfscope}%
\definecolor{textcolor}{rgb}{0.000000,0.000000,0.000000}%
\pgfsetstrokecolor{textcolor}%
\pgfsetfillcolor{textcolor}%
\pgftext[x=0.529878in,y=7.256054in,,bottom,rotate=90.000000]{\color{textcolor}\sffamily\fontsize{14.000000}{16.800000}\selectfont test\_accuracies}%
\end{pgfscope}%
\begin{pgfscope}%
\pgfpathrectangle{\pgfqpoint{1.053750in}{5.716891in}}{\pgfqpoint{6.533250in}{3.078326in}}%
\pgfusepath{clip}%
\pgfsetrectcap%
\pgfsetroundjoin%
\pgfsetlinewidth{1.505625pt}%
\definecolor{currentstroke}{rgb}{0.121569,0.466667,0.705882}%
\pgfsetstrokecolor{currentstroke}%
\pgfsetdash{}{0pt}%
\pgfpathmoveto{\pgfqpoint{1.043750in}{7.704119in}}%
\pgfpathlineto{\pgfqpoint{1.053750in}{7.704137in}}%
\pgfpathlineto{\pgfqpoint{2.026793in}{7.877646in}}%
\pgfpathlineto{\pgfqpoint{2.595987in}{8.073858in}}%
\pgfpathlineto{\pgfqpoint{2.999836in}{8.079533in}}%
\pgfpathlineto{\pgfqpoint{3.313086in}{8.138721in}}%
\pgfpathlineto{\pgfqpoint{3.569030in}{8.193855in}}%
\pgfpathlineto{\pgfqpoint{3.785427in}{8.183314in}}%
\pgfpathlineto{\pgfqpoint{3.972879in}{8.201962in}}%
\pgfpathlineto{\pgfqpoint{4.138224in}{8.221421in}}%
\pgfpathlineto{\pgfqpoint{4.286129in}{8.254664in}}%
\pgfpathlineto{\pgfqpoint{4.419926in}{8.243313in}}%
\pgfpathlineto{\pgfqpoint{4.542073in}{8.262772in}}%
\pgfpathlineto{\pgfqpoint{4.654437in}{8.264393in}}%
\pgfpathlineto{\pgfqpoint{4.758470in}{8.255475in}}%
\pgfpathlineto{\pgfqpoint{4.855323in}{8.290339in}}%
\pgfpathlineto{\pgfqpoint{4.945922in}{8.272501in}}%
\pgfpathlineto{\pgfqpoint{5.031027in}{8.278987in}}%
\pgfpathlineto{\pgfqpoint{5.111267in}{8.282231in}}%
\pgfpathlineto{\pgfqpoint{5.187166in}{8.297636in}}%
\pgfpathlineto{\pgfqpoint{5.259172in}{8.315473in}}%
\pgfpathlineto{\pgfqpoint{5.327664in}{8.306554in}}%
\pgfpathlineto{\pgfqpoint{5.392969in}{8.314662in}}%
\pgfpathlineto{\pgfqpoint{5.455371in}{8.330878in}}%
\pgfpathlineto{\pgfqpoint{5.515116in}{8.328446in}}%
\pgfpathlineto{\pgfqpoint{5.572422in}{8.338986in}}%
\pgfpathlineto{\pgfqpoint{5.627480in}{8.321959in}}%
\pgfpathlineto{\pgfqpoint{5.680460in}{8.329256in}}%
\pgfpathlineto{\pgfqpoint{5.731513in}{8.322770in}}%
\pgfpathlineto{\pgfqpoint{5.780775in}{8.347905in}}%
\pgfpathlineto{\pgfqpoint{5.828366in}{8.314662in}}%
\pgfpathlineto{\pgfqpoint{5.874396in}{8.353580in}}%
\pgfpathlineto{\pgfqpoint{5.918965in}{8.346283in}}%
\pgfpathlineto{\pgfqpoint{5.962163in}{8.341418in}}%
\pgfpathlineto{\pgfqpoint{6.004070in}{8.357634in}}%
\pgfpathlineto{\pgfqpoint{6.044763in}{8.361688in}}%
\pgfpathlineto{\pgfqpoint{6.084310in}{8.358445in}}%
\pgfpathlineto{\pgfqpoint{6.122772in}{8.373039in}}%
\pgfpathlineto{\pgfqpoint{6.160209in}{8.380336in}}%
\pgfpathlineto{\pgfqpoint{6.196674in}{8.354391in}}%
\pgfpathlineto{\pgfqpoint{6.232215in}{8.353580in}}%
\pgfpathlineto{\pgfqpoint{6.266879in}{8.377904in}}%
\pgfpathlineto{\pgfqpoint{6.300707in}{8.376282in}}%
\pgfpathlineto{\pgfqpoint{6.333739in}{8.360066in}}%
\pgfpathlineto{\pgfqpoint{6.366012in}{8.381147in}}%
\pgfpathlineto{\pgfqpoint{6.397560in}{8.370607in}}%
\pgfpathlineto{\pgfqpoint{6.428414in}{8.389255in}}%
\pgfpathlineto{\pgfqpoint{6.458604in}{8.376282in}}%
\pgfpathlineto{\pgfqpoint{6.488159in}{8.388444in}}%
\pgfpathlineto{\pgfqpoint{6.517104in}{8.380336in}}%
\pgfpathlineto{\pgfqpoint{6.545465in}{8.394930in}}%
\pgfpathlineto{\pgfqpoint{6.573264in}{8.380336in}}%
\pgfpathlineto{\pgfqpoint{6.600523in}{8.385201in}}%
\pgfpathlineto{\pgfqpoint{6.627263in}{8.381147in}}%
\pgfpathlineto{\pgfqpoint{6.653503in}{8.390066in}}%
\pgfpathlineto{\pgfqpoint{6.679262in}{8.377904in}}%
\pgfpathlineto{\pgfqpoint{6.704556in}{8.376282in}}%
\pgfpathlineto{\pgfqpoint{6.729403in}{8.394120in}}%
\pgfpathlineto{\pgfqpoint{6.753818in}{8.405471in}}%
\pgfpathlineto{\pgfqpoint{6.777815in}{8.389255in}}%
\pgfpathlineto{\pgfqpoint{6.801409in}{8.401417in}}%
\pgfpathlineto{\pgfqpoint{6.824613in}{8.390876in}}%
\pgfpathlineto{\pgfqpoint{6.847439in}{8.390876in}}%
\pgfpathlineto{\pgfqpoint{6.869901in}{8.399795in}}%
\pgfpathlineto{\pgfqpoint{6.892008in}{8.391687in}}%
\pgfpathlineto{\pgfqpoint{6.913773in}{8.397363in}}%
\pgfpathlineto{\pgfqpoint{6.935206in}{8.403849in}}%
\pgfpathlineto{\pgfqpoint{6.956316in}{8.403038in}}%
\pgfpathlineto{\pgfqpoint{6.977113in}{8.398984in}}%
\pgfpathlineto{\pgfqpoint{6.997607in}{8.397363in}}%
\pgfpathlineto{\pgfqpoint{7.017806in}{8.392498in}}%
\pgfpathlineto{\pgfqpoint{7.037719in}{8.411146in}}%
\pgfpathlineto{\pgfqpoint{7.057353in}{8.380336in}}%
\pgfpathlineto{\pgfqpoint{7.076716in}{8.377904in}}%
\pgfpathlineto{\pgfqpoint{7.095816in}{8.393309in}}%
\pgfpathlineto{\pgfqpoint{7.114659in}{8.403038in}}%
\pgfpathlineto{\pgfqpoint{7.133253in}{8.399795in}}%
\pgfpathlineto{\pgfqpoint{7.151603in}{8.398984in}}%
\pgfpathlineto{\pgfqpoint{7.169717in}{8.342229in}}%
\pgfpathlineto{\pgfqpoint{7.187600in}{8.416822in}}%
\pgfpathlineto{\pgfqpoint{7.205258in}{8.406281in}}%
\pgfpathlineto{\pgfqpoint{7.222697in}{8.414389in}}%
\pgfpathlineto{\pgfqpoint{7.239922in}{8.409525in}}%
\pgfpathlineto{\pgfqpoint{7.256938in}{8.411146in}}%
\pgfpathlineto{\pgfqpoint{7.273750in}{8.401417in}}%
\pgfpathlineto{\pgfqpoint{7.290363in}{8.403038in}}%
\pgfpathlineto{\pgfqpoint{7.306782in}{8.400606in}}%
\pgfpathlineto{\pgfqpoint{7.323011in}{8.417633in}}%
\pgfpathlineto{\pgfqpoint{7.339055in}{8.404660in}}%
\pgfpathlineto{\pgfqpoint{7.354917in}{8.420876in}}%
\pgfpathlineto{\pgfqpoint{7.370603in}{8.413579in}}%
\pgfpathlineto{\pgfqpoint{7.386114in}{8.416011in}}%
\pgfpathlineto{\pgfqpoint{7.401457in}{8.430605in}}%
\pgfpathlineto{\pgfqpoint{7.416633in}{8.416822in}}%
\pgfpathlineto{\pgfqpoint{7.431647in}{8.415200in}}%
\pgfpathlineto{\pgfqpoint{7.446502in}{8.411957in}}%
\pgfpathlineto{\pgfqpoint{7.461202in}{8.417633in}}%
\pgfpathlineto{\pgfqpoint{7.475749in}{8.418443in}}%
\pgfpathlineto{\pgfqpoint{7.490148in}{8.403038in}}%
\pgfpathlineto{\pgfqpoint{7.504399in}{8.423308in}}%
\pgfpathlineto{\pgfqpoint{7.518508in}{8.406281in}}%
\pgfusepath{stroke}%
\end{pgfscope}%
\begin{pgfscope}%
\pgfpathrectangle{\pgfqpoint{1.053750in}{5.716891in}}{\pgfqpoint{6.533250in}{3.078326in}}%
\pgfusepath{clip}%
\pgfsetrectcap%
\pgfsetroundjoin%
\pgfsetlinewidth{1.505625pt}%
\definecolor{currentstroke}{rgb}{1.000000,0.498039,0.054902}%
\pgfsetstrokecolor{currentstroke}%
\pgfsetdash{}{0pt}%
\pgfpathmoveto{\pgfqpoint{1.043750in}{7.656283in}}%
\pgfpathlineto{\pgfqpoint{1.053750in}{7.656301in}}%
\pgfpathlineto{\pgfqpoint{2.026793in}{7.884133in}}%
\pgfpathlineto{\pgfqpoint{2.595987in}{8.054399in}}%
\pgfpathlineto{\pgfqpoint{2.999836in}{8.017102in}}%
\pgfpathlineto{\pgfqpoint{3.313086in}{8.120073in}}%
\pgfpathlineto{\pgfqpoint{3.569030in}{8.194665in}}%
\pgfpathlineto{\pgfqpoint{3.785427in}{8.199530in}}%
\pgfpathlineto{\pgfqpoint{3.972879in}{8.182503in}}%
\pgfpathlineto{\pgfqpoint{4.138224in}{8.222232in}}%
\pgfpathlineto{\pgfqpoint{4.286129in}{8.153315in}}%
\pgfpathlineto{\pgfqpoint{4.419926in}{8.254664in}}%
\pgfpathlineto{\pgfqpoint{4.542073in}{8.270880in}}%
\pgfpathlineto{\pgfqpoint{4.654437in}{8.266015in}}%
\pgfpathlineto{\pgfqpoint{4.758470in}{8.175206in}}%
\pgfpathlineto{\pgfqpoint{4.855323in}{8.322770in}}%
\pgfpathlineto{\pgfqpoint{4.945922in}{8.257907in}}%
\pgfpathlineto{\pgfqpoint{5.031027in}{8.314662in}}%
\pgfpathlineto{\pgfqpoint{5.111267in}{8.248177in}}%
\pgfpathlineto{\pgfqpoint{5.187166in}{8.346283in}}%
\pgfpathlineto{\pgfqpoint{5.259172in}{8.315473in}}%
\pgfpathlineto{\pgfqpoint{5.327664in}{8.342229in}}%
\pgfpathlineto{\pgfqpoint{5.392969in}{8.341418in}}%
\pgfpathlineto{\pgfqpoint{5.455371in}{8.321149in}}%
\pgfpathlineto{\pgfqpoint{5.515116in}{8.386823in}}%
\pgfpathlineto{\pgfqpoint{5.572422in}{8.380336in}}%
\pgfpathlineto{\pgfqpoint{5.627480in}{8.380336in}}%
\pgfpathlineto{\pgfqpoint{5.680460in}{8.255475in}}%
\pgfpathlineto{\pgfqpoint{5.731513in}{8.362499in}}%
\pgfpathlineto{\pgfqpoint{5.780775in}{8.373850in}}%
\pgfpathlineto{\pgfqpoint{5.828366in}{8.103046in}}%
\pgfpathlineto{\pgfqpoint{5.874396in}{8.417633in}}%
\pgfpathlineto{\pgfqpoint{5.918965in}{8.368174in}}%
\pgfpathlineto{\pgfqpoint{5.962163in}{8.396552in}}%
\pgfpathlineto{\pgfqpoint{6.004070in}{8.392498in}}%
\pgfpathlineto{\pgfqpoint{6.044763in}{8.438713in}}%
\pgfpathlineto{\pgfqpoint{6.084310in}{8.393309in}}%
\pgfpathlineto{\pgfqpoint{6.122772in}{8.403038in}}%
\pgfpathlineto{\pgfqpoint{6.160209in}{8.398984in}}%
\pgfpathlineto{\pgfqpoint{6.196674in}{8.411957in}}%
\pgfpathlineto{\pgfqpoint{6.232215in}{8.390876in}}%
\pgfpathlineto{\pgfqpoint{6.266879in}{8.419254in}}%
\pgfpathlineto{\pgfqpoint{6.300707in}{8.424119in}}%
\pgfpathlineto{\pgfqpoint{6.333739in}{8.434659in}}%
\pgfpathlineto{\pgfqpoint{6.366012in}{8.373850in}}%
\pgfpathlineto{\pgfqpoint{6.397560in}{8.433038in}}%
\pgfpathlineto{\pgfqpoint{6.428414in}{8.399795in}}%
\pgfpathlineto{\pgfqpoint{6.458604in}{8.433038in}}%
\pgfpathlineto{\pgfqpoint{6.488159in}{8.433038in}}%
\pgfpathlineto{\pgfqpoint{6.517104in}{8.364120in}}%
\pgfpathlineto{\pgfqpoint{6.545465in}{8.454929in}}%
\pgfpathlineto{\pgfqpoint{6.573264in}{8.416011in}}%
\pgfpathlineto{\pgfqpoint{6.600523in}{8.334121in}}%
\pgfpathlineto{\pgfqpoint{6.627263in}{8.441145in}}%
\pgfpathlineto{\pgfqpoint{6.653503in}{8.454118in}}%
\pgfpathlineto{\pgfqpoint{6.679262in}{8.443578in}}%
\pgfpathlineto{\pgfqpoint{6.704556in}{8.425740in}}%
\pgfpathlineto{\pgfqpoint{6.729403in}{8.435470in}}%
\pgfpathlineto{\pgfqpoint{6.753818in}{8.437091in}}%
\pgfpathlineto{\pgfqpoint{6.777815in}{8.436281in}}%
\pgfpathlineto{\pgfqpoint{6.801409in}{8.446010in}}%
\pgfpathlineto{\pgfqpoint{6.824613in}{8.439524in}}%
\pgfpathlineto{\pgfqpoint{6.847439in}{8.426551in}}%
\pgfpathlineto{\pgfqpoint{6.869901in}{8.456550in}}%
\pgfpathlineto{\pgfqpoint{6.892008in}{8.443578in}}%
\pgfpathlineto{\pgfqpoint{6.913773in}{8.429794in}}%
\pgfpathlineto{\pgfqpoint{6.935206in}{8.451686in}}%
\pgfpathlineto{\pgfqpoint{6.956316in}{8.438713in}}%
\pgfpathlineto{\pgfqpoint{6.977113in}{8.428173in}}%
\pgfpathlineto{\pgfqpoint{6.997607in}{8.444389in}}%
\pgfpathlineto{\pgfqpoint{7.017806in}{8.411957in}}%
\pgfpathlineto{\pgfqpoint{7.037719in}{8.461415in}}%
\pgfpathlineto{\pgfqpoint{7.057353in}{8.451686in}}%
\pgfpathlineto{\pgfqpoint{7.076716in}{8.433848in}}%
\pgfpathlineto{\pgfqpoint{7.095816in}{8.436281in}}%
\pgfpathlineto{\pgfqpoint{7.114659in}{8.424930in}}%
\pgfpathlineto{\pgfqpoint{7.133253in}{8.441145in}}%
\pgfpathlineto{\pgfqpoint{7.151603in}{8.414389in}}%
\pgfpathlineto{\pgfqpoint{7.169717in}{8.359256in}}%
\pgfpathlineto{\pgfqpoint{7.187600in}{8.463037in}}%
\pgfpathlineto{\pgfqpoint{7.205258in}{8.416011in}}%
\pgfpathlineto{\pgfqpoint{7.222697in}{8.458983in}}%
\pgfpathlineto{\pgfqpoint{7.239922in}{8.420876in}}%
\pgfpathlineto{\pgfqpoint{7.256938in}{8.462226in}}%
\pgfpathlineto{\pgfqpoint{7.273750in}{8.463848in}}%
\pgfpathlineto{\pgfqpoint{7.290363in}{8.441956in}}%
\pgfpathlineto{\pgfqpoint{7.306782in}{8.465469in}}%
\pgfpathlineto{\pgfqpoint{7.323011in}{8.463848in}}%
\pgfpathlineto{\pgfqpoint{7.339055in}{8.450064in}}%
\pgfpathlineto{\pgfqpoint{7.354917in}{8.447632in}}%
\pgfpathlineto{\pgfqpoint{7.370603in}{8.458983in}}%
\pgfpathlineto{\pgfqpoint{7.386114in}{8.463037in}}%
\pgfpathlineto{\pgfqpoint{7.401457in}{8.460604in}}%
\pgfpathlineto{\pgfqpoint{7.416633in}{8.451686in}}%
\pgfpathlineto{\pgfqpoint{7.431647in}{8.439524in}}%
\pgfpathlineto{\pgfqpoint{7.446502in}{8.457361in}}%
\pgfpathlineto{\pgfqpoint{7.461202in}{8.454118in}}%
\pgfpathlineto{\pgfqpoint{7.475749in}{8.430605in}}%
\pgfpathlineto{\pgfqpoint{7.490148in}{8.444389in}}%
\pgfpathlineto{\pgfqpoint{7.504399in}{8.458172in}}%
\pgfpathlineto{\pgfqpoint{7.518508in}{8.462226in}}%
\pgfusepath{stroke}%
\end{pgfscope}%
\begin{pgfscope}%
\pgfsetrectcap%
\pgfsetmiterjoin%
\pgfsetlinewidth{0.803000pt}%
\definecolor{currentstroke}{rgb}{0.000000,0.000000,0.000000}%
\pgfsetstrokecolor{currentstroke}%
\pgfsetdash{}{0pt}%
\pgfpathmoveto{\pgfqpoint{1.053750in}{5.716891in}}%
\pgfpathlineto{\pgfqpoint{1.053750in}{8.795217in}}%
\pgfusepath{stroke}%
\end{pgfscope}%
\begin{pgfscope}%
\pgfsetrectcap%
\pgfsetmiterjoin%
\pgfsetlinewidth{0.803000pt}%
\definecolor{currentstroke}{rgb}{0.000000,0.000000,0.000000}%
\pgfsetstrokecolor{currentstroke}%
\pgfsetdash{}{0pt}%
\pgfpathmoveto{\pgfqpoint{7.587000in}{5.716891in}}%
\pgfpathlineto{\pgfqpoint{7.587000in}{8.795217in}}%
\pgfusepath{stroke}%
\end{pgfscope}%
\begin{pgfscope}%
\pgfsetrectcap%
\pgfsetmiterjoin%
\pgfsetlinewidth{0.803000pt}%
\definecolor{currentstroke}{rgb}{0.000000,0.000000,0.000000}%
\pgfsetstrokecolor{currentstroke}%
\pgfsetdash{}{0pt}%
\pgfpathmoveto{\pgfqpoint{1.053750in}{5.716891in}}%
\pgfpathlineto{\pgfqpoint{7.587000in}{5.716891in}}%
\pgfusepath{stroke}%
\end{pgfscope}%
\begin{pgfscope}%
\pgfsetrectcap%
\pgfsetmiterjoin%
\pgfsetlinewidth{0.803000pt}%
\definecolor{currentstroke}{rgb}{0.000000,0.000000,0.000000}%
\pgfsetstrokecolor{currentstroke}%
\pgfsetdash{}{0pt}%
\pgfpathmoveto{\pgfqpoint{1.053750in}{8.795217in}}%
\pgfpathlineto{\pgfqpoint{7.587000in}{8.795217in}}%
\pgfusepath{stroke}%
\end{pgfscope}%
\begin{pgfscope}%
\pgfsetbuttcap%
\pgfsetmiterjoin%
\definecolor{currentfill}{rgb}{1.000000,1.000000,1.000000}%
\pgfsetfillcolor{currentfill}%
\pgfsetlinewidth{0.000000pt}%
\definecolor{currentstroke}{rgb}{0.000000,0.000000,0.000000}%
\pgfsetstrokecolor{currentstroke}%
\pgfsetstrokeopacity{0.000000}%
\pgfsetdash{}{0pt}%
\pgfpathmoveto{\pgfqpoint{1.053750in}{2.022900in}}%
\pgfpathlineto{\pgfqpoint{7.587000in}{2.022900in}}%
\pgfpathlineto{\pgfqpoint{7.587000in}{5.101226in}}%
\pgfpathlineto{\pgfqpoint{1.053750in}{5.101226in}}%
\pgfpathclose%
\pgfusepath{fill}%
\end{pgfscope}%
\begin{pgfscope}%
\pgfpathrectangle{\pgfqpoint{1.053750in}{2.022900in}}{\pgfqpoint{6.533250in}{3.078326in}}%
\pgfusepath{clip}%
\pgfsetbuttcap%
\pgfsetroundjoin%
\definecolor{currentfill}{rgb}{0.121569,0.466667,0.705882}%
\pgfsetfillcolor{currentfill}%
\pgfsetfillopacity{0.300000}%
\pgfsetlinewidth{0.000000pt}%
\definecolor{currentstroke}{rgb}{0.000000,0.000000,0.000000}%
\pgfsetstrokecolor{currentstroke}%
\pgfsetdash{}{0pt}%
\pgfpathmoveto{\pgfqpoint{-3231.325327in}{2.162824in}}%
\pgfpathlineto{\pgfqpoint{-3231.325327in}{2.162824in}}%
\pgfpathlineto{\pgfqpoint{1.053750in}{4.476280in}}%
\pgfpathlineto{\pgfqpoint{2.026793in}{4.576739in}}%
\pgfpathlineto{\pgfqpoint{2.595987in}{4.675600in}}%
\pgfpathlineto{\pgfqpoint{2.999836in}{4.636887in}}%
\pgfpathlineto{\pgfqpoint{3.313086in}{4.691596in}}%
\pgfpathlineto{\pgfqpoint{3.569030in}{4.692876in}}%
\pgfpathlineto{\pgfqpoint{3.785427in}{4.706633in}}%
\pgfpathlineto{\pgfqpoint{3.972879in}{4.719431in}}%
\pgfpathlineto{\pgfqpoint{4.138224in}{4.715911in}}%
\pgfpathlineto{\pgfqpoint{4.286129in}{4.725190in}}%
\pgfpathlineto{\pgfqpoint{4.419926in}{4.729029in}}%
\pgfpathlineto{\pgfqpoint{4.542073in}{4.747265in}}%
\pgfpathlineto{\pgfqpoint{4.654437in}{4.753344in}}%
\pgfpathlineto{\pgfqpoint{4.758470in}{4.733188in}}%
\pgfpathlineto{\pgfqpoint{4.855323in}{4.761662in}}%
\pgfpathlineto{\pgfqpoint{4.945922in}{4.742786in}}%
\pgfpathlineto{\pgfqpoint{5.031027in}{4.755264in}}%
\pgfpathlineto{\pgfqpoint{5.111267in}{4.761342in}}%
\pgfpathlineto{\pgfqpoint{5.187166in}{4.764222in}}%
\pgfpathlineto{\pgfqpoint{5.259172in}{4.784378in}}%
\pgfpathlineto{\pgfqpoint{5.327664in}{4.763902in}}%
\pgfpathlineto{\pgfqpoint{5.392969in}{4.773820in}}%
\pgfpathlineto{\pgfqpoint{5.455371in}{4.755903in}}%
\pgfpathlineto{\pgfqpoint{5.515116in}{4.772220in}}%
\pgfpathlineto{\pgfqpoint{5.572422in}{4.796535in}}%
\pgfpathlineto{\pgfqpoint{5.627480in}{4.779899in}}%
\pgfpathlineto{\pgfqpoint{5.680460in}{4.778299in}}%
\pgfpathlineto{\pgfqpoint{5.731513in}{4.792376in}}%
\pgfpathlineto{\pgfqpoint{5.780775in}{4.783418in}}%
\pgfpathlineto{\pgfqpoint{5.828366in}{4.770301in}}%
\pgfpathlineto{\pgfqpoint{5.874396in}{4.789177in}}%
\pgfpathlineto{\pgfqpoint{5.918965in}{4.800055in}}%
\pgfpathlineto{\pgfqpoint{5.962163in}{4.784378in}}%
\pgfpathlineto{\pgfqpoint{6.004070in}{4.798135in}}%
\pgfpathlineto{\pgfqpoint{6.044763in}{4.802294in}}%
\pgfpathlineto{\pgfqpoint{6.084310in}{4.801654in}}%
\pgfpathlineto{\pgfqpoint{6.122772in}{4.804854in}}%
\pgfpathlineto{\pgfqpoint{6.160209in}{4.800694in}}%
\pgfpathlineto{\pgfqpoint{6.196674in}{4.809333in}}%
\pgfpathlineto{\pgfqpoint{6.232215in}{4.801014in}}%
\pgfpathlineto{\pgfqpoint{6.266879in}{4.800055in}}%
\pgfpathlineto{\pgfqpoint{6.300707in}{4.791096in}}%
\pgfpathlineto{\pgfqpoint{6.333739in}{4.816371in}}%
\pgfpathlineto{\pgfqpoint{6.366012in}{4.817651in}}%
\pgfpathlineto{\pgfqpoint{6.397560in}{4.810612in}}%
\pgfpathlineto{\pgfqpoint{6.428414in}{4.810612in}}%
\pgfpathlineto{\pgfqpoint{6.458604in}{4.816371in}}%
\pgfpathlineto{\pgfqpoint{6.488159in}{4.829809in}}%
\pgfpathlineto{\pgfqpoint{6.517104in}{4.804534in}}%
\pgfpathlineto{\pgfqpoint{6.545465in}{4.809013in}}%
\pgfpathlineto{\pgfqpoint{6.573264in}{4.808373in}}%
\pgfpathlineto{\pgfqpoint{6.600523in}{4.826289in}}%
\pgfpathlineto{\pgfqpoint{6.627263in}{4.823730in}}%
\pgfpathlineto{\pgfqpoint{6.653503in}{4.837807in}}%
\pgfpathlineto{\pgfqpoint{6.679262in}{4.828849in}}%
\pgfpathlineto{\pgfqpoint{6.704556in}{4.814452in}}%
\pgfpathlineto{\pgfqpoint{6.729403in}{4.831408in}}%
\pgfpathlineto{\pgfqpoint{6.753818in}{4.834608in}}%
\pgfpathlineto{\pgfqpoint{6.777815in}{4.840686in}}%
\pgfpathlineto{\pgfqpoint{6.801409in}{4.837167in}}%
\pgfpathlineto{\pgfqpoint{6.824613in}{4.821170in}}%
\pgfpathlineto{\pgfqpoint{6.847439in}{4.845485in}}%
\pgfpathlineto{\pgfqpoint{6.869901in}{4.834928in}}%
\pgfpathlineto{\pgfqpoint{6.892008in}{4.825329in}}%
\pgfpathlineto{\pgfqpoint{6.913773in}{4.830448in}}%
\pgfpathlineto{\pgfqpoint{6.935206in}{4.837487in}}%
\pgfpathlineto{\pgfqpoint{6.956316in}{4.838767in}}%
\pgfpathlineto{\pgfqpoint{6.977113in}{4.844206in}}%
\pgfpathlineto{\pgfqpoint{6.997607in}{4.831088in}}%
\pgfpathlineto{\pgfqpoint{7.017806in}{4.844846in}}%
\pgfpathlineto{\pgfqpoint{7.037719in}{4.848045in}}%
\pgfpathlineto{\pgfqpoint{7.057353in}{4.829809in}}%
\pgfpathlineto{\pgfqpoint{7.076716in}{4.830129in}}%
\pgfpathlineto{\pgfqpoint{7.095816in}{4.847725in}}%
\pgfpathlineto{\pgfqpoint{7.114659in}{4.839727in}}%
\pgfpathlineto{\pgfqpoint{7.133253in}{4.822450in}}%
\pgfpathlineto{\pgfqpoint{7.151603in}{4.853164in}}%
\pgfpathlineto{\pgfqpoint{7.169717in}{4.813812in}}%
\pgfpathlineto{\pgfqpoint{7.187600in}{4.853484in}}%
\pgfpathlineto{\pgfqpoint{7.205258in}{4.851884in}}%
\pgfpathlineto{\pgfqpoint{7.222697in}{4.841006in}}%
\pgfpathlineto{\pgfqpoint{7.239922in}{4.846765in}}%
\pgfpathlineto{\pgfqpoint{7.256938in}{4.851564in}}%
\pgfpathlineto{\pgfqpoint{7.273750in}{4.853484in}}%
\pgfpathlineto{\pgfqpoint{7.290363in}{4.860842in}}%
\pgfpathlineto{\pgfqpoint{7.306782in}{4.852204in}}%
\pgfpathlineto{\pgfqpoint{7.323011in}{4.859883in}}%
\pgfpathlineto{\pgfqpoint{7.339055in}{4.867561in}}%
\pgfpathlineto{\pgfqpoint{7.354917in}{4.847085in}}%
\pgfpathlineto{\pgfqpoint{7.370603in}{4.858283in}}%
\pgfpathlineto{\pgfqpoint{7.386114in}{4.850924in}}%
\pgfpathlineto{\pgfqpoint{7.401457in}{4.857003in}}%
\pgfpathlineto{\pgfqpoint{7.416633in}{4.860842in}}%
\pgfpathlineto{\pgfqpoint{7.431647in}{4.863722in}}%
\pgfpathlineto{\pgfqpoint{7.446502in}{4.854124in}}%
\pgfpathlineto{\pgfqpoint{7.461202in}{4.853484in}}%
\pgfpathlineto{\pgfqpoint{7.475749in}{4.874280in}}%
\pgfpathlineto{\pgfqpoint{7.490148in}{4.850604in}}%
\pgfpathlineto{\pgfqpoint{7.504399in}{4.867881in}}%
\pgfpathlineto{\pgfqpoint{7.518508in}{4.872040in}}%
\pgfpathlineto{\pgfqpoint{7.518508in}{4.872040in}}%
\pgfpathlineto{\pgfqpoint{7.518508in}{4.872040in}}%
\pgfpathlineto{\pgfqpoint{7.504399in}{4.867881in}}%
\pgfpathlineto{\pgfqpoint{7.490148in}{4.850604in}}%
\pgfpathlineto{\pgfqpoint{7.475749in}{4.874280in}}%
\pgfpathlineto{\pgfqpoint{7.461202in}{4.853484in}}%
\pgfpathlineto{\pgfqpoint{7.446502in}{4.854124in}}%
\pgfpathlineto{\pgfqpoint{7.431647in}{4.863722in}}%
\pgfpathlineto{\pgfqpoint{7.416633in}{4.860842in}}%
\pgfpathlineto{\pgfqpoint{7.401457in}{4.857003in}}%
\pgfpathlineto{\pgfqpoint{7.386114in}{4.850924in}}%
\pgfpathlineto{\pgfqpoint{7.370603in}{4.858283in}}%
\pgfpathlineto{\pgfqpoint{7.354917in}{4.847085in}}%
\pgfpathlineto{\pgfqpoint{7.339055in}{4.867561in}}%
\pgfpathlineto{\pgfqpoint{7.323011in}{4.859883in}}%
\pgfpathlineto{\pgfqpoint{7.306782in}{4.852204in}}%
\pgfpathlineto{\pgfqpoint{7.290363in}{4.860842in}}%
\pgfpathlineto{\pgfqpoint{7.273750in}{4.853484in}}%
\pgfpathlineto{\pgfqpoint{7.256938in}{4.851564in}}%
\pgfpathlineto{\pgfqpoint{7.239922in}{4.846765in}}%
\pgfpathlineto{\pgfqpoint{7.222697in}{4.841006in}}%
\pgfpathlineto{\pgfqpoint{7.205258in}{4.851884in}}%
\pgfpathlineto{\pgfqpoint{7.187600in}{4.853484in}}%
\pgfpathlineto{\pgfqpoint{7.169717in}{4.813812in}}%
\pgfpathlineto{\pgfqpoint{7.151603in}{4.853164in}}%
\pgfpathlineto{\pgfqpoint{7.133253in}{4.822450in}}%
\pgfpathlineto{\pgfqpoint{7.114659in}{4.839727in}}%
\pgfpathlineto{\pgfqpoint{7.095816in}{4.847725in}}%
\pgfpathlineto{\pgfqpoint{7.076716in}{4.830129in}}%
\pgfpathlineto{\pgfqpoint{7.057353in}{4.829809in}}%
\pgfpathlineto{\pgfqpoint{7.037719in}{4.848045in}}%
\pgfpathlineto{\pgfqpoint{7.017806in}{4.844846in}}%
\pgfpathlineto{\pgfqpoint{6.997607in}{4.831088in}}%
\pgfpathlineto{\pgfqpoint{6.977113in}{4.844206in}}%
\pgfpathlineto{\pgfqpoint{6.956316in}{4.838767in}}%
\pgfpathlineto{\pgfqpoint{6.935206in}{4.837487in}}%
\pgfpathlineto{\pgfqpoint{6.913773in}{4.830448in}}%
\pgfpathlineto{\pgfqpoint{6.892008in}{4.825329in}}%
\pgfpathlineto{\pgfqpoint{6.869901in}{4.834928in}}%
\pgfpathlineto{\pgfqpoint{6.847439in}{4.845485in}}%
\pgfpathlineto{\pgfqpoint{6.824613in}{4.821170in}}%
\pgfpathlineto{\pgfqpoint{6.801409in}{4.837167in}}%
\pgfpathlineto{\pgfqpoint{6.777815in}{4.840686in}}%
\pgfpathlineto{\pgfqpoint{6.753818in}{4.834608in}}%
\pgfpathlineto{\pgfqpoint{6.729403in}{4.831408in}}%
\pgfpathlineto{\pgfqpoint{6.704556in}{4.814452in}}%
\pgfpathlineto{\pgfqpoint{6.679262in}{4.828849in}}%
\pgfpathlineto{\pgfqpoint{6.653503in}{4.837807in}}%
\pgfpathlineto{\pgfqpoint{6.627263in}{4.823730in}}%
\pgfpathlineto{\pgfqpoint{6.600523in}{4.826289in}}%
\pgfpathlineto{\pgfqpoint{6.573264in}{4.808373in}}%
\pgfpathlineto{\pgfqpoint{6.545465in}{4.809013in}}%
\pgfpathlineto{\pgfqpoint{6.517104in}{4.804534in}}%
\pgfpathlineto{\pgfqpoint{6.488159in}{4.829809in}}%
\pgfpathlineto{\pgfqpoint{6.458604in}{4.816371in}}%
\pgfpathlineto{\pgfqpoint{6.428414in}{4.810612in}}%
\pgfpathlineto{\pgfqpoint{6.397560in}{4.810612in}}%
\pgfpathlineto{\pgfqpoint{6.366012in}{4.817651in}}%
\pgfpathlineto{\pgfqpoint{6.333739in}{4.816371in}}%
\pgfpathlineto{\pgfqpoint{6.300707in}{4.791096in}}%
\pgfpathlineto{\pgfqpoint{6.266879in}{4.800055in}}%
\pgfpathlineto{\pgfqpoint{6.232215in}{4.801014in}}%
\pgfpathlineto{\pgfqpoint{6.196674in}{4.809333in}}%
\pgfpathlineto{\pgfqpoint{6.160209in}{4.800694in}}%
\pgfpathlineto{\pgfqpoint{6.122772in}{4.804854in}}%
\pgfpathlineto{\pgfqpoint{6.084310in}{4.801654in}}%
\pgfpathlineto{\pgfqpoint{6.044763in}{4.802294in}}%
\pgfpathlineto{\pgfqpoint{6.004070in}{4.798135in}}%
\pgfpathlineto{\pgfqpoint{5.962163in}{4.784378in}}%
\pgfpathlineto{\pgfqpoint{5.918965in}{4.800055in}}%
\pgfpathlineto{\pgfqpoint{5.874396in}{4.789177in}}%
\pgfpathlineto{\pgfqpoint{5.828366in}{4.770301in}}%
\pgfpathlineto{\pgfqpoint{5.780775in}{4.783418in}}%
\pgfpathlineto{\pgfqpoint{5.731513in}{4.792376in}}%
\pgfpathlineto{\pgfqpoint{5.680460in}{4.778299in}}%
\pgfpathlineto{\pgfqpoint{5.627480in}{4.779899in}}%
\pgfpathlineto{\pgfqpoint{5.572422in}{4.796535in}}%
\pgfpathlineto{\pgfqpoint{5.515116in}{4.772220in}}%
\pgfpathlineto{\pgfqpoint{5.455371in}{4.755903in}}%
\pgfpathlineto{\pgfqpoint{5.392969in}{4.773820in}}%
\pgfpathlineto{\pgfqpoint{5.327664in}{4.763902in}}%
\pgfpathlineto{\pgfqpoint{5.259172in}{4.784378in}}%
\pgfpathlineto{\pgfqpoint{5.187166in}{4.764222in}}%
\pgfpathlineto{\pgfqpoint{5.111267in}{4.761342in}}%
\pgfpathlineto{\pgfqpoint{5.031027in}{4.755264in}}%
\pgfpathlineto{\pgfqpoint{4.945922in}{4.742786in}}%
\pgfpathlineto{\pgfqpoint{4.855323in}{4.761662in}}%
\pgfpathlineto{\pgfqpoint{4.758470in}{4.733188in}}%
\pgfpathlineto{\pgfqpoint{4.654437in}{4.753344in}}%
\pgfpathlineto{\pgfqpoint{4.542073in}{4.747265in}}%
\pgfpathlineto{\pgfqpoint{4.419926in}{4.729029in}}%
\pgfpathlineto{\pgfqpoint{4.286129in}{4.725190in}}%
\pgfpathlineto{\pgfqpoint{4.138224in}{4.715911in}}%
\pgfpathlineto{\pgfqpoint{3.972879in}{4.719431in}}%
\pgfpathlineto{\pgfqpoint{3.785427in}{4.706633in}}%
\pgfpathlineto{\pgfqpoint{3.569030in}{4.692876in}}%
\pgfpathlineto{\pgfqpoint{3.313086in}{4.691596in}}%
\pgfpathlineto{\pgfqpoint{2.999836in}{4.636887in}}%
\pgfpathlineto{\pgfqpoint{2.595987in}{4.675600in}}%
\pgfpathlineto{\pgfqpoint{2.026793in}{4.576739in}}%
\pgfpathlineto{\pgfqpoint{1.053750in}{4.476280in}}%
\pgfpathlineto{\pgfqpoint{-3231.325327in}{2.162824in}}%
\pgfpathclose%
\pgfusepath{fill}%
\end{pgfscope}%
\begin{pgfscope}%
\pgfpathrectangle{\pgfqpoint{1.053750in}{2.022900in}}{\pgfqpoint{6.533250in}{3.078326in}}%
\pgfusepath{clip}%
\pgfsetbuttcap%
\pgfsetroundjoin%
\definecolor{currentfill}{rgb}{1.000000,0.498039,0.054902}%
\pgfsetfillcolor{currentfill}%
\pgfsetfillopacity{0.300000}%
\pgfsetlinewidth{0.000000pt}%
\definecolor{currentstroke}{rgb}{0.000000,0.000000,0.000000}%
\pgfsetstrokecolor{currentstroke}%
\pgfsetdash{}{0pt}%
\pgfpathmoveto{\pgfqpoint{-3231.325327in}{2.162824in}}%
\pgfpathlineto{\pgfqpoint{-3231.325327in}{2.162824in}}%
\pgfpathlineto{\pgfqpoint{1.053750in}{4.452924in}}%
\pgfpathlineto{\pgfqpoint{2.026793in}{4.587297in}}%
\pgfpathlineto{\pgfqpoint{2.595987in}{4.674640in}}%
\pgfpathlineto{\pgfqpoint{2.999836in}{4.615132in}}%
\pgfpathlineto{\pgfqpoint{3.313086in}{4.684238in}}%
\pgfpathlineto{\pgfqpoint{3.569030in}{4.692876in}}%
\pgfpathlineto{\pgfqpoint{3.785427in}{4.715272in}}%
\pgfpathlineto{\pgfqpoint{3.972879in}{4.720071in}}%
\pgfpathlineto{\pgfqpoint{4.138224in}{4.730948in}}%
\pgfpathlineto{\pgfqpoint{4.286129in}{4.703754in}}%
\pgfpathlineto{\pgfqpoint{4.419926in}{4.737667in}}%
\pgfpathlineto{\pgfqpoint{4.542073in}{4.753984in}}%
\pgfpathlineto{\pgfqpoint{4.654437in}{4.752384in}}%
\pgfpathlineto{\pgfqpoint{4.758470in}{4.708553in}}%
\pgfpathlineto{\pgfqpoint{4.855323in}{4.786297in}}%
\pgfpathlineto{\pgfqpoint{4.945922in}{4.749185in}}%
\pgfpathlineto{\pgfqpoint{5.031027in}{4.777659in}}%
\pgfpathlineto{\pgfqpoint{5.111267in}{4.754624in}}%
\pgfpathlineto{\pgfqpoint{5.187166in}{4.794936in}}%
\pgfpathlineto{\pgfqpoint{5.259172in}{4.802934in}}%
\pgfpathlineto{\pgfqpoint{5.327664in}{4.782458in}}%
\pgfpathlineto{\pgfqpoint{5.392969in}{4.801654in}}%
\pgfpathlineto{\pgfqpoint{5.455371in}{4.762622in}}%
\pgfpathlineto{\pgfqpoint{5.515116in}{4.815092in}}%
\pgfpathlineto{\pgfqpoint{5.572422in}{4.830129in}}%
\pgfpathlineto{\pgfqpoint{5.627480in}{4.821170in}}%
\pgfpathlineto{\pgfqpoint{5.680460in}{4.776059in}}%
\pgfpathlineto{\pgfqpoint{5.731513in}{4.824050in}}%
\pgfpathlineto{\pgfqpoint{5.780775in}{4.815731in}}%
\pgfpathlineto{\pgfqpoint{5.828366in}{4.715911in}}%
\pgfpathlineto{\pgfqpoint{5.874396in}{4.840686in}}%
\pgfpathlineto{\pgfqpoint{5.918965in}{4.816371in}}%
\pgfpathlineto{\pgfqpoint{5.962163in}{4.837167in}}%
\pgfpathlineto{\pgfqpoint{6.004070in}{4.846445in}}%
\pgfpathlineto{\pgfqpoint{6.044763in}{4.851564in}}%
\pgfpathlineto{\pgfqpoint{6.084310in}{4.836527in}}%
\pgfpathlineto{\pgfqpoint{6.122772in}{4.852844in}}%
\pgfpathlineto{\pgfqpoint{6.160209in}{4.855723in}}%
\pgfpathlineto{\pgfqpoint{6.196674in}{4.853484in}}%
\pgfpathlineto{\pgfqpoint{6.232215in}{4.841646in}}%
\pgfpathlineto{\pgfqpoint{6.266879in}{4.847085in}}%
\pgfpathlineto{\pgfqpoint{6.300707in}{4.855723in}}%
\pgfpathlineto{\pgfqpoint{6.333739in}{4.872680in}}%
\pgfpathlineto{\pgfqpoint{6.366012in}{4.837487in}}%
\pgfpathlineto{\pgfqpoint{6.397560in}{4.869481in}}%
\pgfpathlineto{\pgfqpoint{6.428414in}{4.862442in}}%
\pgfpathlineto{\pgfqpoint{6.458604in}{4.872360in}}%
\pgfpathlineto{\pgfqpoint{6.488159in}{4.895715in}}%
\pgfpathlineto{\pgfqpoint{6.517104in}{4.846125in}}%
\pgfpathlineto{\pgfqpoint{6.545465in}{4.882598in}}%
\pgfpathlineto{\pgfqpoint{6.573264in}{4.861802in}}%
\pgfpathlineto{\pgfqpoint{6.600523in}{4.835887in}}%
\pgfpathlineto{\pgfqpoint{6.627263in}{4.885157in}}%
\pgfpathlineto{\pgfqpoint{6.653503in}{4.902754in}}%
\pgfpathlineto{\pgfqpoint{6.679262in}{4.896355in}}%
\pgfpathlineto{\pgfqpoint{6.704556in}{4.887717in}}%
\pgfpathlineto{\pgfqpoint{6.729403in}{4.895395in}}%
\pgfpathlineto{\pgfqpoint{6.753818in}{4.891236in}}%
\pgfpathlineto{\pgfqpoint{6.777815in}{4.905953in}}%
\pgfpathlineto{\pgfqpoint{6.801409in}{4.912992in}}%
\pgfpathlineto{\pgfqpoint{6.824613in}{4.901794in}}%
\pgfpathlineto{\pgfqpoint{6.847439in}{4.908513in}}%
\pgfpathlineto{\pgfqpoint{6.869901in}{4.919071in}}%
\pgfpathlineto{\pgfqpoint{6.892008in}{4.898595in}}%
\pgfpathlineto{\pgfqpoint{6.913773in}{4.902434in}}%
\pgfpathlineto{\pgfqpoint{6.935206in}{4.924190in}}%
\pgfpathlineto{\pgfqpoint{6.956316in}{4.910112in}}%
\pgfpathlineto{\pgfqpoint{6.977113in}{4.914592in}}%
\pgfpathlineto{\pgfqpoint{6.997607in}{4.915551in}}%
\pgfpathlineto{\pgfqpoint{7.017806in}{4.903714in}}%
\pgfpathlineto{\pgfqpoint{7.037719in}{4.922910in}}%
\pgfpathlineto{\pgfqpoint{7.057353in}{4.916511in}}%
\pgfpathlineto{\pgfqpoint{7.076716in}{4.920030in}}%
\pgfpathlineto{\pgfqpoint{7.095816in}{4.922270in}}%
\pgfpathlineto{\pgfqpoint{7.114659in}{4.912032in}}%
\pgfpathlineto{\pgfqpoint{7.133253in}{4.914592in}}%
\pgfpathlineto{\pgfqpoint{7.151603in}{4.905313in}}%
\pgfpathlineto{\pgfqpoint{7.169717in}{4.876519in}}%
\pgfpathlineto{\pgfqpoint{7.187600in}{4.938907in}}%
\pgfpathlineto{\pgfqpoint{7.205258in}{4.911392in}}%
\pgfpathlineto{\pgfqpoint{7.222697in}{4.927709in}}%
\pgfpathlineto{\pgfqpoint{7.239922in}{4.913952in}}%
\pgfpathlineto{\pgfqpoint{7.256938in}{4.937627in}}%
\pgfpathlineto{\pgfqpoint{7.273750in}{4.929309in}}%
\pgfpathlineto{\pgfqpoint{7.290363in}{4.943066in}}%
\pgfpathlineto{\pgfqpoint{7.306782in}{4.949465in}}%
\pgfpathlineto{\pgfqpoint{7.323011in}{4.941466in}}%
\pgfpathlineto{\pgfqpoint{7.339055in}{4.948185in}}%
\pgfpathlineto{\pgfqpoint{7.354917in}{4.923870in}}%
\pgfpathlineto{\pgfqpoint{7.370603in}{4.946585in}}%
\pgfpathlineto{\pgfqpoint{7.386114in}{4.941146in}}%
\pgfpathlineto{\pgfqpoint{7.401457in}{4.945625in}}%
\pgfpathlineto{\pgfqpoint{7.416633in}{4.945305in}}%
\pgfpathlineto{\pgfqpoint{7.431647in}{4.943386in}}%
\pgfpathlineto{\pgfqpoint{7.446502in}{4.946265in}}%
\pgfpathlineto{\pgfqpoint{7.461202in}{4.943066in}}%
\pgfpathlineto{\pgfqpoint{7.475749in}{4.942106in}}%
\pgfpathlineto{\pgfqpoint{7.490148in}{4.941466in}}%
\pgfpathlineto{\pgfqpoint{7.504399in}{4.961302in}}%
\pgfpathlineto{\pgfqpoint{7.518508in}{4.961302in}}%
\pgfpathlineto{\pgfqpoint{7.518508in}{4.961302in}}%
\pgfpathlineto{\pgfqpoint{7.518508in}{4.961302in}}%
\pgfpathlineto{\pgfqpoint{7.504399in}{4.961302in}}%
\pgfpathlineto{\pgfqpoint{7.490148in}{4.941466in}}%
\pgfpathlineto{\pgfqpoint{7.475749in}{4.942106in}}%
\pgfpathlineto{\pgfqpoint{7.461202in}{4.943066in}}%
\pgfpathlineto{\pgfqpoint{7.446502in}{4.946265in}}%
\pgfpathlineto{\pgfqpoint{7.431647in}{4.943386in}}%
\pgfpathlineto{\pgfqpoint{7.416633in}{4.945305in}}%
\pgfpathlineto{\pgfqpoint{7.401457in}{4.945625in}}%
\pgfpathlineto{\pgfqpoint{7.386114in}{4.941146in}}%
\pgfpathlineto{\pgfqpoint{7.370603in}{4.946585in}}%
\pgfpathlineto{\pgfqpoint{7.354917in}{4.923870in}}%
\pgfpathlineto{\pgfqpoint{7.339055in}{4.948185in}}%
\pgfpathlineto{\pgfqpoint{7.323011in}{4.941466in}}%
\pgfpathlineto{\pgfqpoint{7.306782in}{4.949465in}}%
\pgfpathlineto{\pgfqpoint{7.290363in}{4.943066in}}%
\pgfpathlineto{\pgfqpoint{7.273750in}{4.929309in}}%
\pgfpathlineto{\pgfqpoint{7.256938in}{4.937627in}}%
\pgfpathlineto{\pgfqpoint{7.239922in}{4.913952in}}%
\pgfpathlineto{\pgfqpoint{7.222697in}{4.927709in}}%
\pgfpathlineto{\pgfqpoint{7.205258in}{4.911392in}}%
\pgfpathlineto{\pgfqpoint{7.187600in}{4.938907in}}%
\pgfpathlineto{\pgfqpoint{7.169717in}{4.876519in}}%
\pgfpathlineto{\pgfqpoint{7.151603in}{4.905313in}}%
\pgfpathlineto{\pgfqpoint{7.133253in}{4.914592in}}%
\pgfpathlineto{\pgfqpoint{7.114659in}{4.912032in}}%
\pgfpathlineto{\pgfqpoint{7.095816in}{4.922270in}}%
\pgfpathlineto{\pgfqpoint{7.076716in}{4.920030in}}%
\pgfpathlineto{\pgfqpoint{7.057353in}{4.916511in}}%
\pgfpathlineto{\pgfqpoint{7.037719in}{4.922910in}}%
\pgfpathlineto{\pgfqpoint{7.017806in}{4.903714in}}%
\pgfpathlineto{\pgfqpoint{6.997607in}{4.915551in}}%
\pgfpathlineto{\pgfqpoint{6.977113in}{4.914592in}}%
\pgfpathlineto{\pgfqpoint{6.956316in}{4.910112in}}%
\pgfpathlineto{\pgfqpoint{6.935206in}{4.924190in}}%
\pgfpathlineto{\pgfqpoint{6.913773in}{4.902434in}}%
\pgfpathlineto{\pgfqpoint{6.892008in}{4.898595in}}%
\pgfpathlineto{\pgfqpoint{6.869901in}{4.919071in}}%
\pgfpathlineto{\pgfqpoint{6.847439in}{4.908513in}}%
\pgfpathlineto{\pgfqpoint{6.824613in}{4.901794in}}%
\pgfpathlineto{\pgfqpoint{6.801409in}{4.912992in}}%
\pgfpathlineto{\pgfqpoint{6.777815in}{4.905953in}}%
\pgfpathlineto{\pgfqpoint{6.753818in}{4.891236in}}%
\pgfpathlineto{\pgfqpoint{6.729403in}{4.895395in}}%
\pgfpathlineto{\pgfqpoint{6.704556in}{4.887717in}}%
\pgfpathlineto{\pgfqpoint{6.679262in}{4.896355in}}%
\pgfpathlineto{\pgfqpoint{6.653503in}{4.902754in}}%
\pgfpathlineto{\pgfqpoint{6.627263in}{4.885157in}}%
\pgfpathlineto{\pgfqpoint{6.600523in}{4.835887in}}%
\pgfpathlineto{\pgfqpoint{6.573264in}{4.861802in}}%
\pgfpathlineto{\pgfqpoint{6.545465in}{4.882598in}}%
\pgfpathlineto{\pgfqpoint{6.517104in}{4.846125in}}%
\pgfpathlineto{\pgfqpoint{6.488159in}{4.895715in}}%
\pgfpathlineto{\pgfqpoint{6.458604in}{4.872360in}}%
\pgfpathlineto{\pgfqpoint{6.428414in}{4.862442in}}%
\pgfpathlineto{\pgfqpoint{6.397560in}{4.869481in}}%
\pgfpathlineto{\pgfqpoint{6.366012in}{4.837487in}}%
\pgfpathlineto{\pgfqpoint{6.333739in}{4.872680in}}%
\pgfpathlineto{\pgfqpoint{6.300707in}{4.855723in}}%
\pgfpathlineto{\pgfqpoint{6.266879in}{4.847085in}}%
\pgfpathlineto{\pgfqpoint{6.232215in}{4.841646in}}%
\pgfpathlineto{\pgfqpoint{6.196674in}{4.853484in}}%
\pgfpathlineto{\pgfqpoint{6.160209in}{4.855723in}}%
\pgfpathlineto{\pgfqpoint{6.122772in}{4.852844in}}%
\pgfpathlineto{\pgfqpoint{6.084310in}{4.836527in}}%
\pgfpathlineto{\pgfqpoint{6.044763in}{4.851564in}}%
\pgfpathlineto{\pgfqpoint{6.004070in}{4.846445in}}%
\pgfpathlineto{\pgfqpoint{5.962163in}{4.837167in}}%
\pgfpathlineto{\pgfqpoint{5.918965in}{4.816371in}}%
\pgfpathlineto{\pgfqpoint{5.874396in}{4.840686in}}%
\pgfpathlineto{\pgfqpoint{5.828366in}{4.715911in}}%
\pgfpathlineto{\pgfqpoint{5.780775in}{4.815731in}}%
\pgfpathlineto{\pgfqpoint{5.731513in}{4.824050in}}%
\pgfpathlineto{\pgfqpoint{5.680460in}{4.776059in}}%
\pgfpathlineto{\pgfqpoint{5.627480in}{4.821170in}}%
\pgfpathlineto{\pgfqpoint{5.572422in}{4.830129in}}%
\pgfpathlineto{\pgfqpoint{5.515116in}{4.815092in}}%
\pgfpathlineto{\pgfqpoint{5.455371in}{4.762622in}}%
\pgfpathlineto{\pgfqpoint{5.392969in}{4.801654in}}%
\pgfpathlineto{\pgfqpoint{5.327664in}{4.782458in}}%
\pgfpathlineto{\pgfqpoint{5.259172in}{4.802934in}}%
\pgfpathlineto{\pgfqpoint{5.187166in}{4.794936in}}%
\pgfpathlineto{\pgfqpoint{5.111267in}{4.754624in}}%
\pgfpathlineto{\pgfqpoint{5.031027in}{4.777659in}}%
\pgfpathlineto{\pgfqpoint{4.945922in}{4.749185in}}%
\pgfpathlineto{\pgfqpoint{4.855323in}{4.786297in}}%
\pgfpathlineto{\pgfqpoint{4.758470in}{4.708553in}}%
\pgfpathlineto{\pgfqpoint{4.654437in}{4.752384in}}%
\pgfpathlineto{\pgfqpoint{4.542073in}{4.753984in}}%
\pgfpathlineto{\pgfqpoint{4.419926in}{4.737667in}}%
\pgfpathlineto{\pgfqpoint{4.286129in}{4.703754in}}%
\pgfpathlineto{\pgfqpoint{4.138224in}{4.730948in}}%
\pgfpathlineto{\pgfqpoint{3.972879in}{4.720071in}}%
\pgfpathlineto{\pgfqpoint{3.785427in}{4.715272in}}%
\pgfpathlineto{\pgfqpoint{3.569030in}{4.692876in}}%
\pgfpathlineto{\pgfqpoint{3.313086in}{4.684238in}}%
\pgfpathlineto{\pgfqpoint{2.999836in}{4.615132in}}%
\pgfpathlineto{\pgfqpoint{2.595987in}{4.674640in}}%
\pgfpathlineto{\pgfqpoint{2.026793in}{4.587297in}}%
\pgfpathlineto{\pgfqpoint{1.053750in}{4.452924in}}%
\pgfpathlineto{\pgfqpoint{-3231.325327in}{2.162824in}}%
\pgfpathclose%
\pgfusepath{fill}%
\end{pgfscope}%
\begin{pgfscope}%
\pgfsetbuttcap%
\pgfsetroundjoin%
\definecolor{currentfill}{rgb}{0.000000,0.000000,0.000000}%
\pgfsetfillcolor{currentfill}%
\pgfsetlinewidth{0.803000pt}%
\definecolor{currentstroke}{rgb}{0.000000,0.000000,0.000000}%
\pgfsetstrokecolor{currentstroke}%
\pgfsetdash{}{0pt}%
\pgfsys@defobject{currentmarker}{\pgfqpoint{0.000000in}{-0.048611in}}{\pgfqpoint{0.000000in}{0.000000in}}{%
\pgfpathmoveto{\pgfqpoint{0.000000in}{0.000000in}}%
\pgfpathlineto{\pgfqpoint{0.000000in}{-0.048611in}}%
\pgfusepath{stroke,fill}%
}%
\begin{pgfscope}%
\pgfsys@transformshift{1.053750in}{2.022900in}%
\pgfsys@useobject{currentmarker}{}%
\end{pgfscope}%
\end{pgfscope}%
\begin{pgfscope}%
\definecolor{textcolor}{rgb}{0.000000,0.000000,0.000000}%
\pgfsetstrokecolor{textcolor}%
\pgfsetfillcolor{textcolor}%
\pgftext[x=1.053750in,y=1.925678in,,top]{\color{textcolor}\sffamily\fontsize{12.000000}{14.400000}\selectfont \(\displaystyle {10^{0}}\)}%
\end{pgfscope}%
\begin{pgfscope}%
\pgfsetbuttcap%
\pgfsetroundjoin%
\definecolor{currentfill}{rgb}{0.000000,0.000000,0.000000}%
\pgfsetfillcolor{currentfill}%
\pgfsetlinewidth{0.803000pt}%
\definecolor{currentstroke}{rgb}{0.000000,0.000000,0.000000}%
\pgfsetstrokecolor{currentstroke}%
\pgfsetdash{}{0pt}%
\pgfsys@defobject{currentmarker}{\pgfqpoint{0.000000in}{-0.048611in}}{\pgfqpoint{0.000000in}{0.000000in}}{%
\pgfpathmoveto{\pgfqpoint{0.000000in}{0.000000in}}%
\pgfpathlineto{\pgfqpoint{0.000000in}{-0.048611in}}%
\pgfusepath{stroke,fill}%
}%
\begin{pgfscope}%
\pgfsys@transformshift{4.286129in}{2.022900in}%
\pgfsys@useobject{currentmarker}{}%
\end{pgfscope}%
\end{pgfscope}%
\begin{pgfscope}%
\definecolor{textcolor}{rgb}{0.000000,0.000000,0.000000}%
\pgfsetstrokecolor{textcolor}%
\pgfsetfillcolor{textcolor}%
\pgftext[x=4.286129in,y=1.925678in,,top]{\color{textcolor}\sffamily\fontsize{12.000000}{14.400000}\selectfont \(\displaystyle {10^{1}}\)}%
\end{pgfscope}%
\begin{pgfscope}%
\pgfsetbuttcap%
\pgfsetroundjoin%
\definecolor{currentfill}{rgb}{0.000000,0.000000,0.000000}%
\pgfsetfillcolor{currentfill}%
\pgfsetlinewidth{0.803000pt}%
\definecolor{currentstroke}{rgb}{0.000000,0.000000,0.000000}%
\pgfsetstrokecolor{currentstroke}%
\pgfsetdash{}{0pt}%
\pgfsys@defobject{currentmarker}{\pgfqpoint{0.000000in}{-0.048611in}}{\pgfqpoint{0.000000in}{0.000000in}}{%
\pgfpathmoveto{\pgfqpoint{0.000000in}{0.000000in}}%
\pgfpathlineto{\pgfqpoint{0.000000in}{-0.048611in}}%
\pgfusepath{stroke,fill}%
}%
\begin{pgfscope}%
\pgfsys@transformshift{7.518508in}{2.022900in}%
\pgfsys@useobject{currentmarker}{}%
\end{pgfscope}%
\end{pgfscope}%
\begin{pgfscope}%
\definecolor{textcolor}{rgb}{0.000000,0.000000,0.000000}%
\pgfsetstrokecolor{textcolor}%
\pgfsetfillcolor{textcolor}%
\pgftext[x=7.518508in,y=1.925678in,,top]{\color{textcolor}\sffamily\fontsize{12.000000}{14.400000}\selectfont \(\displaystyle {10^{2}}\)}%
\end{pgfscope}%
\begin{pgfscope}%
\pgfsetbuttcap%
\pgfsetroundjoin%
\definecolor{currentfill}{rgb}{0.000000,0.000000,0.000000}%
\pgfsetfillcolor{currentfill}%
\pgfsetlinewidth{0.602250pt}%
\definecolor{currentstroke}{rgb}{0.000000,0.000000,0.000000}%
\pgfsetstrokecolor{currentstroke}%
\pgfsetdash{}{0pt}%
\pgfsys@defobject{currentmarker}{\pgfqpoint{0.000000in}{-0.027778in}}{\pgfqpoint{0.000000in}{0.000000in}}{%
\pgfpathmoveto{\pgfqpoint{0.000000in}{0.000000in}}%
\pgfpathlineto{\pgfqpoint{0.000000in}{-0.027778in}}%
\pgfusepath{stroke,fill}%
}%
\begin{pgfscope}%
\pgfsys@transformshift{2.026793in}{2.022900in}%
\pgfsys@useobject{currentmarker}{}%
\end{pgfscope}%
\end{pgfscope}%
\begin{pgfscope}%
\pgfsetbuttcap%
\pgfsetroundjoin%
\definecolor{currentfill}{rgb}{0.000000,0.000000,0.000000}%
\pgfsetfillcolor{currentfill}%
\pgfsetlinewidth{0.602250pt}%
\definecolor{currentstroke}{rgb}{0.000000,0.000000,0.000000}%
\pgfsetstrokecolor{currentstroke}%
\pgfsetdash{}{0pt}%
\pgfsys@defobject{currentmarker}{\pgfqpoint{0.000000in}{-0.027778in}}{\pgfqpoint{0.000000in}{0.000000in}}{%
\pgfpathmoveto{\pgfqpoint{0.000000in}{0.000000in}}%
\pgfpathlineto{\pgfqpoint{0.000000in}{-0.027778in}}%
\pgfusepath{stroke,fill}%
}%
\begin{pgfscope}%
\pgfsys@transformshift{2.595987in}{2.022900in}%
\pgfsys@useobject{currentmarker}{}%
\end{pgfscope}%
\end{pgfscope}%
\begin{pgfscope}%
\pgfsetbuttcap%
\pgfsetroundjoin%
\definecolor{currentfill}{rgb}{0.000000,0.000000,0.000000}%
\pgfsetfillcolor{currentfill}%
\pgfsetlinewidth{0.602250pt}%
\definecolor{currentstroke}{rgb}{0.000000,0.000000,0.000000}%
\pgfsetstrokecolor{currentstroke}%
\pgfsetdash{}{0pt}%
\pgfsys@defobject{currentmarker}{\pgfqpoint{0.000000in}{-0.027778in}}{\pgfqpoint{0.000000in}{0.000000in}}{%
\pgfpathmoveto{\pgfqpoint{0.000000in}{0.000000in}}%
\pgfpathlineto{\pgfqpoint{0.000000in}{-0.027778in}}%
\pgfusepath{stroke,fill}%
}%
\begin{pgfscope}%
\pgfsys@transformshift{2.999836in}{2.022900in}%
\pgfsys@useobject{currentmarker}{}%
\end{pgfscope}%
\end{pgfscope}%
\begin{pgfscope}%
\pgfsetbuttcap%
\pgfsetroundjoin%
\definecolor{currentfill}{rgb}{0.000000,0.000000,0.000000}%
\pgfsetfillcolor{currentfill}%
\pgfsetlinewidth{0.602250pt}%
\definecolor{currentstroke}{rgb}{0.000000,0.000000,0.000000}%
\pgfsetstrokecolor{currentstroke}%
\pgfsetdash{}{0pt}%
\pgfsys@defobject{currentmarker}{\pgfqpoint{0.000000in}{-0.027778in}}{\pgfqpoint{0.000000in}{0.000000in}}{%
\pgfpathmoveto{\pgfqpoint{0.000000in}{0.000000in}}%
\pgfpathlineto{\pgfqpoint{0.000000in}{-0.027778in}}%
\pgfusepath{stroke,fill}%
}%
\begin{pgfscope}%
\pgfsys@transformshift{3.313086in}{2.022900in}%
\pgfsys@useobject{currentmarker}{}%
\end{pgfscope}%
\end{pgfscope}%
\begin{pgfscope}%
\pgfsetbuttcap%
\pgfsetroundjoin%
\definecolor{currentfill}{rgb}{0.000000,0.000000,0.000000}%
\pgfsetfillcolor{currentfill}%
\pgfsetlinewidth{0.602250pt}%
\definecolor{currentstroke}{rgb}{0.000000,0.000000,0.000000}%
\pgfsetstrokecolor{currentstroke}%
\pgfsetdash{}{0pt}%
\pgfsys@defobject{currentmarker}{\pgfqpoint{0.000000in}{-0.027778in}}{\pgfqpoint{0.000000in}{0.000000in}}{%
\pgfpathmoveto{\pgfqpoint{0.000000in}{0.000000in}}%
\pgfpathlineto{\pgfqpoint{0.000000in}{-0.027778in}}%
\pgfusepath{stroke,fill}%
}%
\begin{pgfscope}%
\pgfsys@transformshift{3.569030in}{2.022900in}%
\pgfsys@useobject{currentmarker}{}%
\end{pgfscope}%
\end{pgfscope}%
\begin{pgfscope}%
\pgfsetbuttcap%
\pgfsetroundjoin%
\definecolor{currentfill}{rgb}{0.000000,0.000000,0.000000}%
\pgfsetfillcolor{currentfill}%
\pgfsetlinewidth{0.602250pt}%
\definecolor{currentstroke}{rgb}{0.000000,0.000000,0.000000}%
\pgfsetstrokecolor{currentstroke}%
\pgfsetdash{}{0pt}%
\pgfsys@defobject{currentmarker}{\pgfqpoint{0.000000in}{-0.027778in}}{\pgfqpoint{0.000000in}{0.000000in}}{%
\pgfpathmoveto{\pgfqpoint{0.000000in}{0.000000in}}%
\pgfpathlineto{\pgfqpoint{0.000000in}{-0.027778in}}%
\pgfusepath{stroke,fill}%
}%
\begin{pgfscope}%
\pgfsys@transformshift{3.785427in}{2.022900in}%
\pgfsys@useobject{currentmarker}{}%
\end{pgfscope}%
\end{pgfscope}%
\begin{pgfscope}%
\pgfsetbuttcap%
\pgfsetroundjoin%
\definecolor{currentfill}{rgb}{0.000000,0.000000,0.000000}%
\pgfsetfillcolor{currentfill}%
\pgfsetlinewidth{0.602250pt}%
\definecolor{currentstroke}{rgb}{0.000000,0.000000,0.000000}%
\pgfsetstrokecolor{currentstroke}%
\pgfsetdash{}{0pt}%
\pgfsys@defobject{currentmarker}{\pgfqpoint{0.000000in}{-0.027778in}}{\pgfqpoint{0.000000in}{0.000000in}}{%
\pgfpathmoveto{\pgfqpoint{0.000000in}{0.000000in}}%
\pgfpathlineto{\pgfqpoint{0.000000in}{-0.027778in}}%
\pgfusepath{stroke,fill}%
}%
\begin{pgfscope}%
\pgfsys@transformshift{3.972879in}{2.022900in}%
\pgfsys@useobject{currentmarker}{}%
\end{pgfscope}%
\end{pgfscope}%
\begin{pgfscope}%
\pgfsetbuttcap%
\pgfsetroundjoin%
\definecolor{currentfill}{rgb}{0.000000,0.000000,0.000000}%
\pgfsetfillcolor{currentfill}%
\pgfsetlinewidth{0.602250pt}%
\definecolor{currentstroke}{rgb}{0.000000,0.000000,0.000000}%
\pgfsetstrokecolor{currentstroke}%
\pgfsetdash{}{0pt}%
\pgfsys@defobject{currentmarker}{\pgfqpoint{0.000000in}{-0.027778in}}{\pgfqpoint{0.000000in}{0.000000in}}{%
\pgfpathmoveto{\pgfqpoint{0.000000in}{0.000000in}}%
\pgfpathlineto{\pgfqpoint{0.000000in}{-0.027778in}}%
\pgfusepath{stroke,fill}%
}%
\begin{pgfscope}%
\pgfsys@transformshift{4.138224in}{2.022900in}%
\pgfsys@useobject{currentmarker}{}%
\end{pgfscope}%
\end{pgfscope}%
\begin{pgfscope}%
\pgfsetbuttcap%
\pgfsetroundjoin%
\definecolor{currentfill}{rgb}{0.000000,0.000000,0.000000}%
\pgfsetfillcolor{currentfill}%
\pgfsetlinewidth{0.602250pt}%
\definecolor{currentstroke}{rgb}{0.000000,0.000000,0.000000}%
\pgfsetstrokecolor{currentstroke}%
\pgfsetdash{}{0pt}%
\pgfsys@defobject{currentmarker}{\pgfqpoint{0.000000in}{-0.027778in}}{\pgfqpoint{0.000000in}{0.000000in}}{%
\pgfpathmoveto{\pgfqpoint{0.000000in}{0.000000in}}%
\pgfpathlineto{\pgfqpoint{0.000000in}{-0.027778in}}%
\pgfusepath{stroke,fill}%
}%
\begin{pgfscope}%
\pgfsys@transformshift{5.259172in}{2.022900in}%
\pgfsys@useobject{currentmarker}{}%
\end{pgfscope}%
\end{pgfscope}%
\begin{pgfscope}%
\pgfsetbuttcap%
\pgfsetroundjoin%
\definecolor{currentfill}{rgb}{0.000000,0.000000,0.000000}%
\pgfsetfillcolor{currentfill}%
\pgfsetlinewidth{0.602250pt}%
\definecolor{currentstroke}{rgb}{0.000000,0.000000,0.000000}%
\pgfsetstrokecolor{currentstroke}%
\pgfsetdash{}{0pt}%
\pgfsys@defobject{currentmarker}{\pgfqpoint{0.000000in}{-0.027778in}}{\pgfqpoint{0.000000in}{0.000000in}}{%
\pgfpathmoveto{\pgfqpoint{0.000000in}{0.000000in}}%
\pgfpathlineto{\pgfqpoint{0.000000in}{-0.027778in}}%
\pgfusepath{stroke,fill}%
}%
\begin{pgfscope}%
\pgfsys@transformshift{5.828366in}{2.022900in}%
\pgfsys@useobject{currentmarker}{}%
\end{pgfscope}%
\end{pgfscope}%
\begin{pgfscope}%
\pgfsetbuttcap%
\pgfsetroundjoin%
\definecolor{currentfill}{rgb}{0.000000,0.000000,0.000000}%
\pgfsetfillcolor{currentfill}%
\pgfsetlinewidth{0.602250pt}%
\definecolor{currentstroke}{rgb}{0.000000,0.000000,0.000000}%
\pgfsetstrokecolor{currentstroke}%
\pgfsetdash{}{0pt}%
\pgfsys@defobject{currentmarker}{\pgfqpoint{0.000000in}{-0.027778in}}{\pgfqpoint{0.000000in}{0.000000in}}{%
\pgfpathmoveto{\pgfqpoint{0.000000in}{0.000000in}}%
\pgfpathlineto{\pgfqpoint{0.000000in}{-0.027778in}}%
\pgfusepath{stroke,fill}%
}%
\begin{pgfscope}%
\pgfsys@transformshift{6.232215in}{2.022900in}%
\pgfsys@useobject{currentmarker}{}%
\end{pgfscope}%
\end{pgfscope}%
\begin{pgfscope}%
\pgfsetbuttcap%
\pgfsetroundjoin%
\definecolor{currentfill}{rgb}{0.000000,0.000000,0.000000}%
\pgfsetfillcolor{currentfill}%
\pgfsetlinewidth{0.602250pt}%
\definecolor{currentstroke}{rgb}{0.000000,0.000000,0.000000}%
\pgfsetstrokecolor{currentstroke}%
\pgfsetdash{}{0pt}%
\pgfsys@defobject{currentmarker}{\pgfqpoint{0.000000in}{-0.027778in}}{\pgfqpoint{0.000000in}{0.000000in}}{%
\pgfpathmoveto{\pgfqpoint{0.000000in}{0.000000in}}%
\pgfpathlineto{\pgfqpoint{0.000000in}{-0.027778in}}%
\pgfusepath{stroke,fill}%
}%
\begin{pgfscope}%
\pgfsys@transformshift{6.545465in}{2.022900in}%
\pgfsys@useobject{currentmarker}{}%
\end{pgfscope}%
\end{pgfscope}%
\begin{pgfscope}%
\pgfsetbuttcap%
\pgfsetroundjoin%
\definecolor{currentfill}{rgb}{0.000000,0.000000,0.000000}%
\pgfsetfillcolor{currentfill}%
\pgfsetlinewidth{0.602250pt}%
\definecolor{currentstroke}{rgb}{0.000000,0.000000,0.000000}%
\pgfsetstrokecolor{currentstroke}%
\pgfsetdash{}{0pt}%
\pgfsys@defobject{currentmarker}{\pgfqpoint{0.000000in}{-0.027778in}}{\pgfqpoint{0.000000in}{0.000000in}}{%
\pgfpathmoveto{\pgfqpoint{0.000000in}{0.000000in}}%
\pgfpathlineto{\pgfqpoint{0.000000in}{-0.027778in}}%
\pgfusepath{stroke,fill}%
}%
\begin{pgfscope}%
\pgfsys@transformshift{6.801409in}{2.022900in}%
\pgfsys@useobject{currentmarker}{}%
\end{pgfscope}%
\end{pgfscope}%
\begin{pgfscope}%
\pgfsetbuttcap%
\pgfsetroundjoin%
\definecolor{currentfill}{rgb}{0.000000,0.000000,0.000000}%
\pgfsetfillcolor{currentfill}%
\pgfsetlinewidth{0.602250pt}%
\definecolor{currentstroke}{rgb}{0.000000,0.000000,0.000000}%
\pgfsetstrokecolor{currentstroke}%
\pgfsetdash{}{0pt}%
\pgfsys@defobject{currentmarker}{\pgfqpoint{0.000000in}{-0.027778in}}{\pgfqpoint{0.000000in}{0.000000in}}{%
\pgfpathmoveto{\pgfqpoint{0.000000in}{0.000000in}}%
\pgfpathlineto{\pgfqpoint{0.000000in}{-0.027778in}}%
\pgfusepath{stroke,fill}%
}%
\begin{pgfscope}%
\pgfsys@transformshift{7.017806in}{2.022900in}%
\pgfsys@useobject{currentmarker}{}%
\end{pgfscope}%
\end{pgfscope}%
\begin{pgfscope}%
\pgfsetbuttcap%
\pgfsetroundjoin%
\definecolor{currentfill}{rgb}{0.000000,0.000000,0.000000}%
\pgfsetfillcolor{currentfill}%
\pgfsetlinewidth{0.602250pt}%
\definecolor{currentstroke}{rgb}{0.000000,0.000000,0.000000}%
\pgfsetstrokecolor{currentstroke}%
\pgfsetdash{}{0pt}%
\pgfsys@defobject{currentmarker}{\pgfqpoint{0.000000in}{-0.027778in}}{\pgfqpoint{0.000000in}{0.000000in}}{%
\pgfpathmoveto{\pgfqpoint{0.000000in}{0.000000in}}%
\pgfpathlineto{\pgfqpoint{0.000000in}{-0.027778in}}%
\pgfusepath{stroke,fill}%
}%
\begin{pgfscope}%
\pgfsys@transformshift{7.205258in}{2.022900in}%
\pgfsys@useobject{currentmarker}{}%
\end{pgfscope}%
\end{pgfscope}%
\begin{pgfscope}%
\pgfsetbuttcap%
\pgfsetroundjoin%
\definecolor{currentfill}{rgb}{0.000000,0.000000,0.000000}%
\pgfsetfillcolor{currentfill}%
\pgfsetlinewidth{0.602250pt}%
\definecolor{currentstroke}{rgb}{0.000000,0.000000,0.000000}%
\pgfsetstrokecolor{currentstroke}%
\pgfsetdash{}{0pt}%
\pgfsys@defobject{currentmarker}{\pgfqpoint{0.000000in}{-0.027778in}}{\pgfqpoint{0.000000in}{0.000000in}}{%
\pgfpathmoveto{\pgfqpoint{0.000000in}{0.000000in}}%
\pgfpathlineto{\pgfqpoint{0.000000in}{-0.027778in}}%
\pgfusepath{stroke,fill}%
}%
\begin{pgfscope}%
\pgfsys@transformshift{7.370603in}{2.022900in}%
\pgfsys@useobject{currentmarker}{}%
\end{pgfscope}%
\end{pgfscope}%
\begin{pgfscope}%
\definecolor{textcolor}{rgb}{0.000000,0.000000,0.000000}%
\pgfsetstrokecolor{textcolor}%
\pgfsetfillcolor{textcolor}%
\pgftext[x=4.320375in,y=1.708827in,,top]{\color{textcolor}\sffamily\fontsize{14.000000}{16.800000}\selectfont epochs}%
\end{pgfscope}%
\begin{pgfscope}%
\pgfsetbuttcap%
\pgfsetroundjoin%
\definecolor{currentfill}{rgb}{0.000000,0.000000,0.000000}%
\pgfsetfillcolor{currentfill}%
\pgfsetlinewidth{0.803000pt}%
\definecolor{currentstroke}{rgb}{0.000000,0.000000,0.000000}%
\pgfsetstrokecolor{currentstroke}%
\pgfsetdash{}{0pt}%
\pgfsys@defobject{currentmarker}{\pgfqpoint{-0.048611in}{0.000000in}}{\pgfqpoint{0.000000in}{0.000000in}}{%
\pgfpathmoveto{\pgfqpoint{0.000000in}{0.000000in}}%
\pgfpathlineto{\pgfqpoint{-0.048611in}{0.000000in}}%
\pgfusepath{stroke,fill}%
}%
\begin{pgfscope}%
\pgfsys@transformshift{1.053750in}{2.495813in}%
\pgfsys@useobject{currentmarker}{}%
\end{pgfscope}%
\end{pgfscope}%
\begin{pgfscope}%
\definecolor{textcolor}{rgb}{0.000000,0.000000,0.000000}%
\pgfsetstrokecolor{textcolor}%
\pgfsetfillcolor{textcolor}%
\pgftext[x=0.691472in,y=2.432499in,left,base]{\color{textcolor}\sffamily\fontsize{12.000000}{14.400000}\selectfont 0.2}%
\end{pgfscope}%
\begin{pgfscope}%
\pgfsetbuttcap%
\pgfsetroundjoin%
\definecolor{currentfill}{rgb}{0.000000,0.000000,0.000000}%
\pgfsetfillcolor{currentfill}%
\pgfsetlinewidth{0.803000pt}%
\definecolor{currentstroke}{rgb}{0.000000,0.000000,0.000000}%
\pgfsetstrokecolor{currentstroke}%
\pgfsetdash{}{0pt}%
\pgfsys@defobject{currentmarker}{\pgfqpoint{-0.048611in}{0.000000in}}{\pgfqpoint{0.000000in}{0.000000in}}{%
\pgfpathmoveto{\pgfqpoint{0.000000in}{0.000000in}}%
\pgfpathlineto{\pgfqpoint{-0.048611in}{0.000000in}}%
\pgfusepath{stroke,fill}%
}%
\begin{pgfscope}%
\pgfsys@transformshift{1.053750in}{3.134661in}%
\pgfsys@useobject{currentmarker}{}%
\end{pgfscope}%
\end{pgfscope}%
\begin{pgfscope}%
\definecolor{textcolor}{rgb}{0.000000,0.000000,0.000000}%
\pgfsetstrokecolor{textcolor}%
\pgfsetfillcolor{textcolor}%
\pgftext[x=0.691472in,y=3.071347in,left,base]{\color{textcolor}\sffamily\fontsize{12.000000}{14.400000}\selectfont 0.4}%
\end{pgfscope}%
\begin{pgfscope}%
\pgfsetbuttcap%
\pgfsetroundjoin%
\definecolor{currentfill}{rgb}{0.000000,0.000000,0.000000}%
\pgfsetfillcolor{currentfill}%
\pgfsetlinewidth{0.803000pt}%
\definecolor{currentstroke}{rgb}{0.000000,0.000000,0.000000}%
\pgfsetstrokecolor{currentstroke}%
\pgfsetdash{}{0pt}%
\pgfsys@defobject{currentmarker}{\pgfqpoint{-0.048611in}{0.000000in}}{\pgfqpoint{0.000000in}{0.000000in}}{%
\pgfpathmoveto{\pgfqpoint{0.000000in}{0.000000in}}%
\pgfpathlineto{\pgfqpoint{-0.048611in}{0.000000in}}%
\pgfusepath{stroke,fill}%
}%
\begin{pgfscope}%
\pgfsys@transformshift{1.053750in}{3.773509in}%
\pgfsys@useobject{currentmarker}{}%
\end{pgfscope}%
\end{pgfscope}%
\begin{pgfscope}%
\definecolor{textcolor}{rgb}{0.000000,0.000000,0.000000}%
\pgfsetstrokecolor{textcolor}%
\pgfsetfillcolor{textcolor}%
\pgftext[x=0.691472in,y=3.710195in,left,base]{\color{textcolor}\sffamily\fontsize{12.000000}{14.400000}\selectfont 0.6}%
\end{pgfscope}%
\begin{pgfscope}%
\pgfsetbuttcap%
\pgfsetroundjoin%
\definecolor{currentfill}{rgb}{0.000000,0.000000,0.000000}%
\pgfsetfillcolor{currentfill}%
\pgfsetlinewidth{0.803000pt}%
\definecolor{currentstroke}{rgb}{0.000000,0.000000,0.000000}%
\pgfsetstrokecolor{currentstroke}%
\pgfsetdash{}{0pt}%
\pgfsys@defobject{currentmarker}{\pgfqpoint{-0.048611in}{0.000000in}}{\pgfqpoint{0.000000in}{0.000000in}}{%
\pgfpathmoveto{\pgfqpoint{0.000000in}{0.000000in}}%
\pgfpathlineto{\pgfqpoint{-0.048611in}{0.000000in}}%
\pgfusepath{stroke,fill}%
}%
\begin{pgfscope}%
\pgfsys@transformshift{1.053750in}{4.412356in}%
\pgfsys@useobject{currentmarker}{}%
\end{pgfscope}%
\end{pgfscope}%
\begin{pgfscope}%
\definecolor{textcolor}{rgb}{0.000000,0.000000,0.000000}%
\pgfsetstrokecolor{textcolor}%
\pgfsetfillcolor{textcolor}%
\pgftext[x=0.691472in,y=4.349043in,left,base]{\color{textcolor}\sffamily\fontsize{12.000000}{14.400000}\selectfont 0.8}%
\end{pgfscope}%
\begin{pgfscope}%
\pgfsetbuttcap%
\pgfsetroundjoin%
\definecolor{currentfill}{rgb}{0.000000,0.000000,0.000000}%
\pgfsetfillcolor{currentfill}%
\pgfsetlinewidth{0.803000pt}%
\definecolor{currentstroke}{rgb}{0.000000,0.000000,0.000000}%
\pgfsetstrokecolor{currentstroke}%
\pgfsetdash{}{0pt}%
\pgfsys@defobject{currentmarker}{\pgfqpoint{-0.048611in}{0.000000in}}{\pgfqpoint{0.000000in}{0.000000in}}{%
\pgfpathmoveto{\pgfqpoint{0.000000in}{0.000000in}}%
\pgfpathlineto{\pgfqpoint{-0.048611in}{0.000000in}}%
\pgfusepath{stroke,fill}%
}%
\begin{pgfscope}%
\pgfsys@transformshift{1.053750in}{5.051204in}%
\pgfsys@useobject{currentmarker}{}%
\end{pgfscope}%
\end{pgfscope}%
\begin{pgfscope}%
\definecolor{textcolor}{rgb}{0.000000,0.000000,0.000000}%
\pgfsetstrokecolor{textcolor}%
\pgfsetfillcolor{textcolor}%
\pgftext[x=0.691472in,y=4.987890in,left,base]{\color{textcolor}\sffamily\fontsize{12.000000}{14.400000}\selectfont 1.0}%
\end{pgfscope}%
\begin{pgfscope}%
\definecolor{textcolor}{rgb}{0.000000,0.000000,0.000000}%
\pgfsetstrokecolor{textcolor}%
\pgfsetfillcolor{textcolor}%
\pgftext[x=0.635917in,y=3.562063in,,bottom,rotate=90.000000]{\color{textcolor}\sffamily\fontsize{14.000000}{16.800000}\selectfont train\_accuracies}%
\end{pgfscope}%
\begin{pgfscope}%
\pgfpathrectangle{\pgfqpoint{1.053750in}{2.022900in}}{\pgfqpoint{6.533250in}{3.078326in}}%
\pgfusepath{clip}%
\pgfsetrectcap%
\pgfsetroundjoin%
\pgfsetlinewidth{1.505625pt}%
\definecolor{currentstroke}{rgb}{0.121569,0.466667,0.705882}%
\pgfsetstrokecolor{currentstroke}%
\pgfsetdash{}{0pt}%
\pgfpathmoveto{\pgfqpoint{1.043750in}{4.476272in}}%
\pgfpathlineto{\pgfqpoint{1.053750in}{4.476280in}}%
\pgfpathlineto{\pgfqpoint{2.026793in}{4.576739in}}%
\pgfpathlineto{\pgfqpoint{2.595987in}{4.675600in}}%
\pgfpathlineto{\pgfqpoint{2.999836in}{4.636887in}}%
\pgfpathlineto{\pgfqpoint{3.313086in}{4.691596in}}%
\pgfpathlineto{\pgfqpoint{3.569030in}{4.692876in}}%
\pgfpathlineto{\pgfqpoint{3.785427in}{4.706633in}}%
\pgfpathlineto{\pgfqpoint{3.972879in}{4.719431in}}%
\pgfpathlineto{\pgfqpoint{4.138224in}{4.715911in}}%
\pgfpathlineto{\pgfqpoint{4.286129in}{4.725190in}}%
\pgfpathlineto{\pgfqpoint{4.419926in}{4.729029in}}%
\pgfpathlineto{\pgfqpoint{4.542073in}{4.747265in}}%
\pgfpathlineto{\pgfqpoint{4.654437in}{4.753344in}}%
\pgfpathlineto{\pgfqpoint{4.758470in}{4.733188in}}%
\pgfpathlineto{\pgfqpoint{4.855323in}{4.761662in}}%
\pgfpathlineto{\pgfqpoint{4.945922in}{4.742786in}}%
\pgfpathlineto{\pgfqpoint{5.031027in}{4.755264in}}%
\pgfpathlineto{\pgfqpoint{5.111267in}{4.761342in}}%
\pgfpathlineto{\pgfqpoint{5.187166in}{4.764222in}}%
\pgfpathlineto{\pgfqpoint{5.259172in}{4.784378in}}%
\pgfpathlineto{\pgfqpoint{5.327664in}{4.763902in}}%
\pgfpathlineto{\pgfqpoint{5.392969in}{4.773820in}}%
\pgfpathlineto{\pgfqpoint{5.455371in}{4.755903in}}%
\pgfpathlineto{\pgfqpoint{5.515116in}{4.772220in}}%
\pgfpathlineto{\pgfqpoint{5.572422in}{4.796535in}}%
\pgfpathlineto{\pgfqpoint{5.627480in}{4.779899in}}%
\pgfpathlineto{\pgfqpoint{5.680460in}{4.778299in}}%
\pgfpathlineto{\pgfqpoint{5.731513in}{4.792376in}}%
\pgfpathlineto{\pgfqpoint{5.780775in}{4.783418in}}%
\pgfpathlineto{\pgfqpoint{5.828366in}{4.770301in}}%
\pgfpathlineto{\pgfqpoint{5.874396in}{4.789177in}}%
\pgfpathlineto{\pgfqpoint{5.918965in}{4.800055in}}%
\pgfpathlineto{\pgfqpoint{5.962163in}{4.784378in}}%
\pgfpathlineto{\pgfqpoint{6.004070in}{4.798135in}}%
\pgfpathlineto{\pgfqpoint{6.044763in}{4.802294in}}%
\pgfpathlineto{\pgfqpoint{6.084310in}{4.801654in}}%
\pgfpathlineto{\pgfqpoint{6.122772in}{4.804854in}}%
\pgfpathlineto{\pgfqpoint{6.160209in}{4.800694in}}%
\pgfpathlineto{\pgfqpoint{6.196674in}{4.809333in}}%
\pgfpathlineto{\pgfqpoint{6.232215in}{4.801014in}}%
\pgfpathlineto{\pgfqpoint{6.266879in}{4.800055in}}%
\pgfpathlineto{\pgfqpoint{6.300707in}{4.791096in}}%
\pgfpathlineto{\pgfqpoint{6.333739in}{4.816371in}}%
\pgfpathlineto{\pgfqpoint{6.366012in}{4.817651in}}%
\pgfpathlineto{\pgfqpoint{6.397560in}{4.810612in}}%
\pgfpathlineto{\pgfqpoint{6.428414in}{4.810612in}}%
\pgfpathlineto{\pgfqpoint{6.458604in}{4.816371in}}%
\pgfpathlineto{\pgfqpoint{6.488159in}{4.829809in}}%
\pgfpathlineto{\pgfqpoint{6.517104in}{4.804534in}}%
\pgfpathlineto{\pgfqpoint{6.545465in}{4.809013in}}%
\pgfpathlineto{\pgfqpoint{6.573264in}{4.808373in}}%
\pgfpathlineto{\pgfqpoint{6.600523in}{4.826289in}}%
\pgfpathlineto{\pgfqpoint{6.627263in}{4.823730in}}%
\pgfpathlineto{\pgfqpoint{6.653503in}{4.837807in}}%
\pgfpathlineto{\pgfqpoint{6.679262in}{4.828849in}}%
\pgfpathlineto{\pgfqpoint{6.704556in}{4.814452in}}%
\pgfpathlineto{\pgfqpoint{6.729403in}{4.831408in}}%
\pgfpathlineto{\pgfqpoint{6.753818in}{4.834608in}}%
\pgfpathlineto{\pgfqpoint{6.777815in}{4.840686in}}%
\pgfpathlineto{\pgfqpoint{6.801409in}{4.837167in}}%
\pgfpathlineto{\pgfqpoint{6.824613in}{4.821170in}}%
\pgfpathlineto{\pgfqpoint{6.847439in}{4.845485in}}%
\pgfpathlineto{\pgfqpoint{6.869901in}{4.834928in}}%
\pgfpathlineto{\pgfqpoint{6.892008in}{4.825329in}}%
\pgfpathlineto{\pgfqpoint{6.913773in}{4.830448in}}%
\pgfpathlineto{\pgfqpoint{6.935206in}{4.837487in}}%
\pgfpathlineto{\pgfqpoint{6.956316in}{4.838767in}}%
\pgfpathlineto{\pgfqpoint{6.977113in}{4.844206in}}%
\pgfpathlineto{\pgfqpoint{6.997607in}{4.831088in}}%
\pgfpathlineto{\pgfqpoint{7.017806in}{4.844846in}}%
\pgfpathlineto{\pgfqpoint{7.037719in}{4.848045in}}%
\pgfpathlineto{\pgfqpoint{7.057353in}{4.829809in}}%
\pgfpathlineto{\pgfqpoint{7.076716in}{4.830129in}}%
\pgfpathlineto{\pgfqpoint{7.095816in}{4.847725in}}%
\pgfpathlineto{\pgfqpoint{7.114659in}{4.839727in}}%
\pgfpathlineto{\pgfqpoint{7.133253in}{4.822450in}}%
\pgfpathlineto{\pgfqpoint{7.151603in}{4.853164in}}%
\pgfpathlineto{\pgfqpoint{7.169717in}{4.813812in}}%
\pgfpathlineto{\pgfqpoint{7.187600in}{4.853484in}}%
\pgfpathlineto{\pgfqpoint{7.205258in}{4.851884in}}%
\pgfpathlineto{\pgfqpoint{7.222697in}{4.841006in}}%
\pgfpathlineto{\pgfqpoint{7.239922in}{4.846765in}}%
\pgfpathlineto{\pgfqpoint{7.256938in}{4.851564in}}%
\pgfpathlineto{\pgfqpoint{7.273750in}{4.853484in}}%
\pgfpathlineto{\pgfqpoint{7.290363in}{4.860842in}}%
\pgfpathlineto{\pgfqpoint{7.306782in}{4.852204in}}%
\pgfpathlineto{\pgfqpoint{7.323011in}{4.859883in}}%
\pgfpathlineto{\pgfqpoint{7.339055in}{4.867561in}}%
\pgfpathlineto{\pgfqpoint{7.354917in}{4.847085in}}%
\pgfpathlineto{\pgfqpoint{7.370603in}{4.858283in}}%
\pgfpathlineto{\pgfqpoint{7.386114in}{4.850924in}}%
\pgfpathlineto{\pgfqpoint{7.401457in}{4.857003in}}%
\pgfpathlineto{\pgfqpoint{7.416633in}{4.860842in}}%
\pgfpathlineto{\pgfqpoint{7.431647in}{4.863722in}}%
\pgfpathlineto{\pgfqpoint{7.446502in}{4.854124in}}%
\pgfpathlineto{\pgfqpoint{7.461202in}{4.853484in}}%
\pgfpathlineto{\pgfqpoint{7.475749in}{4.874280in}}%
\pgfpathlineto{\pgfqpoint{7.490148in}{4.850604in}}%
\pgfpathlineto{\pgfqpoint{7.504399in}{4.867881in}}%
\pgfpathlineto{\pgfqpoint{7.518508in}{4.872040in}}%
\pgfusepath{stroke}%
\end{pgfscope}%
\begin{pgfscope}%
\pgfpathrectangle{\pgfqpoint{1.053750in}{2.022900in}}{\pgfqpoint{6.533250in}{3.078326in}}%
\pgfusepath{clip}%
\pgfsetrectcap%
\pgfsetroundjoin%
\pgfsetlinewidth{1.505625pt}%
\definecolor{currentstroke}{rgb}{1.000000,0.498039,0.054902}%
\pgfsetstrokecolor{currentstroke}%
\pgfsetdash{}{0pt}%
\pgfpathmoveto{\pgfqpoint{1.043750in}{4.452917in}}%
\pgfpathlineto{\pgfqpoint{1.053750in}{4.452924in}}%
\pgfpathlineto{\pgfqpoint{2.026793in}{4.587297in}}%
\pgfpathlineto{\pgfqpoint{2.595987in}{4.674640in}}%
\pgfpathlineto{\pgfqpoint{2.999836in}{4.615132in}}%
\pgfpathlineto{\pgfqpoint{3.313086in}{4.684238in}}%
\pgfpathlineto{\pgfqpoint{3.569030in}{4.692876in}}%
\pgfpathlineto{\pgfqpoint{3.785427in}{4.715272in}}%
\pgfpathlineto{\pgfqpoint{3.972879in}{4.720071in}}%
\pgfpathlineto{\pgfqpoint{4.138224in}{4.730948in}}%
\pgfpathlineto{\pgfqpoint{4.286129in}{4.703754in}}%
\pgfpathlineto{\pgfqpoint{4.419926in}{4.737667in}}%
\pgfpathlineto{\pgfqpoint{4.542073in}{4.753984in}}%
\pgfpathlineto{\pgfqpoint{4.654437in}{4.752384in}}%
\pgfpathlineto{\pgfqpoint{4.758470in}{4.708553in}}%
\pgfpathlineto{\pgfqpoint{4.855323in}{4.786297in}}%
\pgfpathlineto{\pgfqpoint{4.945922in}{4.749185in}}%
\pgfpathlineto{\pgfqpoint{5.031027in}{4.777659in}}%
\pgfpathlineto{\pgfqpoint{5.111267in}{4.754624in}}%
\pgfpathlineto{\pgfqpoint{5.187166in}{4.794936in}}%
\pgfpathlineto{\pgfqpoint{5.259172in}{4.802934in}}%
\pgfpathlineto{\pgfqpoint{5.327664in}{4.782458in}}%
\pgfpathlineto{\pgfqpoint{5.392969in}{4.801654in}}%
\pgfpathlineto{\pgfqpoint{5.455371in}{4.762622in}}%
\pgfpathlineto{\pgfqpoint{5.515116in}{4.815092in}}%
\pgfpathlineto{\pgfqpoint{5.572422in}{4.830129in}}%
\pgfpathlineto{\pgfqpoint{5.627480in}{4.821170in}}%
\pgfpathlineto{\pgfqpoint{5.680460in}{4.776059in}}%
\pgfpathlineto{\pgfqpoint{5.731513in}{4.824050in}}%
\pgfpathlineto{\pgfqpoint{5.780775in}{4.815731in}}%
\pgfpathlineto{\pgfqpoint{5.828366in}{4.715911in}}%
\pgfpathlineto{\pgfqpoint{5.874396in}{4.840686in}}%
\pgfpathlineto{\pgfqpoint{5.918965in}{4.816371in}}%
\pgfpathlineto{\pgfqpoint{5.962163in}{4.837167in}}%
\pgfpathlineto{\pgfqpoint{6.004070in}{4.846445in}}%
\pgfpathlineto{\pgfqpoint{6.044763in}{4.851564in}}%
\pgfpathlineto{\pgfqpoint{6.084310in}{4.836527in}}%
\pgfpathlineto{\pgfqpoint{6.122772in}{4.852844in}}%
\pgfpathlineto{\pgfqpoint{6.160209in}{4.855723in}}%
\pgfpathlineto{\pgfqpoint{6.196674in}{4.853484in}}%
\pgfpathlineto{\pgfqpoint{6.232215in}{4.841646in}}%
\pgfpathlineto{\pgfqpoint{6.266879in}{4.847085in}}%
\pgfpathlineto{\pgfqpoint{6.300707in}{4.855723in}}%
\pgfpathlineto{\pgfqpoint{6.333739in}{4.872680in}}%
\pgfpathlineto{\pgfqpoint{6.366012in}{4.837487in}}%
\pgfpathlineto{\pgfqpoint{6.397560in}{4.869481in}}%
\pgfpathlineto{\pgfqpoint{6.428414in}{4.862442in}}%
\pgfpathlineto{\pgfqpoint{6.458604in}{4.872360in}}%
\pgfpathlineto{\pgfqpoint{6.488159in}{4.895715in}}%
\pgfpathlineto{\pgfqpoint{6.517104in}{4.846125in}}%
\pgfpathlineto{\pgfqpoint{6.545465in}{4.882598in}}%
\pgfpathlineto{\pgfqpoint{6.573264in}{4.861802in}}%
\pgfpathlineto{\pgfqpoint{6.600523in}{4.835887in}}%
\pgfpathlineto{\pgfqpoint{6.627263in}{4.885157in}}%
\pgfpathlineto{\pgfqpoint{6.653503in}{4.902754in}}%
\pgfpathlineto{\pgfqpoint{6.679262in}{4.896355in}}%
\pgfpathlineto{\pgfqpoint{6.704556in}{4.887717in}}%
\pgfpathlineto{\pgfqpoint{6.729403in}{4.895395in}}%
\pgfpathlineto{\pgfqpoint{6.753818in}{4.891236in}}%
\pgfpathlineto{\pgfqpoint{6.777815in}{4.905953in}}%
\pgfpathlineto{\pgfqpoint{6.801409in}{4.912992in}}%
\pgfpathlineto{\pgfqpoint{6.824613in}{4.901794in}}%
\pgfpathlineto{\pgfqpoint{6.847439in}{4.908513in}}%
\pgfpathlineto{\pgfqpoint{6.869901in}{4.919071in}}%
\pgfpathlineto{\pgfqpoint{6.892008in}{4.898595in}}%
\pgfpathlineto{\pgfqpoint{6.913773in}{4.902434in}}%
\pgfpathlineto{\pgfqpoint{6.935206in}{4.924190in}}%
\pgfpathlineto{\pgfqpoint{6.956316in}{4.910112in}}%
\pgfpathlineto{\pgfqpoint{6.977113in}{4.914592in}}%
\pgfpathlineto{\pgfqpoint{6.997607in}{4.915551in}}%
\pgfpathlineto{\pgfqpoint{7.017806in}{4.903714in}}%
\pgfpathlineto{\pgfqpoint{7.037719in}{4.922910in}}%
\pgfpathlineto{\pgfqpoint{7.057353in}{4.916511in}}%
\pgfpathlineto{\pgfqpoint{7.076716in}{4.920030in}}%
\pgfpathlineto{\pgfqpoint{7.095816in}{4.922270in}}%
\pgfpathlineto{\pgfqpoint{7.114659in}{4.912032in}}%
\pgfpathlineto{\pgfqpoint{7.133253in}{4.914592in}}%
\pgfpathlineto{\pgfqpoint{7.151603in}{4.905313in}}%
\pgfpathlineto{\pgfqpoint{7.169717in}{4.876519in}}%
\pgfpathlineto{\pgfqpoint{7.187600in}{4.938907in}}%
\pgfpathlineto{\pgfqpoint{7.205258in}{4.911392in}}%
\pgfpathlineto{\pgfqpoint{7.222697in}{4.927709in}}%
\pgfpathlineto{\pgfqpoint{7.239922in}{4.913952in}}%
\pgfpathlineto{\pgfqpoint{7.256938in}{4.937627in}}%
\pgfpathlineto{\pgfqpoint{7.273750in}{4.929309in}}%
\pgfpathlineto{\pgfqpoint{7.290363in}{4.943066in}}%
\pgfpathlineto{\pgfqpoint{7.306782in}{4.949465in}}%
\pgfpathlineto{\pgfqpoint{7.323011in}{4.941466in}}%
\pgfpathlineto{\pgfqpoint{7.339055in}{4.948185in}}%
\pgfpathlineto{\pgfqpoint{7.354917in}{4.923870in}}%
\pgfpathlineto{\pgfqpoint{7.370603in}{4.946585in}}%
\pgfpathlineto{\pgfqpoint{7.386114in}{4.941146in}}%
\pgfpathlineto{\pgfqpoint{7.401457in}{4.945625in}}%
\pgfpathlineto{\pgfqpoint{7.416633in}{4.945305in}}%
\pgfpathlineto{\pgfqpoint{7.431647in}{4.943386in}}%
\pgfpathlineto{\pgfqpoint{7.446502in}{4.946265in}}%
\pgfpathlineto{\pgfqpoint{7.461202in}{4.943066in}}%
\pgfpathlineto{\pgfqpoint{7.475749in}{4.942106in}}%
\pgfpathlineto{\pgfqpoint{7.490148in}{4.941466in}}%
\pgfpathlineto{\pgfqpoint{7.504399in}{4.961302in}}%
\pgfpathlineto{\pgfqpoint{7.518508in}{4.961302in}}%
\pgfusepath{stroke}%
\end{pgfscope}%
\begin{pgfscope}%
\pgfsetrectcap%
\pgfsetmiterjoin%
\pgfsetlinewidth{0.803000pt}%
\definecolor{currentstroke}{rgb}{0.000000,0.000000,0.000000}%
\pgfsetstrokecolor{currentstroke}%
\pgfsetdash{}{0pt}%
\pgfpathmoveto{\pgfqpoint{1.053750in}{2.022900in}}%
\pgfpathlineto{\pgfqpoint{1.053750in}{5.101226in}}%
\pgfusepath{stroke}%
\end{pgfscope}%
\begin{pgfscope}%
\pgfsetrectcap%
\pgfsetmiterjoin%
\pgfsetlinewidth{0.803000pt}%
\definecolor{currentstroke}{rgb}{0.000000,0.000000,0.000000}%
\pgfsetstrokecolor{currentstroke}%
\pgfsetdash{}{0pt}%
\pgfpathmoveto{\pgfqpoint{7.587000in}{2.022900in}}%
\pgfpathlineto{\pgfqpoint{7.587000in}{5.101226in}}%
\pgfusepath{stroke}%
\end{pgfscope}%
\begin{pgfscope}%
\pgfsetrectcap%
\pgfsetmiterjoin%
\pgfsetlinewidth{0.803000pt}%
\definecolor{currentstroke}{rgb}{0.000000,0.000000,0.000000}%
\pgfsetstrokecolor{currentstroke}%
\pgfsetdash{}{0pt}%
\pgfpathmoveto{\pgfqpoint{1.053750in}{2.022900in}}%
\pgfpathlineto{\pgfqpoint{7.587000in}{2.022900in}}%
\pgfusepath{stroke}%
\end{pgfscope}%
\begin{pgfscope}%
\pgfsetrectcap%
\pgfsetmiterjoin%
\pgfsetlinewidth{0.803000pt}%
\definecolor{currentstroke}{rgb}{0.000000,0.000000,0.000000}%
\pgfsetstrokecolor{currentstroke}%
\pgfsetdash{}{0pt}%
\pgfpathmoveto{\pgfqpoint{1.053750in}{5.101226in}}%
\pgfpathlineto{\pgfqpoint{7.587000in}{5.101226in}}%
\pgfusepath{stroke}%
\end{pgfscope}%
\end{pgfpicture}%
\makeatother%
\endgroup%


\section{Results}
\begin{markdown}
	
- The algorithm greatly depends on the initialization used for the model. For some models, incorrect initialization means we can't find anything. (DeepOBS mnist\_vae, Seed 45 failed to provide anything)
- Preconditioning for the largest two eigenvectors was no better than using only the adaptive step size.

\end{markdown}

\section{Discussion}
- 
\section{Further research/development}
- Make the process more robust using tests and catching exceptions and so on.
- Formulate a probabilistic version of the auto-restart feature.


\chapter{Conclusion}

\appendix %% Start the appendices.
\chapter{An appendix}
Here you can insert the appendices of your thesis.

\addcontentsline{toc}{chapter}{References}
\bibliographystyle{apalike}
\bibliography{bibliography}

\begin{otherlanguage}{ngerman}
	\chapter*{Selbstst\"andigkeitserkl\"arung}
	
	Hiermit versichere ich, dass ich die vorliegende Bachelorarbeit selbst\"andig und
	nur mit den angegebenen Hilfsmitteln angefertigt habe und dass alle Stellen,
	die dem Wortlaut oder dem Sinne nach anderen Werken entnommen sind,
	durch Angaben von Quellen als Entlehnung kenntlich gemacht worden sind.
	Diese Bachelorarbeit wurde in gleicher oder \"ahnlicher Form in keinem anderen
	Studiengang als Pr\"ufungsleistung vorgelegt.
	
	\vspace*{8ex}
	\hrule
	\vspace*{2ex}
	\noindent
	Tübingen, \today \hfill Ludwig Bald
\end{otherlanguage}

\end{document}