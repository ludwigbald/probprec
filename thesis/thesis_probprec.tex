%%%%%%%%%%%%%%%%%%%%%%%%%%%%%%%%%%%%%%%%%%%%%%%%%%%%%%%%%%%%%%%%%%%%%%%%%%%%%
%%% LaTeX-Rahmen fuer das Erstellen von Masterarbeiten
%%%%%%%%%%%%%%%%%%%%%%%%%%%%%%%%%%%%%%%%%%%%%%%%%%%%%%%%%%%%%%%%%%%%%%%%%%%%%

%%%%%%%%%%%%%%%%%%%%%%%%%%%%%%%%%%%%%%%%%%%%%%%%%%%%%%%%%%%%%%%%%%%%%%%%%%%%%
%%% allgemeine Einstellungen
%%%%%%%%%%%%%%%%%%%%%%%%%%%%%%%%%%%%%%%%%%%%%%%%%%%%%%%%%%%%%%%%%%%%%%%%%%%%%

\documentclass[twoside,12pt,a4paper]{report}
%\usepackage{reportpage}
\usepackage{epsf}
\usepackage{graphics, graphicx}
\usepackage{latexsym}
\usepackage[margin=10pt,font=small,labelfont=bf]{caption}
\usepackage[utf8]{inputenc}
\usepackage[toc,page]{appendix}
\usepackage{todonotes}
\usepackage[ngerman, english]{babel}

\textwidth 14cm
\textheight 22cm
\topmargin 0.0cm
\evensidemargin 1cm
\oddsidemargin 1cm
%\footskip 2cm
\parskip0.5explus0.1exminus0.1ex

% Kann von Student auch nach pers\"onlichem Geschmack ver\"andert werden.
\pagestyle{headings}

\sloppy

\begin{document}

%%%%%%%%%%%%%%%%%%%%%%%%%%%%%%%%%%%%%%%%%%%%%%%%%%%%%%%%%%%%%%%%%%%%%%%%%%%%
%%% hier steht die neue Titelseite 
%%%%%%%%%%%%%%%%%%%%%%%%%%%%%%%%%%%%%%%%%%%%%%%%%%%%%%%%%%%%%%%%%%%%%%%%%%%%

\begin{titlepage}
	\begin{center}
		{\LARGE Eberhard Karls Universit\"at T\"ubingen}\\
		{\large Mathematisch-Naturwissenschaftliche Fakult\"at \\
			Wilhelm-Schickard-Institut f\"ur Informatik\\[4cm]}
		{\huge Bachelor Thesis Cognitive Science\\[2cm]}
		{\Large\bf  Title of thesis\\[1.5cm]}
		{\large Ludwig Bald}\\[0.5cm]
		\today \\[4cm]
		\parbox{7cm}{\begin{center}{{\small\bf Gutachter}\\
				[0.5cm]\large Prof. Dr. Philipp Hennig}\\
				(Methoden des Maschinellen Lernens)\\
				{\footnotesize Wilhelm-Schickard-Institut f\"ur Informatik\\
				Universit\"at T\"ubingen}\end{center}}\hfill\parbox{7cm}
				{\begin{center}
				{\small\bf Betreuer}\\[0.5cm]
				{\large Filip De Roos}\\
				(Methoden des Maschinellen Lernens)\\
				{\footnotesize Wilhelm-Schickard-Institut f\"ur Informatik\\
					Universit\"at T\"ubingen}\end{center}
		}
	\end{center}
\end{titlepage}


%%%%%%%%%%%%%%%%%%%%%%%%%%%%%%%%%%%%%%%%%%%%%%%%%%%%%%%%%%%%%%%%%%%%%%%%%%%%
%%% Titelr"uckseite: Bibliographische Angaben
%%%%%%%%%%%%%%%%%%%%%%%%%%%%%%%%%%%%%%%%%%%%%%%%%%%%%%%%%%%%%%%%%%%%%%%%%%%%

\thispagestyle{empty}
\vspace*{\fill}
\begin{minipage}{11.2cm}
	\textbf{Bald, Ludwig:}\\
	\emph{Title of thesis}\\ Bachelor Thesis Cognitive Science\\
	Eberhard Karls Universit\"at T\"ubingen\\
	Thesis period: von-bis
\end{minipage}
\newpage

%%%%%%%%%%%%%%%%%%%%%%%%%%%%%%%%%%%%%%%%%%%%%%%%%%%%%%%%%%%%%%%%%%%%%%%%%%%%

\pagenumbering{roman}
\setcounter{page}{1}

%%%%%%%%%%%%%%%%%%%%%%%%%%%%%%%%%%%%%%%%%%%%%%%%%%%%%%%%%%%%%%%%%%%%%%%%%%%%
%%% Seite I: Zusammenfassug, Danksagung
%%%%%%%%%%%%%%%%%%%%%%%%%%%%%%%%%%%%%%%%%%%%%%%%%%%%%%%%%%%%%%%%%%%%%%%%%%%%


\section*{Abstract}
\todo{theabstract, citing!}
In machine learning, stochastic gradient descent is a widely used optimizsation algorithm, used to update the parameters of a model after a minibatch of data has been observed, in order to improve the model's predictions. It has been shown to converge much faster when the condition number (i.e. the ratio between the largest and the smallest eigenvalue) of ... is closer to 1. A preconditioner reduces the condition value 
In this thesis I present my implementation of the probabilistic preconditioning algorithm proposed in \cite{de2019active}. I use DeepOBS \todo{cite} as a benchmarking toolbox, examining the effect of this kind of preconditioning on various optimizers and test problems. 
The results...

\newpage
\section*{Zusammenfassung}
\todo{Abstract auf Deutsch}

\newpage
\section*{Acknowledgments}
If you have someone to Acknowledge ;)
\todo{Aaron, Filip}
\cleardoublepage

%%%%%%%%%%%%%%%%%%%%%%%%%%%%%%%%%%%%%%%%%%%%%%%%%%%%%%%%%%%%%%%%%%%%%%%%%%%%%
%%% Table of Contents
%%%%%%%%%%%%%%%%%%%%%%%%%%%%%%%%%%%%%%%%%%%%%%%%%%%%%%%%%%%%%%%%%%%%%%%%%%%%%

\renewcommand{\baselinestretch}{1.3}
\small\normalsize

\tableofcontents

\renewcommand{\baselinestretch}{1}
\small\normalsize

\cleardoublepage

%%%%%%%%%%%%%%%%%%%%%%%%%%%%%%%%%%%%%%%%%%%%%%%%%%%%%%%%%%%%%%%%%%%%%%%%%%%%%
%%% Der Haupttext, ab hier mit arabischer Numerierung
%%% Mit \input{dateiname} werden die Datei `dateiname' eingebunden
%%%%%%%%%%%%%%%%%%%%%%%%%%%%%%%%%%%%%%%%%%%%%%%%%%%%%%%%%%%%%%%%%%%%%%%%%%%%%

\pagenumbering{arabic}
\setcounter{page}{1}


\chapter{Introduction}
What is this all about?
%Cite like this: \cite{agarwal2011}


\chapter{Fundamentals and Related Rork}

\section{Deep learning}
\subsection{Artificial Neural Networks}
	Minibatched
\subsection{Regularization}
\subsection{Automatic Differentiation}
\subsection{Stochastic Gradient Descent}

\section{Something about Probabilistic things}

\section{Preconditioning}
Condition number vs Spectral Radius

The PROBLEM:

Up until now, there was no easy way to make use of preconditioning in a noisy setting, such as minibatched deep learning. I present an implementation of Filips Algorithm in an easy-to-use python class and demonstrate the algorithm's strengths and weaknesses.

\section{Benchmarking}
There are no standard established benchmarking protocols for new optimizers. It isn't even clear what measures to consider, or how they are to be measured. As a result, nobody knows which optimizers are actually good. And some bad optimizers will seem good.
DeepOBS is a solution to this problem, standardizing a protocol, providing benchmarks and standard test problems.

\section{Related Work}
\todo{in which other ways has this problem been adressed? (What even is the problem?)}




\chapter{Implementation}
\todo{High-Level to low-level details}

\section{Overview}

\section{Realization of the Test problems}

\section{Technical details}
The experiments were run on the TCML cluster at the University of Tübingen.
A Singularity container was set up on Ubuntu 16.4 LTS with python 3.5, pytorch (version) and DeepOBS (see Appendix for Singulariy recipe).
Computation was distributed over the compute nodes using the workload manager Slurm.


\chapter{Experiment}

\section{Results}

\section{Analysis}

\section{Discussion}


\chapter{Conclusion}


\appendix %% Start the appendices.
\chapter{An appendix}
Here you can insert the appendices of your thesis.

\addcontentsline{toc}{chapter}{References}
\bibliographystyle{apalike}
\bibliography{bibliography}

\begin{otherlanguage}{ngerman}
	\chapter*{Selbstst\"andigkeitserkl\"arung}
	
	Hiermit versichere ich, dass ich die vorliegende Bachelorarbeit selbst\"andig und
	nur mit den angegebenen Hilfsmitteln angefertigt habe und dass alle Stellen,
	die dem Wortlaut oder dem Sinne nach anderen Werken entnommen sind,
	durch Angaben von Quellen als Entlehnung kenntlich gemacht worden sind.
	Diese Bachelorarbeit wurde in gleicher oder \"ahnlicher Form in keinem anderen
	Studiengang als Pr\"ufungsleistung vorgelegt.
	
	\vspace*{8ex}
	\hrule
	\vspace*{2ex}
	\noindent
	Tübingen, \today \hfill Ludwig Bald
\end{otherlanguage}

\end{document}