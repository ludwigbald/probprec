If you decide that \LaTeX{} is too wordy for some parts of your document, there are \markdownRendererLink{packages}{https://www.ctan.org/pkg/markdown}{https://www.ctan.org/pkg/markdown}{Markdown} that allow you to use more lightweight markup next to it.\markdownRendererInterblockSeparator
{}\markdownRendererImage{logo}{fithesis/logo/mu/fithesis-base.pdf}{fithesis/logo/mu/fithesis-base.pdf}{The logo of the
  Masaryk University}\markdownRendererInterblockSeparator
{}This is a bullet list. Unlike numbered lists, bulleted lists contain an \markdownRendererStrongEmphasis{unordered} set of bullet points. When a bullet point contains multiple paragraphs, the list is typeset as follows:\markdownRendererInterblockSeparator
{}\markdownRendererUlBegin
\markdownRendererUlItem The first item of a bullet list\markdownRendererInterblockSeparator
{}that spans several paragraphs,\markdownRendererUlItemEnd 
\markdownRendererUlItem the second item of a bullet list,\markdownRendererUlItemEnd 
\markdownRendererUlItem the third item of a bullet list.\markdownRendererUlItemEnd 
\markdownRendererUlEnd \markdownRendererInterblockSeparator
{}When none of the bullet points contains multiple paragraphs, the list has a more compact form:\markdownRendererInterblockSeparator
{}\markdownRendererUlBeginTight
\markdownRendererUlItem The first item of a bullet list,\markdownRendererUlItemEnd 
\markdownRendererUlItem the second item of a bullet list,\markdownRendererUlItemEnd 
\markdownRendererUlItem the third item of a bullet list.\markdownRendererUlItemEnd 
\markdownRendererUlEndTight \markdownRendererInterblockSeparator
{}Unlike a bulleted list, a numbered list implies chronology or ordering of the bullet points. When a bullet point contains multiple paragraphs, the list is typeset as follows:\markdownRendererInterblockSeparator
{}\markdownRendererOlBegin
\markdownRendererOlItemWithNumber{1}The first item of an ordered list\markdownRendererInterblockSeparator
{}that spans several paragraphs,\markdownRendererOlItemEnd 
\markdownRendererOlItemWithNumber{2}the second item of an ordered list,\markdownRendererOlItemEnd 
\markdownRendererOlItemWithNumber{3}the third item of an ordered list.\markdownRendererOlItemEnd 
\markdownRendererOlItemWithNumber{4}If you are feeling lazy,\markdownRendererOlItemEnd 
\markdownRendererOlItemWithNumber{5}you can use hash enumerators as well.\markdownRendererOlItemEnd 
\markdownRendererOlEnd \markdownRendererInterblockSeparator
{}When none of the bullet points contains multiple paragraphs, the list has a more compact form:\markdownRendererInterblockSeparator
{}\markdownRendererOlBeginTight
\markdownRendererOlItemWithNumber{6}The first item of an ordered list,\markdownRendererOlItemEnd 
\markdownRendererOlItemWithNumber{7}the second item of an ordered list,\markdownRendererOlItemEnd 
\markdownRendererOlItemWithNumber{8}the third item of an ordered list.\markdownRendererOlItemEnd 
\markdownRendererOlEndTight \markdownRendererInterblockSeparator
{}Definition lists are used to provide definitions of terms. When a definition contains multiple paragraphs, the list is typeset as follows:\markdownRendererInterblockSeparator
{}\markdownRendererDlBegin
\markdownRendererDlItem{Term 1}\markdownRendererDlDefinitionBegin Definition 1\markdownRendererDlDefinitionEnd \markdownRendererDlItemEnd \markdownRendererDlItem{\markdownRendererEmphasis{Term 2}}\markdownRendererDlDefinitionBegin Definition 2\markdownRendererInterblockSeparator
{}\markdownRendererInputVerbatim{./_markdown_thesis_probprec/341d9e6b078ab6ffb77652958c88c4ef.verbatim}\markdownRendererInterblockSeparator
{}Third paragraph of Definition 2.\markdownRendererDlDefinitionEnd \markdownRendererDlItemEnd 
\markdownRendererDlEnd\markdownRendererInterblockSeparator
{}When none of the bullet points contains multiple paragraphs, the list has a more compact form:\markdownRendererInterblockSeparator
{}\markdownRendererDlBeginTight
\markdownRendererDlItem{Term 1}\markdownRendererDlDefinitionBegin Definition 1\markdownRendererDlDefinitionEnd \markdownRendererDlItemEnd \markdownRendererDlItem{\markdownRendererEmphasis{Term 2}}\markdownRendererDlDefinitionBegin Definition 2\markdownRendererDlDefinitionEnd \markdownRendererDlItemEnd 
\markdownRendererDlEndTight\markdownRendererInterblockSeparator
{}Block quotations are used to include an excerpt from an external document in way that visually clearly separates the excerpt from the rest of the work:\markdownRendererInterblockSeparator
{}\markdownRendererBlockQuoteBegin
This is the first level of quoting.\markdownRendererInterblockSeparator
{}\markdownRendererBlockQuoteBegin
This is nested blockquote.
\markdownRendererBlockQuoteEnd \markdownRendererInterblockSeparator
{}Back to the first level.
\markdownRendererBlockQuoteEnd \markdownRendererInterblockSeparator
{}Footnotes are used to include additional information to the document that are not necessary for the understanding of the main text. Here is a footnote reference\markdownRendererFootnote{Here is the footnote.} and another.\markdownRendererFootnote{Here's one with multiple blocks.\markdownRendererInterblockSeparator
{}Subsequent paragraphs are indented to show that they belong to the previous footnote.\markdownRendererInterblockSeparator
{}\markdownRendererInputVerbatim{./_markdown_thesis_probprec/384313ed77c3960a536af8343b57216d.verbatim}\markdownRendererInterblockSeparator
{}The whole paragraph can be indented, or just the first line. In this way, multi-paragraph footnotes work like multi-paragraph list items.}\markdownRendererInterblockSeparator
{}Citations are used to provide bibliographical references to other documents. This is a regular citation~\markdownRendererCite{1}+{}{p.\markdownRendererNbsp{}123}{borgman03}. This is an in-text citation: \markdownRendererTextCite{1}+{}{}{borgman03}. You can also cite several authors at once using both regular~\markdownRendererCite{3}+{see}{p.\markdownRendererNbsp{}123}{borgman03}+{}{sec.\markdownRendererNbsp{}3.2}{greenberg98}+{and}{}{thanh01} and in-text citations: \markdownRendererTextCite{3}+{}{p.123}{borgman03}+{}{sec.\markdownRendererNbsp{}3.2}{greenberg98}+{}{}{thanh01}.\markdownRendererInterblockSeparator
{}Code blocks are used to include source code listings into the document:\markdownRendererInterblockSeparator
{}\markdownRendererInputVerbatim{./_markdown_thesis_probprec/b340ab97900e0188d144f54df648acc3.verbatim}\markdownRendererInterblockSeparator
{}There is an alternative syntax for code blocks that allows you to specify additional information, such as the language of the source code. This information can be used for syntax highlighting:\markdownRendererInterblockSeparator
{}\markdownRendererInputFencedCode{./_markdown_thesis_probprec/0b47e3a0da462ec457b209465e3bbfab.verbatim}{sh}\markdownRendererInterblockSeparator
{}\markdownRendererInputFencedCode{./_markdown_thesis_probprec/cd8890f759f5208a623143b4bbfe664b.verbatim}{Ruby}\relax