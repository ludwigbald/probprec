%%%%%%%%%%%%%%%%%%%%%%%%%%%%%%%%%%%%%%%%%%%%%%%%%%%%%%%%%%%%%%%%%%%%
%% I, the copyright holder of this work, release this work into the
%% public domain. This applies worldwide. In some countries this may
%% not be legally possible; if so: I grant anyone the right to use
%% this work for any purpose, without any conditions, unless such
%% conditions are required by law.
%%%%%%%%%%%%%%%%%%%%%%%%%%%%%%%%%%%%%%%%%%%%%%%%%%%%%%%%%%%%%%%%%%%%

\documentclass[
  printed, %% This option enables the default options for the
           %% digital version of a document. Replace with `printed`
           %% to enable the default options for the printed version
           %% of a document.
  twoside, %% This option enables double-sided typesetting. Use at
           %% least 120 g/m² paper to prevent show-through. Replace
           %% with `oneside` to use one-sided typesetting; use only
           %% if you don’t have access to a double-sided printer,
           %% or if one-sided typesetting is a formal requirement
           %% at your faculty.
  table,   %% This option causes the coloring of tables. Replace
           %% with `notable` to restore plain LaTeX tables.
  nolof,     %% This option prints the List of Figures. Replace with
           %% `nolof` to hide the List of Figures.
  nolot,     %% This option prints the List of Tables. Replace with
           %% `nolot` to hide the List of Tables.
  %% More options are listed in the user guide at
  %% <http://mirrors.ctan.org/macros/latex/contrib/fithesis/guide/mu/fi.pdf>.
]{fithesis3}
\include{header}


%% The following section sets up the locales used in the thesis.
\usepackage[resetfonts]{cmap} %% We need to load the T2A font encoding
\usepackage[T1,T2A]{fontenc}  %% to use the Cyrillic fonts with Russian texts.
\usepackage[
  main=english, %% By using `czech` or `slovak` as the main locale
                %% instead of `english`, you can typeset the thesis
                %% in either Czech or Slovak, respectively.
  english, ngerman, russian, czech, slovak %% The additional keys allow
]{babel}        %% foreign texts to be typeset as follows:
%%
%%   \begin{otherlanguage}{german}  ... \end{otherlanguage}
%%   \begin{otherlanguage}{russian} ... \end{otherlanguage}
%%   \begin{otherlanguage}{czech}   ... \end{otherlanguage}
%%   \begin{otherlanguage}{slovak}  ... \end{otherlanguage}
%%
%% For non-Latin scripts, it may be necessary to load additional
%% fonts:
\usepackage{paratype}
\def\textrussian#1{{\usefont{T2A}{PTSerif-TLF}{m}{rm}#1}}
%%
%% The following section sets up the metadata of the thesis.
\thesissetup{
	autoLayout    = false,
    date          = \the\year/\the\month/\the\day,
    university    = mu,
    faculty       = fi,
    type          = bc,
    author        = Ludwig Bald,
    gender        = m,
    advisor       = Philipp Hennig \hfill \textbf{Supervisor:} Filip De Roos,
    title         = {The Proof of P = NP},
    TeXtitle      = {The Proof of $\mathsf{P}=\mathsf{NP}$},
    keywords      = {keyword1, keyword2, ...},
    TeXkeywords   = {keyword1, keyword2, \ldots},
    abstract      = { },
	assignment	  = {}, %include additional files
	declaration   = {\begin{otherlanguage}{ngerman}
					Hiermit erkläre ich, dass ich diese schriftliche Abschlussarbeit selbständig verfasst habe,
					keine anderen als die angegebenen Hilfsmittel und Quellen benutzt habe und alle wörtlich
					oder sinngemäß aus anderen Werken übernommenen Aussagen als solche gekennzeichnet habe.
					\end{otherlanguage}
					},
    thanks        = {%
      These are the acknowledgements for my thesis, which can

      span multiple paragraphs.
    },
}
\usepackage{makeidx}      %% The `makeidx` package contains
\makeindex                %% helper commands for index typesetting.
%% These additional packages are used within the document:
\usepackage{paralist} %% Compact list environments
\usepackage{amsmath}  %% Mathematics
\usepackage{amsthm}
\usepackage{amsfonts}
\usepackage{url}      %% Hyperlinks
\usepackage{markdown} %% Lightweight markup
\usepackage{tabularx} %% Tables
\usepackage{tabu}
\usepackage{booktabs}
\usepackage{listings} %% Source code highlighting
\lstset{
  basicstyle      = \ttfamily,
  identifierstyle = \color{black},
  keywordstyle    = \color{blue},
  keywordstyle    = {[2]\color{cyan}},
  keywordstyle    = {[3]\color{olive}},
  stringstyle     = \color{teal},
  commentstyle    = \itshape\color{magenta},
  breaklines      = true,
}
\usepackage{floatrow} %% Putting captions above tables
\floatsetup[table]{capposition=top}
%% The following code fixes the rendering of BibLaTeX ISO 690
%% references in old TeX Live (such as the one at Overleaf).
\thesisload
\makeatletter
\def\thesis@biblatexiso@fix@package{iso-numeric.bbx}
\def\thesis@biblatexiso@fix@end{\relax}
\newif\ifthesis@biblatexiso@fix@
\thesis@biblatexiso@fix@false
\def\thesis@biblatexiso@fix@next#1,{%
  \def\thesis@biblatexiso@fix@current{#1}%
  \ifx\thesis@biblatexiso@fix@current\thesis@biblatexiso@fix@package
    \thesis@biblatexiso@fix@true
  \fi
  \ifx\thesis@biblatexiso@fix@current\thesis@biblatexiso@fix@end
    \expandafter
    \@gobble
  \fi
  \thesis@biblatexiso@fix@next
}
\expandafter\expandafter\expandafter\thesis@biblatexiso@fix@next\@filelist,\relax,
\ifthesis@biblatexiso@fix@
  \defbibenvironment{bibliography}
    {\list%
       {\MethodFormat}%
       {\setlength{\labelwidth}{\labelnumberwidth}%
        \setlength{\leftmargin}{\labelwidth}%
        \setlength{\labelsep}{\biblabelsep}%
        \addtolength{\leftmargin}{\labelsep}%
        \setlength{\itemsep}{\bibitemsep}%
        \setlength{\parsep}{\bibparsep}}%
        \renewcommand*{\makelabel}[1]{\hss##1}
        }%
    {\endlist}%
  {\item}%
\fi
\makeatother



\begin{document}
\makeatletter
	\def \thesis@blocks@preamble {\begin{tabular}{lr}
% \includegraphics[width=0.5\linewidth]{logo_sw} % logo bw
 \includegraphics[width=0.5\linewidth]{UT_WBMW_Rot_4C} % logo red
 & \hspace{0.2\linewidth}
 \parbox{0.5\linewidth}{
   \large\bf\textsf{\color{rot}{Mathematisch-\\Naturwissenschaftliche\\Fakultät\\\\
   	\normalsize Methoden des \\ Maschinellen Lernens}}
   \vspace{0.6cm}
 }
\end{tabular}

\vspace*{10ex}
Bachelorarbeit \\

{\huge\bf\textsf{Title of Thesis}}

\vspace*{20ex}

Eberhard Karls Universität Tübingen\\
Mathematisch-Naturwissenschaftliche Fakultät\\
Wilhelm-Schickard-Institut für Informatik\\
Methoden des Maschinellen Lernens\\
Ludwig Bald,~ \verb+ludwig.bald@student.uni-tuebingen.de+,~ 2019

\vspace*{5ex}

\begin{tabular}{@{}l@{\hspace{2em}}l}
  Bearbeitungszeitraum:& von-bis \vspace*{5ex} \\
  Betreuer:& Filip De Roos, Universität Tübingen\\
  Gutachter:& Prof. Dr. Philipp Hennig, Universität Tübingen
\end{tabular}

\thispagestyle{empty}
\newpage

\thesis@blocks@frontMatter

\chapter*{Selbstst\"andigkeitserkl\"arung}
Hiermit versichere ich, dass ich die vorliegende Bachelorarbeit selbst\"andig und
nur mit den angegebenen Hilfsmitteln angefertigt habe und dass alle Stellen,
die dem Wortlaut oder dem Sinne nach anderen Werken entnommen sind,
durch Angaben von Quellen als Entlehnung kenntlich gemacht worden sind.
Diese Bachelorarbeit wurde in gleicher oder \"ahnlicher Form in keinem anderen
Studiengang als Pr\"ufungsleistung vorgelegt.

\vspace*{8ex}
\hrule
\vspace*{2ex}
Ludwig Bald (Matrikelnummer 4125813), \today

\chapter*{Abstract}
\todo{theabstract, citing!}
In machine learning, stochastic gradient descent is a widely used optimizsation algorithm, used to update the parameters of a model after a minibatch of data has been observed, in order to improve the model's predictions. It has been shown to converge much faster when the condition number (i.e. the ratio between the largest and the smallest eigenvalue) of ... is closer to 1. A preconditioner reduces the condition value 
In this thesis I present my implementation of the probabilistic preconditioning algorithm proposed in \cite{de2019active}. I use DeepOBS \todo{cite} as a benchmarking toolbox, examining the effect of this kind of preconditioning on various optimizers and test problems. 
The results...


\chapter*{Acknowledgments}
If you have someone to Acknowledge ;)
\todo{Aaron, Filip}



								  %\thesis@blocks@frontMatter %styling, moved to Teaser
						  	  	  \thesis@blocks@toc}
	\thesis@preamble
\makeatother

\chapter{Introduction}
What is this all about?
%Cite like this: \cite{agarwal2011}


\chapter{Fundamentals and related work}


\chapter{Implementation}
\todo{High-Level to low-level details}

\section{Overview}

\section{Realization of the Test problems}

\section{Technical details}
The experiments were run on the TCML cluster at the University of Tübingen.
A Singularity container was set up on Ubuntu 16.4 LTS with python 3.5, pytorch (version) and DeepOBS (see Appendix for Singulariy recipe).
Computation was distributed over the compute nodes using the workload manager Slurm.


\chapter{Experiment}

\section{Results}

\section{Analysis}

\section{Discussion}


\chapter{Conclusion}


%  \makeatletter\thesis@blocks@clear\makeatother
%  \phantomsection %% Print the index and insert it into the
%  \addcontentsline{toc}{chapter}{\indexname} %% table of contents.
%  \printindex

\appendix %% Start the appendices.
\chapter{An appendix}
Here you can insert the appendices of your thesis.

\makeatletter\thesis@blocks@clear\makeatother
\phantomsection
\addcontentsline{toc}{chapter}{References}
\bibliographystyle{apalike}
\bibliography{bibliography}
\end{document}